\begin{savequote}[8cm]
  ``It is not knowledge, but the act of learning, not possession but the act of getting there, which grants the greatest enjoyment''
  \qauthor{Carl Friedrich Gauss}
\end{savequote}
\makeatletter
\chapter{Introduction}



\section{Introduction}

In recent years, there has been an increase in demand for research and applications within the areas of human computer interaction, these include: virtual reality, augmented reality and 3D reconstruction. The research presented in this thesis is about 3D reconstruction. 3D Reconstruction is concerned with building dense 3D models out of various sensor input such as monocular video data and active sensor video data. This area is very similar to Simultaneous Localization and Mapping or SLAM. However, the topics are differentiated because SLAM research has not historically focused solely on producing dense 3D reconstructions. SLAM also has a focus on accurate camera localization. In 3D reconstruction, as long as pleasant, dense and useful 3D reconstructions are computed, accurate scans do not matter. \\

3D reconstruction is important in a wide variety of areas including business, engineering and architecture, virtual reality and augmented reality. For example, an architect may want to record 3D structural data in order to study it later. An engineer may want to study the under area of a bridge in order to assess the presence of possible faults. Additionally, a software engineer may want to create an augmented reality application where possible home buyers take virtual tours through an existing property. The specifications and 3D structures of these areas may be recorded, in which case 3D models may be built by artists. This however, costs both time and money, furthermore there may be no existing blueprints for the structures, we may also want to provide virtual walking tours through a rainforest, this definitely has no blueprint. \\

Currently, there exist many methods which are capable of extracting 3D reconstructions and computing camera pose information from image and video data. Methods include: the Fundamental matrix method, Feature Matching (FM) (both 2D and 3D) with the Random Sample Consensus (RANSAC) algorithm, Principal Components Analysis (PCA) and Iterative Closest Point (ICP). There also exist other forms of non-linear optimization, but these are the techniques used in the state of the art methods. These algorithms work well in simplistic environments where features are plentiful and data noise/corruption is of no concern. The Fundamental matrix method is used in the case of monocular video data, it requires camera calibration and currently cannot be used to generate accurate dense 3D data. Feature Matching methods (both 2D and 3D) used with RANSAC may be used in conjunction with an active camera to generate dense 3D reconstructions. ICP may also be used in the same way. The problem with these methods is that they are not robust to noise, they also tend to fail in low-textured scenes and cannot register wide-baselines. The Feature matching methods also fail when texture confusion occurs. Texture confusion occurs in scenes with many features which look identical locally but appear in different locations. PCA may also be used in conjunction with an active camera to register 3D frames to create 3D reconstructions. Whilst PCA can handle wider base-lines, it is more susceptible to noise. Furthermore, these iterative methods have complexity which depends on the input data. \\


During research in 3D reconstruction, it became apparent that 3D reconstruction data uses large amounts of storage. Since 3D reconstruction algorithms may be storing multiple 3D frames, and may also use some global representation for output, a compression strategy would be able to improve data processing, transmission and storage. Whilst lossless compression research for such data has begun, lossy compression is still unexplored. Therefore, research on a lossy compression system for 3D data is also presented in this thesis. \\


\section{Research Aims \& Contributions}

In the current state of research, Fourier based methods have not been investigated. Whilst Fourier based registration techniques have proven to be reliable, accurate and robust to noise within the context of 2D registration, research on 3D registration in the context of 3D reconstruction does not exist. It is benefit to use an approach based on Fourier techniques as they are robust to noise, accurate and work on wide base-lines. Fourier based approaches are also closed form solutions, in that their complexity does not depend on their input data. Additionally, lossy compression methods based on the hierarchical Octree data structure and modifications of it \cite{Lincoln13Interpolating} were investigated for compression of 3D data including 3D reconstruction frames and outputs. \\

The primary aim of this research is to improve the accuracy, noise robustness, speed and storage, quantitative quality and perceptual quality of 3D models generated from image data. To this end, Fourier based registration schemes were investigated as well as hierarchical data compression systems. This motivated the research question, ``Can Fourier based registration techniques improve accuracy and noise robustness in 3D reconstruction applications?'' and ``Can hierarchical techniques improve compression, storage and processing of 3D reconstruction data?'' \\


The quest to answer these questions has let to several novel algorithms and data structures being proposed. These are presented and discussed in chapter \ref{ch:Metho}. Firstly, a novel method applying Fourier based registration to 3D reconstruction named Fourier Volume Reconstruction (FVR) is proposed \cite{Lincoln16Fourier,Lincoln16Dense}. This algorithm is shown to be able to generate accurate dense 3D reconstructions whilst being robust to noise and moving objects within the scene (see sections \ref{Sec:CamTransTrackExp,Sec:CamRoteTrackExp,Sec:FVRMotionExp,Sec:FVRQual1Exp}). The FVR method works well when provided dense 3D frames in which to reconstruct a global model from, to allow Fourier based methods to register monocular video data a novel algorithm based on FVR was proposed named Monocular View Volume Reconstruction (MVVR) \cite{Lincoln16Monocular}. This method was shown to be able to accurately register camera translations whilst being robust to noise (see section \ref{Sec:MonocularExperimentsSection} for results). \\

Since, the FVR method is only able to register against a single axis of camera rotation, a novel method which incorporates PCA into FVR is proposed. This method, named FVR-3D is shown to outperform several pose registration methods used in state of the art 3D reconstruction algorithms in terms of accuracy and robustness (see section \ref{Sec:FVRSOTA} for results). \\

In terms of compression research, a novel data representation based on Octree 3D data compression is proposed named the Plane-Tree data representation \cite{Lincoln15Plane}. This data representation and subsequent compression and decompression algorithm are shown to outperform the generic Octree as well as several state of the art 3D data compression algorithms. \\


\section{Overview}

This thesis is laid out as follows: chapter two presents a survey of techniques which are critical to different 3D reconstruction techniques. Following this, chapter three introduces the proposed techniques. These are used to answer the research questions and accomplish the primary aim of this project. The fourth chapter details the experiments performed, and presents quantitative results showing the effectiveness of these techniques. The fifth chapter contains an analysis of these results and presents findings and discusses results. Finally, the sixth chapter concludes the thesis and discusses the results in terms of the primary aim and research question. 


