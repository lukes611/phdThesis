\makeatletter
\chapter{Introduction}

\section{Introduction}

In recent years, there has been increasing demand for research and applications within the areas of human-computer interaction, including: virtual reality, augmented reality and 3D reconstruction. This thesis presents research on 3D reconstruction. 3D reconstruction is concerned with building dense 3D models out of various sensor input, such as video recorded with a stereo camera set-up, active sensor video data and monocular camera video data. This area is very similar to Simultaneous Localization and Mapping or SLAM. However, the topics differ in that SLAM research has not historically focused solely on producing dense 3D reconstructions. SLAM also has a focus on accurate camera localization. In 3D reconstruction, as long as pleasant, dense and useful 3D reconstructions are computed, accurate scans and camera localization do not matter. \\

3D reconstruction is important in a wide variety of fields including business, engineering and architecture, virtual reality and augmented reality. For example, an architect may want to record 3D structural data for later study. An engineer may want to study the under area of a bridge to assess the presence of possible faults while a software engineer may want to create an augmented reality application where possible home buyers take virtual tours through an existing property. Of course, the specifications and 3D structures of these areas may be reverse engineered and recorded. In this case, 3D models may be manually built by artists or engineers. This however, costs both time and money, furthermore some structures have no blueprint. For example, we may also want to provide virtual walking tours through a rainforest. This definitely has no blueprint. \\

Currently, there exist many methods capable of extracting 3D reconstructions and computing camera pose information from image and video data. Methods include: the Fundamental matrix method, Feature Matching (FM) (both 2D and 3D) with the Random Sample Consensus (RANSAC) algorithm, Principal Components Analysis (PCA) and Iterative Closest Point (ICP). There also exist other forms of non-linear optimization, however these algorithms are used most often in state-of-the-art methods. These algorithms work well in simplistic environments where features are plentiful and data noise/corruption is of no concern. The Fundamental matrix method is used in the case of monocular video data, it requires camera calibration and currently cannot be used to generate accurate dense 3D data. Feature Matching methods (both 2D and 3D) used with RANSAC may be used in conjunction with an active or stereo camera set-up to generate dense 3D reconstructions. ICP may also be used in the same way. The problem with these methods is that they are not robust to noise. They also tend to fail in low-textured scenes and cannot register wide-baselines. Feature Matching based methods also fail when texture confusion occurs. Texture confusion occurs in scenes with many features which look identical locally but appear in different locations spatially, throwing off registration calculations. PCA may also be used in conjunction with an active camera or stereo camera set-up to register 3D frames to create 3D reconstructions. Whilst PCA can handle wider base-lines, it is more susceptible to noise. These iterative methods have complexity which depends on the input data, furthermore they may also get stuck in a local minimum prior to finding an optimal registration. \\


During 3D reconstruction research, it became apparent that 3D reconstruction data uses large amounts of storage. Since 3D reconstruction algorithms may be storing multiple 3D frames, and may also use some global representation for output, a compression could improve data processing, transmission and storage. Whilst lossless compression research for such data has begun, lossy compression remains unexplored. Therefore, research on a lossy compression system for 3D data is also presented in this thesis. \\


\section{Research Aims}

In the current state of 3D reconstruction research, Fourier based methods have not been investigated. Research has proven 2D Fourier based registration techniques able to register wide base-lines and to be reliable, accurate and robust to noise, but in the context of 3D registration, such research does not exist. It is beneficial to use Fourier techniques as they are robust to noise, accurate and work well on wide base-lines. Fourier based approaches are also closed form solutions, in that their complexity does not depend on their input data only upon the dimension sizes of the data. In the case of compression systems for 3D reconstruction purposes, lossy compression methods based on the hierarchical octree data structure and modifications \cite{Lincoln13Interpolating} were investigated for compression of 3D data including 3D reconstruction frames and outputs. \\

The primary aim of this research is to improve the accuracy, noise robustness, speed and storage, quantitative quality and perceptual quality of 3D models generated from image data. To this end, Fourier based registration schemes were investigated as well as hierarchical data compression systems. Investigation into Fourier methods motivated the research questions, \\

1. ``Can Fourier based registration techniques improve accuracy and noise robustness in 3D reconstruction applications?'' \\

and \\

2. ``Can hierarchical techniques improve compression, storage and processing of 3D reconstruction data?'' \\

Answering these questions would contribute several missing links in the research field. Firstly, since using Fourier based techniques for 3D reconstruction has not been tried before, researchers will be able to know whether to investigate such techniques further, not only in 3D reconstruction but in other emerging fields such as virtual reality and augmented reality. Next, by investigating the second research question will not only product one or several new hierarchical compression methods, but if proven true, encourages further investigation into hierarchical compression methods which may further improve upon the state-of-the-art.

\section{Research Contributions}
\label{sec:INTO_RESEARCH_CONTRIBUTIONS}
Efforts to answer these questions have led to several novel algorithms and data structures being proposed. These are presented and discussed in Chapter \ref{ch:Metho}. Firstly, a novel method applying Fourier based registration to 3D reconstruction named Fourier Volume Reconstruction (FVR) is proposed \cite{Lincoln16Fourier,Lincoln16Dense}. This algorithm is able to generate accurate and dense 3D reconstructions whilst being robust to noise (see sections \ref{Sec:CamTransTrackExp} and \ref{Sec:FVRSOTA}). The FVR method works well when provided with dense 3D frames from which to reconstruct a global model. The FVR method may be used to reconstruct data input via a RGB-D active camera or stereo camera set-up. To provide support for monocular camera sensors, a novel algorithm based on FVR was proposed named Monocular View Volume Reconstruction (MVVR) \cite{Lincoln16Monocular}. Our research suggests, the higher the quality of the input depth data (produced via active, stereo or software based methods), the better the registration produced by FVR (section \ref{Sec:CamTransTrackExp}). \\

Since, the FVR method is only able to register against a single axis of camera rotation, a novel method which incorporates PCA into FVR is proposed. This method named FVR-3D outperforms several pose registration methods used in state-of-the-art 3D reconstruction algorithms using different input sensors (stereo \& active) in terms of accuracy and robustness (see section \ref{Sec:FVRSOTA} for results). A novel speed-up for the FVR algorithm is also proposed (section \ref{Sec:Efficiency}). This method, named Fast Fourier Volume Reconstruction (FFVR) improves efficiency by reducing the data to be processed by an entire dimension. Our results suggest this method improves upon the FVR method at the expense of registration accuracy. \\

In terms of compression research, a novel data representation based on octree 3D data compression is proposed named the Plane-Tree data representation \cite{Lincoln15Plane}. This data representation and subsequent compression and decompression algorithm outperform the generic octree as well as several state-of-the-art 3D data compression algorithms. \\

\section{Publications Resulting from Thesis}

A number of research publications resulted from the completion of this thesis. The contributions within these research publications were discussed in section \ref{sec:INTO_RESEARCH_CONTRIBUTIONS} above, but the publications themselves are listed below for reference.

\begin{enumerate}
  \item Fourier Volume Registration based Dense 3D Mapping \cite{Lincoln16Fourier}
  \item Dense 3D Mapping Using Volume Registration \cite{Lincoln16Dense}
  \item Dense 3D Mapping Using Volume Reigstration from Monocular View \cite{Lincoln16Monocular}
  \item Plane-Tree Low-Bitrate Mesh Compression \cite{Lincoln15Plane}
\end{enumerate}

\section{Thesis Structure}

This thesis is laid out as follows: Chapter \ref{ch:TheLiteratureReviewChapter} presents a review of techniques which are critical to different 3D reconstruction techniques. Following this, Chapter \ref{ch:Metho} introduces the proposed techniques used in this research project. These techniques are used to answer the research questions and accomplish the primary aim of this project. Chapter \ref{ch:Experiments} details the experiments performed, and presents quantitative results showing the effectiveness of the techniques. Chapter \ref{ch:Conclusion} concludes the thesis with an analysis and discussion of the results and by discussing the results in terms of the primary aim and research questions. The results of this study bring Fourier based reconstruction methods into frame and show them to be capable of state-of-the-art 3D reconstruction and registration. These results also provide a platform for future Hierarchical compression methods and show that such methods outperform state-of-the-art compression systems.

\section{Summary}

This chapter provided an introduction to 3D reconstruction and how it differs from the SLAM area. Some of the many possible applications of 3D reconstruction were listed and a discussion of how further research could benefit such applications was presented. The current 3D reconstruction methods rely on iterative techniques which have several shortcomings in that they are not robust to noise, they only work when features are plentiful and accurate and they may get stuck in local minima. Additionally, 3D reconstructions require large storage spaces. To solve these problems, two research questions were presented aiming to answer 1. whether Fourier based 3D reconstruction could improve upon the shortcomings and 2. whether hierarchical methods could improve the compression and transmission of 3D reconstruction data. Several contributions made in this thesis were highlighted and a list of publications was provided. In chapter \ref{ch:TheLiteratureReviewChapter} a review of the current techniques within 3D reconstruction and compression are presented. \\



