\begin{savequote}[8cm]
  ``It is not knowledge, but the act of learning, not possession but the act of getting there, which grants the greatest enjoyment''
  \qauthor{Carl Friedrich Gauss}
\end{savequote}
\makeatletter
\chapter{Introduction}



\section{Introduction}

In recent years, there has been a resurgence of research and applications within the areas of human computer interaction, these include: virtual reality, augmented reality and 3D reconstruction. This thesis concentrates on the area of 3D reconstruction. In this domain, image and video data is processed to collect 3D structural information. This information has many applications in virtual reality, engineering, architecture and business. Currently, there exist many methods which are capable of extracting 3D structural information from image and video data. Methods include: the Fundamental matrix method, Feature Matching and the Random Sample Consensus (RANSAC) algorithm, Principal Components Analysis (PCA), Iterative Closest Point (ICP) as well as other forms of non-linear optimization. These algorithms work well in simplistic environments where features are plentiful and data noise/corruption is of no concern. \\

In this thesis, experiments prove that a closed form solution may be found in the use of Fourier techniques. Fourier based registration may be used to register 3D data captured from scenes at different camera poses, and may be used to compute the camera pose change between two frames. This approach is shown to be robust to noise, object movement and lack of texture. The original technique applied to 3D data can solve up to 1 axis of rotation as well as scale and translation. In cases where 3D rotation must be registered, a novel PCA/Fourier registration method is proposed. Experiments show that in terms of registration accuracy, this method improves over ICP, feature matching (2D/3D) and PCA by itself. This is especially true in the presence of noise and in scenes which lack texture. \\

During research, it became obvious that 3D reconstruction data uses large amounts of storage. Whilst lossless compression research for such data has begun, lossy compression is still unexplored. During research, a novel 3D compression method was developed for such a case. Results show this outperforms several state of the art methods in terms of low bit rate mesh compression and also outperforms existing methods for representing 3D reconstructions. \\

3D Reconstruction research requires the development, testing and analysis of functions which input video and image data and output 3D reconstructed environments. This area is very similar to Simultaneous Localization and Mapping or SLAM. However, we have differentiated both topics because SLAM research has not historically focused solely on producing dense 3D reconstructions. SLAM also has a focus on accurate camera localization. In 3D reconstruction, as long as pleasant, dense and useful 3D reconstructions are computed, accurate scans do not matter. \\

3D reconstruction is important in a wide variety of areas including business, engineering and architecture, virtual reality and augmented reality. For example, an architect may want to record 3D structural data in order to study it later. An engineer may want to study the under area of a bridge in order to assess the presence of possible faults. Additionally, a software engineer may want to create an augmented reality application where possible home buyers take virtual tours through an existing property. The specifications and 3D structures of these areas may be recorded, in which case 3D models may be built by artists. This however, costs both time and money, furthermore there may be no existing blueprints for the structures, we may also want to provide virtual walking tours through a rainforest, this definitely has no blueprint. \\

Using image and video capturing hardware coupled with 3D reconstruction software, we would be able to scan in an environment and generate a dense 3D model of it for use in engineering/architectural analysis as well as any kind of virtual or augmented reality application. Furthermore, autonomous navigation systems may also generate and use this information as they navigate through a previously unknown environment. Furthermore, with the recent progress made in 3D printing, we may come across a situation where 3D objects and environments may be scanned in using 3D reconstruction and copied via 3D printing. \\


Without such a system, artists, architects, engineers or scientists would have to build, draw or find some alternative means of generating the 3D data they require. This would be a manual process costing much time and money.

As in other areas of image processing, 3D reconstruction is dominated by feature matching and RANSAC techniques. This involves computing matches between 2D pixels across images, these matches are typically used with RANSAC to compute a camera pose relationship between frames. 3D data is then projected and registered using the relationship. This approach is efficient but is not robust to data noise. Furthermore, without some outlier removal function to filter matches, this method fails as feature detection methods typically over-computes matches to the point that only around 30 - 50 percent of features are actually matched. This method also fails in cases where large baselines are used because affine distortion is too large, feature confusion occurs or any other time feature matching fails. Another popular method ICP can be used. This method is often in place of feature matching and RANSAC as a slower but more accurate solution. However, it is also susceptible to failure in scenes where texture is lacking. It also fails when the search iterations become trapped in a local minima in terms of matching error. This typically occurs when registering wide-baseline images. \\

Despite the apparent flaws in these methods, they are still popular in both research and industry. In image processing however, alternatives exist. One in particular: Fourier based registration works well at computing 2D rotation, scale and translation. The benefits of this technique are that it is robust to noise and outliers as it takes into account the full signal. It uses the frequency domain to perform fast correlation of data. It is a closed form solution (its speed does not depend on the amount of features or on the data itself) it also lends itself more easily to parallel processing in comparison with its iterative peers. As noted by both researchers and industry experts, parallel processing is a paradigm which is set to take over the next generation of both software and hardware. Such techniques are frequently used in medical image processing. This raises the question of how well such a technique or related set of techniques would apply to the area of 3D reconstruction. \\

During the research conducted in quest of answering this question, it was found that the storage and thus manipulation of 3D data became a bottleneck for database and network based operations on the 3D reconstructed data. Research into lossless compression has begun in this area. To alleviate the issue and add to knowledge within the area, lossy compression techniques were researched. A novel lossy technique for compression and transmission of 3D data is thus proposed in this work.  \\

\section{Research Aims \& Contributions}

The primary aim of this research is to improve the accuracy, noise robustness, speed and storage, quantitative quality and perceptual quality of 3D models generated from image data. To this end, Fourier based registration schemes were investigated as well as data compression systems. This motivated the research question, ``Can Fourier based registration techniques improve accuracy and noise robustness in 3D reconstruction applications?'' and ``Can hierarchical techniques improve compression, storage and processing of 3D reconstruction data?'' 

The quest to answer these questions has led to new 3D registration techniques \cite{Lincoln16Fourier,Lincoln16Dense,Lincoln16Monocular} which outperform ICP and feature matching approaches in terms of noise robustness and accuracy. A novel compression scheme was also developed \cite{Lincoln13Interpolating,Lincoln15Plane} which was shown to be capable of outperforming existing schemes, this method may be applied to 3D reconstruction storage and retrieval applications and research.

\section{Overview}

This thesis is laid out as follows: chapter two presents a survey of techniques which are critical to different 3D reconstruction techniques. Following this, chapter three introduces the proposed techniques. These are used to answer the research questions and accomplish the primary aim of this project. The fourth chapter details the experiments performed, and presents quantitative results showing the effectiveness of these techniques. The fifth chapter contains an analysis of these results and presents findings and discusses results. Finally, the sixth chapter concludes the thesis and discusses the results in terms of the primary aim and research question. 


