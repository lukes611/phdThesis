

\subsection{Camera Movement Tests}

In this section we present registration results for the FVR algorithm. Both camera translations and rotations were recorded. Here, translations in centimetres and rotations in degrees were recorded for each frame-pair. Different amounts of noise were added to both pairs prior to registration. \\

The noise is measured in both range (with random noise being a certain percentage of the maximum signal value) and Signal Noise Ration or SNR). The registration error in both centimetres and voxel error (the registration error in voxels). 

%translation
\begin{table}[ht]
\centering
\scalebox{0.75}{
\begin{tabular}{ccccc}
\hline
\textbf{translation (cm)} & \textbf{noise range (\%)} & \textbf{SNR} & \textbf{error (cm)} & \textbf{error (voxel)}\\ \hline
5cm & 0 & $\infty$ & 0 & 0\\
5cm & 10 & 20db & 0 & 0\\
5cm & 25 & 12db & 0 & 0\\
5cm & 50 & 6db & 0 & 0\\
5cm & 75 & 2.5db & 112.28 & 89.83\\
10cm & 0 & $\infty$ & 0 & 0\\
10cm & 10 & 20db & 0 & 0\\
10cm & 25 & 12db & 0 & 0\\
10cm & 50 & 6db & 156.65 & 125.32\\
15cm & 0 & $\infty$ & 2.8 & 2.24\\
15cm & 10 & 20db & 2.8 & 2.24\\
15cm & 25 & 12db & 2.8 & 2.24\\
15cm & 50 & 6db & 198.55 & 158.84\\
\\
\end{tabular}}
\\
\caption{Translation Tracking}
\label{table:trans}
\end{table}[ht]

To assess robustness to noise, the estimated camera parameters are compared to ground truth data under different noise conditions. In each experiment, varying amounts of random noise were added per voxel prior to registration. This is expressed in decibels using the Signal to Noise Ratio (SNR). Each voxel value lies in the range [0-1]. Here, a noise value of 10\% means random noise was added in the range [-0.05, 0.05]. Tracking error is measured in centimetres and voxel error (the error in the phase correlation volume). The first experiment evaluated noise robustness whilst the camera was translated by varying amounts (5cm, 10cm and 15cm). Results in Table \ref{table:trans} show that, for camera translations up to 15cm and SNR values above 6.0 our method is robust to noise. At video rates, a displacement of 10cm per frame equates to a camera velocity of 3 m/s (about twice the normal walking speed). 



\begin{table*}[ht]
\parbox{.45\linewidth}{
\centering
\scalebox{0.75}{
\begin{tabular}{ccccc}\hline
\textbf{rotation} & \textbf{noise (\%)} & \textbf{SNR} & \textbf{error ($\theta$)} & \textbf{error (voxel)}\\ \hline
$10^{\circ}$ & 0 & $\infty$ & 0.31 & 0\\
$10^{\circ}$ & 10 & 20db & 0.31 & 0\\
$10^{\circ}$ & 25 & 12db & 0.63 & 1\\
$10^{\circ}$ & 30 & 10.5db & 90.62 & 96\\
$20^{\circ}$ & 0 & $\infty$ & 0.31 & 0\\
$20^{\circ}$ & 10 & 20db & 0.63 & 1\\
$20^{\circ}$ & 15 & 16.5db & 38.13 & 40\\
$30^{\circ}$ & 0 & $\infty$ & 3.75 & 4\\
$30^{\circ}$ & 10 & 20db & 3.28 & 3\\
$30^{\circ}$ & 15 & 16.5db & 30 & 32\\
\\
\end{tabular}}
\caption{Rotation Tracking}
\label{table:rote}
}



Table \ref{table:rote} shows the results for tracking camera rotations of 10, 20 and 30 degrees per frame. At video rates, 12 degrees per frame is almost a full rotation per second. In rotations of 10 degrees, the error was less than a degree for all but a noise level of 30\% and above. This base line error is due to the sampling resolution of the volume, as voxel error was in fact zero. As with pure translation, the effect of noise increases with camera disparity. At 30 degrees, little matching information is available. However, for noise levels of 10\% or less, voxel distance error was as low as 4 with an angular error less than 3.8. Rotations of this magnitude are unlikely, moreover motion blur would occur.

\subsection{Object Motion Test}

\hfill
\parbox{.45\linewidth}{
\centering
\scalebox{0.75}{
\begin{tabular}{ccc}\hline
\textbf{Object Size} & \textbf{error (cm)} & \textbf{error (voxel)}\\ \hline
0.35 & 0 & 0\\
2.95 & 0 & 0\\
6.22 & 0 & 0\\
12.28 & 0 & 0\\
19.82 & 0 & 0\\
22.39 & 0 & 0\\
26.09 & 0 & 0\\
31.00 & 0 & 0\\
48.23 & 38.42 & 15\\
74.32 & 113.57 & 44\\
\\
\end{tabular}}
\caption{Object Motion Test}
\label{table:OBJECT_MOVE_EXP}
}
\end{table*}[ht]


To assess robustness to object motion, experiments were conducted by moving the camera backwards along the z-axis by 5cm per frame whilst moving objects in and out of the scene so that they only appear in one of the volumes being registered. Various sized objects including stacks of CDs, large boxes, people and several pieces of furniture were used and are measured by the percentage of the frame they occupy. Results from Table \ref{table:OBJECT_MOVE_EXP} show the proposed method was accurate upto an object size of 31\%, but failed for objects taking up over 48.23\%.

