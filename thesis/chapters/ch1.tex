\begin{savequote}[8cm]
  ``He who controls the past controls the future.''
  \qauthor{George Orwell, 1984}
\end{savequote}
\makeatletter
\chapter{Literature Review}

\section{Introduction}

\section{3D Reconstruction and Simultaneous Localization and Mapping}
\subsection{Introduction}
hi there
\subsection{Monocular Camera Feature Based Systems}
Monocular Feature based SLAM systems use feature matches to estimate camera pose and location changes across frames \cite{Davison02Simultaneous}. Variations of this method use different features including: corners and lines \cite{Jeong06Visual}, image patches \cite{Silveira08Efficient} and exemplar feature matching \cite{Chekhlov07Robust}. SIFT features are used most often in SLAM \cite{Jensfelt06Framework,Pollefeys08Detailed,Beall11Bundle,Eudes10Fast}, in addition FAST features have been explored \cite{Kundu10Realtime,Leelasawassuk133d,Konolige10View,Konolige08Frameslam}. Beall et al \cite{Beall11Bundle} made use of both SIFT and SURF features in their underwater SLAM system. Real-time monocular SLAM systems based on this approach have also been proposed \cite{Chekhlov07Robust,Pollefeys08Detailed}. RANSAC is often used in monocular SLAM \cite{Eudes10Fast,Kundu10Realtime,Konolige10View,Konolige08Frameslam,Pradeep13Monofusion} to remove outliers which cause incorrect camera parameter estimates. Bundle adjustment is also used as an additional step to refine camera parameter estimation \cite{Eudes10Fast}. 
\subsection{Stereo Camera Feature Based Systems}
Stereo based SLAM systems also use features to estimate camera parameters. However, stereo based systems are capable of generating dense depth data more easily using stereo algorithms. Miro et al \cite{Miro06Towards} proposed a stereo based method which uses SIFT and the extended Kalman filter. The method by Van Gool et al \cite{Pollefeys04Visual} works with un-calibrated stereo pairs. It uses Harris corner features and a multi-view stereo algorithm. Sim et al \cite{Sim05Vision} and Gil et al \cite{Gil06Improving} both presented stereo based SLAM systems which use SIFT.

\subsection{Kinect Fusion}

accurate, realtime, complex indoor environments

in variable lighting conditions

fuses depth data into a global surface model

uses a coarse to fine ICP method

uses all of the depth data available

global rather than frame-to-frame based

commodity sensor and GPU hardware

area promises new augmented reality and mixed reality applications

SFM and MVS (multi view stereo) has given good results : camera tracking  and sparse reconstructions [10], [24]

SLAM has focused on real time markerless tracking and live scene reconstruction based on a single sensor RGB eg. MonoSLAM[8] which is less accurate than PTAM [17] but these only produce sparse reconstructions

some methods have grown to combine PTAM's camera tracking ability with MVS-style reconstructions [19/26]

recently: iterative image alignment has been used to replace features in tracking [20]

this scene is promising: but monocular based dense 3D reconstruction is difficult and requires suitable camera motion and scene illumination

camera technologies have also been evolving: enter the world of new depth cameras or rgb-d cameras,
these have become availbable to consumers and may soon be found in mobile technologies [cite goog]

first real-time rgb-d depth sensor : with this system users can wave around the device to generate smooth, continuously updating, fully reconstructed data: using only depth, 6dof are tracked

works in full darkness : mitigating issues for passive cameras: [17/19/26] and other rgb systems [14]

kw : dense volumetric reconstruction

GPGPU kw: is one of their design keys, their system can perform in real time at above these frame rates

they perform qualitative analysis

Kinect: incorporates a structured light based depth sensor, uses an on board ASIC generating an 11-bit 640x480 depth map at 30hz

depth images often contain holes : this is an issue (caused by no structured light could be read on the surface, certain materials, which do not reflect infra-red light (very thin structures or surfaces at incendence angles

when moving fast, the device can also experience motion blur, this leads to missing data















\ref{Newcombe11Kinectfusion}

\subsection{RGB-D Sensor Feature Based Systems}
RGB-D SLAM systems use both depth and image data and are capable of generating dense 3D reconstructions. Many of these methods rely on feature matching techniques \cite{Engelhard11Real,Henry10Rgb,Endres12Evaluation}. RANSAC is often used to filter outliers for the estimation of camera parameters\cite{Engelhard11Real,Henry10Rgb,Endres12Evaluation}. Another method which has also been used extensively in the area is Iterative Closest Point (ICP) \cite{Engelhard11Real,Henry10Rgb,Bylow13Real,Newcombe11Kinectfusion,Stuckler12Robust,Izadi11Kinectfusion}. ICP iteratively registers point cloud data, and is used to refine camera parameter estimates. A method named KinectFusion was proposed by Newcombe et al \cite{Newcombe11Kinectfusion} which uses RANSAC and a GPU implementation of IPC. Whelan et al \cite{Whelan12Kintinuous} extended this method allowing it to map larger areas using Fast Odometry From Vision (FOVIS) over ICP. Bylow et al \cite{Bylow13Real} improved the ICP approach by registering data using a signed distance function.
\subsection{Non-Feature Based Methods}
Several RGB-D SLAM systems are also non-feature based \cite{Weikersdorfer14Event,Izadi11Kinectfusion,Kerl13Dense}. Weikersdorfer et al \cite{Weikersdorfer14Event} presented a novel sensor system named D-eDVS along with an event based SLAM algorithm. The D-eDVS sensor combines depth and event driven contrast detection. Rather than using features, it uses all detected data for registration. Kerl et al \cite{Kerl13Dense} proposed a dense RGB-D SLAM system which uses a probabilistic camera parameter estimation procedure. It uses the entire image rather than features to perform SLAM.
\subsection{Summary}
As is evident from the current literature, SLAM typically relies on feature matching and RANSAC. However, these approaches fail when there are too few features, when feature confusion occurs or, when features are non-stationary due to object motion. As the extent of random feature displacement becomes more global the effectiveness of these approaches diminishes. Feature matching also dominates in image registration. However, Fourier based methods have been shown to work well under larger rotations and scales \cite{Gonzalez11Improving} whilst being closed form, insensitive to object motion and scaling naturally to GPU implementations. Accordingly, we propose a novel, closed form Fourier based SLAM method.

Simultaneous localization and mapping (SLAM) has applications in many fields including: robotics, business, architecture and engineering, and science. Its goal is to generate a map (2D birds-eye view, or 3D) of an environment captured by camera and/or other means. In this work we focus on monocular systems, or systems which generate location and mapping data using information generated by a single basic video camera. To this end, current methods rely on the computation of the fundamental and essential matrices. These feature matching techniques fail in cases where features are not stable or where feature confusion occurs. 

It has been shown [1] that using volume registration to compute dense 3D maps is not only independent of feature matching, but it is a closed form solution and is robust to noise and object motion. However, this method requires RGB-D video input provided by special hardware. In this paper we present preliminary results in applying volume registration to generate dense 3D maps from monocular video data. To achieve this, disparity maps are generated between video frames. This data is then used as input for the RGB-D volume registration method.

\section{Feature Matching and RANSAC}

\section{Iterative Closest Point}

\section{Fourier Based Registration}

\section{3D Data Compression Schemes}

\subsection{Model Compression}

The Feature-Oriented Geometric Progressive Lossless Mesh coder (FOLProM) \cite{Peng10Feature} is a state of the art codec which is progressive. It also aims to be an effective low-bitrate codec. It classifies segments of the mesh as being visually salient or not. Salient segments are preserved more during compression compared to non-salient ones. \\

\subsection{Spectral Compression}

Karni and Gotsman \cite{Karni00Spectral} proposed a lossy method which compresses a spectral representation of a mesh. This algorithm generally partitions the mesh and compresses each partition separately since it does not work on large meshes. Encoding a basis function for each partition, coefficients are quantized, truncated and entropy coded. Results show this method outperforms the valence method \cite{touma98triangle} at coarse quantization levels. Bayazit et al. \cite{Bayazit103DMesh} also developed a progressive method based on spectral compression. This method is based on the region adaptive transform in the spectral domain and is advertised as a current state of the art lossy 3D data compression method. \\

\subsection{Wavelet Methods}

A lossy wavelet based compression system was proposed by Khodakovsky et al \cite{Khodakovsky00Progressive}. This technique samples the mesh, and uses the wavelet transform to decorrelate the data. Coefficients are quantized and stored in a structure called a zero tree which increases compression performance. This method is shown to outperform the valence method. Other wavelet approaches \cite{Guskov00Normal,Khodakovsky04Normalmesh} also sample the mesh and use a multi-resolution representation in which the data is described using local normal directions on the mesh surface.

Gu et al \cite{Gu02Geometry} devised a solution for representing 3D models as 2D images which are then compressed using state of the art image compression methods (based on wavelets). To form this representation, the mesh is cut along a network of edge paths, opening the mesh into a topological disk, which is then sampled onto a 2D grid. Each pixel in the image has a corresponding coordinate in the model, with pixel neighbourhoods describing connectivity. Comparisons with the method by Khodakovsky et al reveal the geometry image codec does not have as high compression performance.



\section{Conclusion}

