
To answer the research questions about the impact of Fourier based 3D reconstruction techniques and hierarchical 3D data compression, the context for the area of 3D data processing, 3D reconstruction and registration techniques must be explored by the reader. By providing context for these areas, the algorithms proposed in Chapter \ref{ch:Metho}: Methodology can be evaluated (Chapter \ref{ch:Experiments}: Experiments). \\

At the beginning of this chapter, a background into registration, depth data generation and 3D data representation is provided followed by an in-depth survey of 3D reconstruction procedures. The initial background in section \ref{Sec:LitRevBackgroundX} details several important areas in 3D reconstruction, 3D registration and 3D data generation and representation. Section \ref{Sec:LitRevBackgroundX} starts with subsection \ref{sec:LitRevFeatureMatching11} which contains a review of feature matching methods. Feature matching methods are the first-step for several registration techniques which make up 3D reconstruction algorithms. Subsection \ref{Sec:SuperficialPCSection} describes the phase correlation method, a Fourier based technique for data registration. Subsection \ref{Sec:SuperficialPCSection} provides context for several of the algorithms proposed in chapter \ref{ch:Metho}. Subsection \ref{DepthDataGenSection} describes depth data generation using various sensors and software techniques. Such depth data is the input to several 3D reconstruction techniques, therefore a solid understanding of depth data generation is important for understanding the algorithms described in section \ref{sec:LitRevReconstructionTechniques} and the novel methods proposed in Chapter \ref{ch:Metho}. Subsection \ref{sec:LitRevDataRepresentation} reviews 3D data representations. These are compared to the proposed Plane-Tree representation technique in section \ref{sec:PlaneTreeMetho18}. \\

Later in this chapter, section \ref{sec:LitRevReconstructionTechniques} surveys several current techniques for 3D reconstruction and registration. These techniques are compared to the proposed Fourier based techniques in Chapter \ref{ch:Experiments}. The techniques covered include: the Fundamental matrix method, Feature matching with RANSAC, ICP, Direct optimization, 3D feature matching and PCA. \\

There are various terms used in the literature survey and methodology which have unique meanings within the 3D reconstruction context. When an algorithm or technique is said to be fast, it refers to the practical time it takes to run through the algorithm as compared with others. If one algorithm is faster than another it means it completed processing the same data in a shorter time. Additionally, if an algorithm is said to be more robust than another, it means it produces better 3D reconstructions given noisy data. Several algorithms are also referred to as popular. In this context, this means the algorithm is often used in research as well as in applications. \\


