Feature Matching is of major importance not just in SLAM and 3D Reconstruction but in much of the research and applied applications within Computer Vision. Therefore in this section we provide a brief survey of popular feature detection, description and matching algorithms. \\

Harris and Stephens \cite{Harris88Combined} invented the Harris corner detector. This detector uses a variable sized window around each pixel, in which a Harris matrix is formed from the x, y and xy gradients. The Harris response is calculated using the determinant and trace of this matrix. Several years later, Smith and Brady \cite{Smith97Susan, Smith92New} presented a feature detector called SUSAN. SUSAN is an alternative to second order methods for corner detection and uses a non-linear filter to find corners and edges. SUSAN also naturally provides feature vectors. It works by surrounding each pixel with a circular non-linear kernel for filtering. The kernel's response is defined as the area within the kernel having the same or similar value to the nucleus (centre of the kernel). Walker et al. \cite{Walker98Locating} used a classification (machine learning) method to find salient places in images.\\


Trajkovic \& Hedley \cite{Trajkovic98Fast} presented a fast yet simple corner detection algorithm. This method computes the minimum intensity changes in all directions. It is fast as it only uses a 3x3 window for the corner response function. This method is compared to the Harris and Susan corner detectors. It is faster than both these methods whilst also being more robust than the Susan corner detector. Harris is the more accurate corner detector in terms of repeatability and detection.\\


Lowe's \cite{Lowe04Distinctive,Lowe99Object} method, SIFT (Scale Invariant Feature Transform) is a popular method for feature extraction and description. The method computes features at different scales using an image pyramid and difference of Gaussian to approximate the Laplacian of Gaussian. Features are represented using a vector of weighted gradients surrounding the feature. The descriptor is invariant to scale because each descriptor is found at some scale within the multi-resolution pyramid. SIFT is rotationally invariant if the window is chosen so that its angle of origin is based on the angles surrounding the feature. It is invariant to luminance because of the use of gradients, and since features are described by the surrounding window of the feature, they are invariant to translation. This method is said to be robust to 3D viewing transforms and affine transforms. SIFT was also developed to run faster on GPUs \cite{Wu07Siftgpu}. \\


Tuytelaars \& Van Gool \cite{Tuytelaars00Wide} developed a method for detecting and describing affine invariant regions. These regions are computed directly from intensity values in the image using rays extending from the centre of regions. Feature vectors made up from statistical moments within the regions, then nearest neighbour matching is used to match features. Boykov \& Jolly \cite{Boykov01Interactive} presented a method for region based feature extraction. This method uses graph cuts to find which regions are adequate features. This method requires some soft constraints performed by humans so is not good for automatic detection of features, also because features are not very localized their size and shape is not viewpoint invariant.\\


Itti \& Koch \cite{Itti01Computational} presented a biologically inspired bottom up image saliency detector. Schaffalizky and Zisserman \cite{Schaffalitzky01Viewpoint} developed a texture based region descriptor. It is invariant to photometric and affine transformations. It is also insensitive to the shape of the region and can be used to compute epipolar geometry. This method makes use of the second order matrix. Mikolajczyk and Schmid \cite{Mikolajczyk01Indexing} presented a new feature point detector and descriptor. Their detector is based on multi-scale Harris corner detection which is used to filter the points by the value of the surrounding Laplace. For feature description they use Gaussian derivatives. \\


Carson et al. \cite{Carson02Blobworld} developed a method for image feature classification and matching called Blobworld. This method segments an image before using region vectors for image querying and feature matching. This technique begins by defining a pixel neighbourhood size, it then groups pixels together based on the texture and colour data within one of these neighbourhoods. Finally, vectors describing colour and texture are formed for each region and these are used in image queries. Sebe et al. \cite{Sebe03Evaluation} compared local based feature detectors, their proposed method uses a wavelet saliency extractor, making use of textures and colour in order to obtain invariance in its descriptor. \\


Kadir et al. \cite{Kadir04Affine} developed a saliency based method for feature detection. This method is scale, viewpoint and perturbation invariant. Carbonetto et al. \cite{Carbonetto04Statistical} presented a method which segments images, labelling them with feature vectors made up of descriptive words. Image region mapping can be performed by statistically comparing feature description vectors. Matas et al. \cite{Matas04Robust} developed a new feature detection and description method called MSER. This algorithm uses small regions as features instead of a single point and surrounding window. These regions are calculated by taking the foreground blobs of an image at every possible binary threshold. MSER detection and representation is invariant to scale (3.5 x), illumination, out-of-plane rotation, occlusion, locally anisotropic scale change and 3D translation of viewpoint. \\

Mikolajczyk and Schmid \cite{Mikolajczyk05Performance} performed an evaluation of local feature descriptors. They performed a comparison between shape-context, PCA-SIFT, differential invariants, spin images, SIFT, complex filters, moment invariants and cross-correlation. They also present their method called GLOH, which is an extension of the SIFT descriptor. Results indicate that GLOH and SIFT perform the best. Rosten \& Drummond \cite{Rosten06Machine,Rosten05Fusing} aimed to improve the speed of feature detection over SIFT and SUSAN. They developed a technique called FAST (Features from Accelerated Segment Test), this method tests the difference between the centre pixel and its surrounding pixels within the surrounding circle. They also improved this approach by first extracting FAST corners, then classifying these corners using a decision tree to extract better feature points, whilst retaining speed. Both FAST and FAST-ML (FAST with Machine Learning) methods are shown to be faster than SIFT and SUSAN whilst FAST-ML is also shown to be more reliable at classifying the same features from different viewpoints. Another method, ORB \cite{Rublee11Orb} is based on FAST and BRIEF \cite{Calonder10Brief}. ORB detects features similarly to FAST and computes rotation and feature descriptors similarly to BRIEF. It is shown to be robust to viewpoint changes and is faster than SIFT and SURF.  \\ 


Bay et al. \cite{Bay06Surf,Bay08Speeded} improved the speed and accuracy of feature matching compared with SIFT using their method named SURF (Speeded Up Robust Features). In SURF, a Hessian matrix is used for the detection of features whilst Haar wavelet components are used as descriptors. A non rotation invariant version was also analysed and proved to be faster. This version is suitable if rotation invariance is not required for a particular application. Lepetit and Fua \cite{Lepetit06Keypoint} turn the wide stereo baseline matching problem into a classification problem. This method is stated to be robust, accurate and real-time. The training phase attempts to classify repeatable corners. The training set is built from many different rasterizations (which affect illumination) of only a few images. Key-points are extracted at multiple octaves and scales at the surrounding image patch. The authors report randomized trees as being the optimal machine learning method for their technique. \\


Cabani \& MacLean \cite{Cabani07Implementation} presented a feature detector based on the Harris point operator. This method detects affine invariant features using the speed of an FPGA. It can process images of 640 x 480 at up to 30 frames per second. This technique is compared to C and Matlab versions of the same process. Finally, Tuytelaars and Mikolajczyk \cite{Tuytelaars08Local} presented a survey on feature detectors. They provided a detailed introduction to the subject and categorize different feature types and techniques. \\
