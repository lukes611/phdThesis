\subsection{Summary}

As is evident from the current literature, SLAM and 3D reconstruction typically rely on feature matching and RANSAC or ICP. However, these approaches fail when there are too few features, when feature confusion occurs or, when features are non-stationary due to object motion. As the extent of random feature displacement becomes more global the effectiveness of these approaches diminishes. Feature matching also dominates in image registration. However, Fourier based methods have been shown to work well under larger rotations and scales \cite{Gonzalez11Improving} whilst being closed form, insensitive to object motion and scaling naturally to GPU implementations. Accordingly, we propose a novel, closed form Fourier based SLAM method. \\

Simultaneous localization and mapping (SLAM) has applications in many fields including: robotics, business, architecture and engineering, and science. Its goal is to generate a map (2D birds-eye view, or 3D) of an environment captured by camera and/or other means. In this work we focus on monocular systems, or systems which generate location and mapping data using information generated by a single basic video camera. To this end, current methods rely on the computation of the fundamental and essential matrices. These feature matching techniques fail in cases where features are not stable or where feature confusion occurs. 

%It has been shown \cite{Lincoln16Dense} that using volume registration to compute dense 3D maps is not only independent of feature matching, but it is a closed form solution and is robust to noise and object motion. However, this method requires RGB-D video input provided by special hardware. In this paper we present preliminary results in applying volume registration to generate dense 3D maps from monocular video data. To achieve this, disparity maps are generated between video frames. This data is then used as input for the RGB-D volume registration method.