
This chapter provided an important background for 3D reconstruction. The background at the start of the chapter (section \ref{Sec:LitRevBackgroundX}) provided insight into feature matching (which is an important first step for several 3D reconstruction techniques), phase correlation (a technique related to several proposed algorithms in the next chapter), depth data generation (techniques to generate the input data for 3D reconstruction algorithms), and data representations (for storing and processing the 3D data which make up both the input and output to 3D reconstruction algorithms). \\


\begin{table}[!t]
\centering
\caption{Literature Algorithm Comparison}
\resizebox{\textwidth}{!}{%
\begin{tabular}{ccccc}\hline
\textbf{Algorithm} & \textbf{Robust to Noise} & \textbf{Iterative} & \textbf{Closed-Form} & \textbf{real-time}\\ \hline
Fundamental Matrix Methods & no & yes & no & yes\\
2D Feature Matching + RANSAC & no & yes & no & yes\\
RGB-D SLAM & no & yes & no & yes\\
ICP & no & yes & no & no\\
Direct Optimisation & yes & yes & no & no\\
3D Feature Matching + RANSAC & no & yes & no & no\\
PCA & no & yes & no & yes\\
\\
\end{tabular}}
\label{tab:litRev_Alg_Comp}
\end{table}


Additionally several 3D reconstruction algorithms were surveyed in section \ref{sec:LitRevReconstructionTechniques}. Techniques including: the Fundamental matrix method, Feature matching with RANSAC, ICP, direct optimization techniques, 3D feature matching and PCA were discussed. It is evident from the current literature that SLAM and 3D reconstruction have a focus on feature matching and RANSAC or ICP and most methods are iterative (see Table \ref{tab:litRev_Alg_Comp}). However, these approaches fail when there are too few features, when feature confusion occurs or, when features are non-stationary due to object motion. As the extent of random feature displacement becomes more global the effectiveness of these approaches diminishes. Table \ref{tab:litRev_Alg_Comp} summarises findings. Most algorithms are not cited as being robust to noise and object motion. Additionally, since most methods are iterative, they are not closed form solutions, their effciency degrades with feature size. Feature matching also dominates in image registration. Some more current methods are based on optimization over an SDF \cite{Bylow13Real,Rusinkiewicz02Real}. These methods promise more accurate results since they find accurate poses using the entire data. In 2D image processing, Fourier based methods have also been shown to register large rotations and scales using all of the input data \cite{Gonzalez11Improving}. Fourier approaches are also closed form solutions, insensitive to noise (object motion) and scale easily to GPU based algorithms. Accordingly, in this thesis, we propose a novel, closed form Fourier based 3D reconstruction method. Such a method would also answer the first research question ``Can Fourier based registration techniques improve accuracy and noise robustness in 3D reconstruction applications?''. The next chapter presents several techniques aimed at answering this research question. Also presented is a novel data structure aimed at answering the second research question, ``Can hierarchical techniques improve compression, storage and processing of 3D reconstruction data?'' \\
