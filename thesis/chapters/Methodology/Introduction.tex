
The previous chapter gave important context to 3D reconstruction, phase correlation, depth data generation and data representation techniques. In this chapter, several novel algorithms and data structures are used to answer the two research questions presented in the thesis introduction (chapter \ref{ch:IntroIntroduction}). The first part of this chapter (section \ref{FVRSectionA}) focuses on the first research question. The second part of this chapter describes a novel data structure named the Plane-Tree aimed at answering the second research question. \\

Section \ref{Sec:VolumeRegistrationSection} describes the basic method of 3D Fourier based registration. This technique is not novel on its own but the FVR (Fourier Volume Registration) technique proposed in section \ref{Sec:AFVRApproach} is novel because it applies this technique to 3D reconstruction. The FVR algorithm is designed to register 3D data acquired via stereo or active camera sensors. By registering the data, 3D reconstructions may be built. Section \ref{Sec:AMonoFVRApproach} presents a novel method named MVVR (Monocular View Volume Registration) which is an  extension to the FVR method enabling 2D monocular frames as input. This is useful for datasets in which no 3D depth data was captured, in which the video frames alone may be used to build the 3D reconstructions. Section \ref{Sec:Efficiency} proposes yet another novel method named Fast Fourier Volume Registration (FFVR). This algorithm improves upon the efficiency of the FVR algorithm. It reduces the amount of data by an entire dimension to achieve this performance improvement. Section \ref{FullRecovery3DSection} presents the FVR-3D algorithm which solves another shortcoming of FVR, that is it cannot solve for all three axes of rotation. These methods can be used to answer the first research question because they are novel Fourier based techniques which can be evaluated in comparison to existing 3D reconstruction methods. \\

Section \ref{sec:PlaneTreeMetho18} provides an introduction to the Plane-Tree data structure. This is a novel hierarchical representation method which can be used to answer the second research question. Section \ref{sec:method_pt_overview} gives an overview of the Octree data structure of which the inspiration for the Plane-Tree. Section \ref{sec:dr:coding} describes an intermediary design names Shade-Octree which incorporates a hierarchical structure to 3D data compression. Sections \ref{sec:pt_ptcoding}, \ref{sec:pt_ptsubdiv} and \ref{NRep} describe the process, subdivision, and leaf node representation of the Plane-Tree. Section \ref{sec:pt_ptcd} describes the algorithm for compressing and decompressing 3D data to and from the Plane-Tree representation. Finally, section \ref{sec:pt_ptvssot} describes the difference between the Plane-Tree and the Shade-Octree. \\
