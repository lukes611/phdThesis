
It is evident from the literature that many algorithms compute satisfactory 3D reconstructions in simple small scale man-made environments. These environments are ideal because there are often suitable levels of texture and features. This works well for both feature matching approaches and ICP. In some environments however, textures are few and far between. These environments often do not produce enough accurate textures for 3D reconstruction or SLAM. \\

Other environments may have an abundance of similar textures (grassy meadows, out-door environments which include trees, gardens and plants). These environments contain plenty of texture and thus generate many features. Feature matching is difficult though, as features tend to look identical within a local frame. \\

Some algorithms, such as the method by Bylow et al. (see section \ref{BylowsSection}), work on finding the optimal solution using all the data. Finding the best solution using all the data mitigates issues with feature confusion and incomplete feature sets. However, this algorithm, as well as other iterative methods for SLAM and 3D reconstruction (Feature Matching + RANSAC, ICP, using the Fundamental Matrix), may still get stuck in local extrema, especially when registering against wide baselines. This is the nature of iterative methods. \\

In this section I propose using Fourier and phase correlation approaches to compute camera pose as an alternative to iterative and feature based methods. Phase correlation is not novel on its own, but using it for 3D reconstruction is an unexplored and exciting topic. Empirical results presented in Chapter \ref{ch:Experiments} suggest this method is highly competitive, outperforming the previously proposed iterative solutions to the camera pose alignment problem given input from various sensor types (stereo and active RGB-D). \\

In the first section (\ref{Sec:VolumeRegistrationSection}), a description of a basic variation of Fourier-based 3D registration is discussed. This method is capable of registering 3D data with respect to x/y/z axis translation as well as y-axis rotation. In cases where data is scaled to fit volumes or in cases where scale is not known, such as when depth data is estimated from monocular data, scale may also be recovered for registration. \\

Section \ref{Sec:AFVRApproach} describes the implementation of a SLAM/3D reconstruction method for computing pose named the Fourier Volume Registration (FVR) approach. The FVR is made up of the techniques described in section \ref{Sec:VolumeRegistrationSection}. This approach makes use of any depth data produced by stereo camera set-ups or active RGB-D sensors such as the Microsoft Kinect or Asus Xtion Pro Live sensor. A pipeline for registration, camera pose tracking and 3D reconstruction is proposed. In Chapter \ref{ch:Experiments}, this approach is quantitatively and qualitatively evaluated against other algorithms using stereo and active camera sensors (sections \ref{StereoSOTA} and \ref{ActiveSOTA}), as well as robustness to noise and input captured at wider baselines (section \ref{Sec:CamTransTrackExp}). Results show this method is capable of robust and accurate registration in the face of noise and that it is highly competitive compared to current algorithms presented in the literature. They also show it produces high quality models fit for visualization and analytical purposes and can be used with different sensor types.  \\

Section \ref{Sec:AMonoFVRApproach} presents an extension of the FVR SLAM/3D reconstruction method to monocular sensors and data. Such a technique would be useful in situations where depth information is not directly available. Since most portable devices (watches, phones and tablets) contain passive monocular RGB camera sensors, extending FVR to monocular data makes the technique much more accessible. This method's usability is highly dependent on the accuracy of the depth data computed via monocular depth estimation procedures. Results in section \ref{Sec:MonocularSOTA} show that this method is capable of registering camera translations near the accuracy of the FVR method given dense depth data input approximating quality associated with either active or stereo camera set-up.  \\

A novel phase correlation technique is proposed in section \ref{Sec:Efficiency} as a trade-off between efficiency, accuracy and noise robustness. This technique, named FFVR (Fast FVR), exploits data projection (reduction) in a similar yet simpler way than other techniques which reduce the amount of data to process (such as Principal Components Analysis). \\

Finally, a novel technique to extend the FVR approach to register full 3D rotation is presented. This method uses the PCA algorithm and is capable of registering 3D data against seven types of transforms including: x-axis rotation, y-axis rotation, z-axis rotation, x-axis translation, y-axis translation, z-axis translation as well as scale. This technique may also be applied to 3D reconstruction and SLAM as well as monocular sensor FVR. Results in sections \ref{Sec:FVRSOTA} show that the FVR/FVR-3D method outperforms several current iterative techniques in terms of registration error using both stereo and active camera set-ups.  \\