
The proposed FVR method is interesting in comparison with other major techniques such as Feature Matching + RANSAC, ICP and other optimization methods in that it is not an iterative method but is a closed form solution. That is, its computational complexity is fixed and does not depend on the input data. Despite this, some parts of the pipeline (see figure \ref{fig:PIPELINENo1}) remain intensive, even for GPGPU and other parallel processing devices. \\

In order to reduce complexity, the different parts of the pipeline were examined in order to reveal any possible improvements. It is understood that each Hanning window function in \ref{fig:PIPELINENo1} is required to reduce noise on the phase correlation surface. The technique is already highly parallelized    and is not very computationally intensive. Much effort has already been given to improving efficiency in computing the Fourier transform. \\

The element-wise log function is similar to the Hanning window function, it is required to correlate the data to find the rotation and scale factors separating both volumes. Furthermore it is already highly parallelized and cannot be simplified. The 3D phase correlation technique is by far the most computationally intensive operation within the pipeline. It requires 2 $\times$ FFTs, 1 $\times$ element-wise operation, 1 $\times$ inverse FFT and 1 $\times$ peak search operation. Moreover there are two required 3D Phase Correlation operations during the pipeline. The other transform operations are also element-wise operations introducing minimal computation expense into the pipeline. \\

In order to reduce the computational complexity of the 3D phase correlation, several projection operations are used to retain as much information as possible whilst reducing the data to 2 dimensions in such a way that 2D phase correlation (a much faster operation) may be used in place of 3D phase correlation to retrieve transformation factors. Two transforms are proposed to achieve this. The Spherical-Map Transform (section \ref{SMTransform} reduces the original 3D frames to 2D. One useful property of this transformation is that correlation between two Spherical-Map domains retrieved both two 3D frames yields the y-axis rotation and scale factor parameters between the original 3D frames. Moreover, because the spherical-map space is a 2D space, phase correlation may be used in place of manual correlation.  \\

The other transformation is proposed to efficiently compute translation factors separating two 3D volumes. This transform is simply named a projection transform (see section \ref{sec:PMTramsform}). It reduces the 3D input frames to 2D images whilst retaining the translational information along two remaining axes. Correlating two projection transform domain images yields two translation parameters (depending on the type of projection transform) which separate the two original 3D frames. \\

A block diagram integrating these speed improvements into the FVR method introduced in section \ref{FVRSectionA} is shown in figure \ref{fig:PIPELINE3}. This procedure is referred to as the Fast Fourier Volume Reconstruction method (FFVR). As shown in the block diagram, this procedure takes two 3D volume frames as input, $Volume_1$ and $Volume_2$. The second frame may be taken after the camera has changed pose about the y-axis and/or has moved locations. Both inputs are then put through a 3D FFT function to produce the magnitude values of the 3D frequency domain of both volumes. These operations may be performed on a GPGPU and may both be performed in parallel with each other. \\

The magnitude of the frequency domain is independent to translation and any rotation and scale occurs about the center of both volumes. Both frequency domain volumes are then transformed into 2D spherical space using the $Spherical2DMap$ function. This function produces an image in which 3D y-axis rotation from the original volume is interpreted as 2D translation. To recover the rotation parameter, phase correlation is used to measure the translational component separating the spherical-map domain images. This translational component is then processed to compute the y-axis rotational factor separating the original input volumes. The rotation factor be be directly output as a parameter if required. \\


\begin{figure}[!htb]
\centering
\includegraphics[width=5.0in]{images/ch2/pipeline3}
\caption{System Diagram for Fast Volume Registration}
\label{fig:PIPELINE3}
\end{figure}

Next, the first 3D frame, $Volume_1$ is transformed by the computed rotation factor. This leaves only a 3D translation transform separating both inputs $Volume_1$ and $Volume_2$. Two projection map transforms of both $Volume_1$ and $Volume_2$ (equivalent to 4 transforms in total) are then used to efficiently find this translation factor. The first projection map transform is along the z-axis. The z-axis projection map transform of $Volume_1$ produces 2D image, $Image_{za}$ the z-axis projection map transform of $Volume_2$ produces 2D image, $Image_{zb}$. Both $Image_{za}$ and $Image_{zb}$ may be phase correlated producing the x and y axis components of the translation separating $Volume_1$ and $Volume_2$. \\

The other two projection map transforms are along the x-axis. The projection map transform of $Volume_1$ produces $Image_{wa}$, whilst the projection map transform of $Volume_2$ produces $Image_{wb}$. These two images ($Image_{wa}$ and $Image_{wb}$) may be phase correlated producing the z-axis component of the translation. A composite registration matrix aligning $Volume_1$ to $Volume_2$ may be formed by translating each volume's center to the origin, rotating by the computed rotation factor, translating the origin to the volume's center and finally translating by the $[X,Y,Z]^T$ translation vector computed. \\

Noticeably, the Hanning window function and post $|DFT|(x)$ $Log(x)$ function from the original pipeline (figure \ref{fig:PIPELINENo1}) are missing. The Hanning window function may be incorporated in the first phase correlation procedure, which processes 2D images, therefore the Hanning window function would be in 2D rather than 3D which also improves efficiency as an entire dimension is removed from the process. The log function may also be performed as part of the phase correlation procedure, however it was found that the FFVR procedure estimates rotation more reliably using the spherical-transform if the log function is not performed. This saves additional computational power. \\

It is advantageous to use 2D phase correlation over 3D phase correlation from a computational complexity perspective. The 3D Fourier transform has complexity of $N^3 \times Log(N^3)$ whilst the 2D has complexity $N^2 \times Log(N^2)$. Essentially the amount of data to process has been reduced by an entire dimension. The Phase Correlation method requires 2 $\times$ FFTs, 2 $\times$ element-wise computations and 1 $\times$ inverse FFT. The corresponding 3D phase correlation complexity equates to $3N^3Log(N^3) + 2N^3$ whilst the 2D equivalent is only $3N^2Log(N^2) + 2N^2$. \\ 

\subsubsection{Spherical-map transform}
\label{SMTransform}

As noted, the Spherical-Map transform both reduces the 3D volume to a 2D image whilst retaining information about y-axis rotation. In the new domain, and rotation about the y-axis becomes x-axis translation within the output image. The transform requires a single iteration over the input volume, so it has identical complexity to the 3D Log-Polar transform whilst additionally reducing computational complexity further down the pipeline by compacting the data to process from three dimensions to two dimensions.  \\

An example of an input model (figure \ref{fig:bunnyOrigAA}) and the Spherical-Map domain of the model (figure \ref{fig:bunnySPTed}) is shown in figure \ref{fig:smtExample}. The relationship between the input 3D volume $Vol$ and the output 2D image $Im$ is defined using equations \ref{eqn:invLPFuncs} and \ref{eqn:smtUpdate}. The value of pixel $Im_{x,y}$ located at coordinate $x,y$ is computed by summing the values along a given ray within the volume. The ray is defined by a  


Given a coordinate in 2D Cartesian space x,y, we compute the ray $[Ray_x Ray_y Ray_z]^T$ from the volume center and sum up the voxel values along the ray (equation \ref{eqn:smtUpdate}). \\


\begin{equation} \label{eqn:invLPFuncs}
\begin{split}
Ray(x,y,r) & = f(r)i + f(r)j + f(r)k
Ray_x(x,y) & = cos\left(\frac{360x}{N}\right)sin\left(\frac{180y}{N}\right)  + \frac{N}{2} \\
Ray_y(y) & = cos\left(\frac{180y}{N}\right) + \frac{N}{2} \\
Ray_z(x,y) & = 	sin\left(\frac{360x}{N}\right)sin\left(\frac{180y}{N}\right) + \frac{N}{2}
\end{split}
\end{equation}

\begin{equation} \label{eqn:smtUpdate}
Im_{x,y} = \sum_{r=1}^{(2^{-1}N)^{1.5}}{Vol(Ray_x(x,y)r, Ray_y(y)r, Ray_z(x,y)r)} 
\end{equation}

This process essentially sums up the values along a given ray defined by scaling spherical coordinates and adding up the values intersecting the ray. The resulting image, maps 3D y-axis rotation to 2D x-axis translation.  \\

\begin{figure}[!htb] 
        \centering
        \begin{subfigure}[b]{2.5in}
                \includegraphics[width=2.5in]{images/ch2/bunny}
                \caption{original}
                \label{fig:bunnyOrigAA}
        \end{subfigure}
        \begin{subfigure}[b]{2.5in}
                \includegraphics[width=2.5in]{images/ch2/spherical2DMap}
                \caption{transform}
                \label{fig:bunnySPTed}
        \end{subfigure}%
        \caption{The Spherical Map Transform.}
       \label{fig:smtExample}
\end{figure}


\subsubsection{Projection-map transform}
\label{sec:PMTramsform}
The projection map transform is similar to an orthogonal projection of the volume along some given axis. For the projection map transform, given an output image $Im_a$ and an input volume $Vol_a$, each pixel in $Im_a$ is defined mathematically as the summation of values along a particular axis given the image coordinates. The x-axis transform and the z-axis transform are defined in equations \ref{eqn:xPMT} and \ref{eqn:zPMT} respectively. \\

WARNING: add in a picture here

\begin{equation} \label{eqn:xPMT}
Im(z,y) = \sum_{x=0}^{N}{Vol_a(x,y,z)}
\end{equation}

\begin{equation} \label{eqn:zPMT}
Im(x,y) = \sum_{z=0}^{N}{Vol_a(x,y,z)}
\end{equation}

The process defined by equation \ref{eqn:xPMT} maps 3D z-axis translation to 2D x-axis translation, whilst equation \ref{eqn:zPMT} maps 3D x-axis and y-axis translation into 2D x-axis and y-axis translation.


To assess the performance of our method, the size of the volumes being registered is defined as $N^3$ whilst each frame is sampled at a resolution of $W$ $\times$ $H$. The projection process requires $12WH$ operations whilst re-sampling the point cloud requires $2WH$ operations. The Volume Registration process, $VolumeRegister{\theta \varphi t_x t_y t_z}(V_1, V_2)$ consists of 2 $\times$ Hanning windowing processes, 2 $\times$ 3D FFTs, 2 $\times$ volume-logs, 2 $\times$ log-spherical transforms, 2 $\times$ phase correlation processes and 1 $\times$ linear transformation and peak finding. 

The Hanning windowing function requires 26 operations. The 3D FFT has complexity of $3N^3\log{N}$, the log and log-spherical transform functions require 3 and 58 operations per voxel respectively. Multiplying two frequency spectra together and transforming a volume requires 15 and 30 operations per voxel respectively. Finding the peak value requires $2N^3$ operations. The complexity in terms of number of operations for the phase correlation process is given in Eq. \ref{eqn:PCFULLPERFORMANCE} This process requires 2 $\times$ 3D FFTs, 1 $\times$ frequency spectra multiplication, and 1 $\times$ peak finding operation. 
\begin{equation} \label{eqn:PCFULLPERFORMANCE}
6N^3\log{N} + 2N^3 + 15
\end{equation}
The total complexity can then be found by taking into account the projection and re-sampling totals as well as the total for $VolumeRegister{\theta \varphi t_x t_y t_z}(V_1, V_2)$. Tallying the number of operations for each process and multiplying them by number of times the process is performed gives us the number of operations as a function of $W$, $H$ and $N$ in Eq. \ref{eqn:FULLPERFORMANCE}.
\begin{equation} \label{eqn:FULLPERFORMANCE}
6N^3 + 28WH + 18(N^3\log{N}) + 230
\end{equation}

To compare performance of the generic volume registration method with the speed up, we use the complexity defined in equation \ref{eqn:FULLPERFORMANCE}. Here, we ignore the cost of projecting the depth map. The 3D DFT has complexity $3N^3log(N)$. This is the first step (see figure \ref{fig:PIPELINE3}), the next is the spherical-map transform which is complexity $45N^3$. If processed on the GPU the performance becomes 45 operations per voxel assuming that one voxel is assigned to one unit of processing. A 3D transform is 30 operations per voxel, 2D phase correlation requires 15 operations to multiply the frequency spectra and $2N^2log(N)$ operations to do the 2D FFT. Finally a projection map transform requires 1 operation per voxel. \\

In total, the proposed method requires $2 \times$ 3D FFTs, $2 \times$ spherical-map transforms, $1 \times$ 3D geometrical transformation, $3 \times$ 2D phase correlations and $4 \times$ projection map transforms. The total complexity is added up for all of these functions and given in equation \ref{eqn:FULLPERF2}. \\

\begin{equation} \label{eqn:FULLPERF2}
6log(N)\times (N^3 + N^2) + 169
\end{equation}

Figure \ref{fig:perfComp} provides a visualization of the performance improvement which the proposed method achieves over the original Fourier volume registration approach. It is clear that the proposed method is around 3 times faster
than the original Fourier based volume registration approach. This is due to the reduction in the amount of data to process afforded by the novel spherical-map transform and orthogonal projection methods.

\begin{figure}[!htb]
\centering
\includegraphics[width=4.0in]{images/ch2/perfcomp}
\caption{Comparison of performance between volume registration and the proposed speed up for different volume sizes.}
\label{fig:perfComp}
\end{figure}
