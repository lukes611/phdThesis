
This chapter presented a set of novel 3D reconstruction methods as well as a novel hierarchical compression method. The novel 3D reconstruction methods include: the FVR (a novel technique bringing Fourier registration to 3D reconstruction), MVVR (an extension of the FVR method enabling it to process monocular video data), FFVR (a novel extension to the FVR which improves upon the performance of the FVR method) and the FVR-3D method (another extension of the FVR method which enables the FVR to register against all six degrees of freedom). The novel hierarchical method proposed is the Plane-Tree method, which is based off of the Octree data structure. \\

These novel proposed techniques help solve the research questions. The FVR related techniques assist in answering the first research question ``Can Fourier based registration techniques improve accuracy and noise robustness in 3D reconstruction applications?'' These algorithms are compared to the state-of-the-art in Chapter \ref{ch:Experiments} to provide insight into their performance. The Plane-Tree can be used to answer the second research question ``Can hierarchical techniques improve compression, storage and processing of 3D reconstruction data?'' The Plane-Tree is a hierarchical compression scheme which can be compared with the state-of-the-art compression methods from the literature. The next chapter presents experiments which compare these novel techniques with the methods from the literature to answer the research questions. \\
