\begin{savequote}[8cm]
  ``I just wondered how things were put together.''
  \qauthor{Claude Shannon}
\end{savequote}
\makeatletter
\chapter{Experiments}
\label{ch:Experiments}

\section{Experiments}
Several experiments were designed in order to evaluate both the FVR method as well as the Plane-Tree and 3D Shade-Tree method. The types of these experiments as well as their conditions and results are presented in this section. In the next section (section \ref{TestDataSection}) the test data used in experimentation is introduced. Following that (in section \ref{ToolsSection}), the tools used to run these experiments are also introduced along with various error metrics (section \ref{metricsSection}) which provide means for comparisons between algorithms. \\

Then different results for experiments analysing the performance of FVR and comparing it to methods used in the current literature are presented. These experiments compare the FVR's ability to track camera movements (sections \ref{Sec:CamTransTrackExp} and \ref{Sec:CamRoteTrackExp}), its qualitative performance (section \ref{Sec:FVRQual1Exp}) and its robustness to noise (section \ref{Sec:FVRMotionExp}). Experiments evaluating the performance of Monocular FVR are then presented in section \ref{Sec:MonocularExperimentsSection} followed by experiments comparing FVR-3D to commonly used procedures from the literature. After these experiments, the Plane-Tree is compared to the Octree. These experiments compare both method's efficacy in compressing 3D mesh data. The Plane-Tree is also compared to various state-of-the-art compression techniques from the literature. Next, qualitative compression results are presented comparing the Plane-Tree to several other methods. Finally, the Plane-Tree is compared to both the Octree and the 3D-ShadeTree in compressing 3D reconstructions. \\



\section{Test Data}
\label{TestDataSection}



In order to assess results using an active camera sensor, test data was generated using the ASUS Xtion PRO LIVE camera. Because these data sets were generated during experimentation, the opportunity was taken to generate scenes, each captured using a specific type of camera movement. Some scenes were captured by rotating the camera about the x or y-axes, others by translating the camera. By testing with different movements, future algorithms may be constructed by switching to different registration methods based on camera movement. The different camera transformations recorded in the test data include: translation (left and right), and rotation about different axes (x and y axis). \\

Some samples from these data sets are shown in appendix \ref{ActiveSensorDataSet}. The first scene: the Apartment Texture Rotate scene was taken by rotating the camera around the y-axis across an apartment. This scene contains a lot of texture information. The Apartment Texture X Axis scene is similar in terms of texture but contains both x and y axis rotation. This tests the FVR's ability to handle multiple axes of rotation. The Office textured blind-spot rotation scene is a textured office scene where the camera is rotated about the y-axis. The scene is focused on a large divider which separates two desks. The divider may confuse registration methods which rely too heavily on minimization by aligning the large divider as a priority rather than taking into account the smaller details within the scene. An example of such an algorithm would be ICP and its derivatives. The Office scenes contain a plenty of usable texture and different sets were created by translating, rotating about the y-axis and rotating about the x-axis. \\

%mvvr
The MVVR method was analysed using test data generated using a Microsoft passive RGB camera or the rgb component of the ASUS Xtion PRO LIVE active camera. The data output is in basic video format, and all depth information was generated implicitly by the MVVR algorithm. This data was captured by moving the camera whilst focusing on a set of textured boxes within an indoor environment. Several frames were captured at 30 frames per second and registered for analysis of the MVVR method qualitatively. Quantitative tests of the MVVR make use of the same data as tests for the FVR method (testing translation and rotational registration error). \\

To assess qualitative reconstruction, three scenes were captured and registered using the FVR method. These scenes include the apartment scene, office scene and the garden scene. All three scenes were captured using the Asus XTION PRO live active camera at 30 frames per second. For each scene, only one in every 30 frames were registered, constituting real time FVR performance. The apartment scene was generated by moving and rotating the camera around an apartment living room before moving towards the kitchen area. This scene contains an abundance of features and would be considered a basic test for most 3D reconstruction algorithms. The second scene is the Office scene. This scene was generated by rotating the camera whilst zooming in and out on different objects within the room. Again this scene was reconstructed by registering every 30th frame. Despite both of the scenes being trivial to reconstruct, most algorithms (especially feature matching based methods) would find registering large rotations (such as those present in this data) difficult. The Garden scene is a difficult scene to reconstruct regardless of the reconstruction algorithm or technique used for registration. This scene was captured by rotating the camera around a garden outside the university. The scene contains many textures which are similar at the local level but are located in totally different locations. Therefore the scene is difficult to reconstruct using local and feature based methods such as FM+RANSAC and ICP. \\


\begin{figure}[!htb]
\centering
\includegraphics[width=4.0in]{images/experiments/test_data/modelsused}
\caption{Models used to assess the Plane-Tree compression algorithm.}
\label{fig:MODELSUSEDA}
\end{figure}

To test the Plane-Tree algorithm, several 3D objects commonly used within the literature were used to compare with known state of the art compression methods in terms of 3D mesh compression. The models used for testing are shown in figure \ref{fig:MODELSUSEDA}. These include: the bunny model with 34835 vertices, the rabbit model with 67039 vertices, the fandisk model with 6475 vertices and the horse model with 19851 vertices. \\

Several 3D reconstructions generated from the data set in appendix \ref{AppendixA} were used also to assess the Plane-Tree in compressing 3D reconstructions for storage and transmission. \\



\section{Tools}
\label{ToolsSection}

In the experiments, various machines, devices and software were used to generate and perform tests. This equipment is discussed here. All experiments were performed on two machines. One is an Asus laptop running both Windows 10 and Ubuntu 16. This laptop has an Intel i7 CPU and an NVIDIA GeForce 840 M GPU with 4 GB of RAM. The other is a Dell desktop computer running Windows 7. This machine has an Intel i5 CPU and 4 GB of RAM. \\ 

Both the Visual Studio C/C++ compiler on Windows and the GCC C/C++ compiler on Ubuntu were used to write programs for testing purposes. C++ version 17 was primarily used to write programs. Libraries used include: the OPEN-CV 3 computer vision library used for capturing, writing and processing image data, and CUDA used for writing General Purpose GPU (GPGPU) programs. Both Visual Studio (12 and 15) and Code Blocks 16 were used to write programs using the libraries and compilers. The METRO 3D object comparison tool was used for comparing 3D objects in Plane-Tree experiments. \\

To capture depth (RGB-D) video frames, the Asus Xtion Pro Live active camera was used to capture $640\times480$ video frames at 30 frames per second. This camera can capture depth between 0.8 and 3.5 m and was used to capture most of the test-data used in FVR experiments. \\ 

All source code and experiments are available online. The link for the FVR method is \url{https://github.com/lukes611/phdThesis} \href{https://github.com/lukes611/phdThesis} {FVR} and the link for the Plane-Tree source code is \url{https://github.com/lukes611/PlaneTree} \href{https://github.com/lukes611/PlaneTree}{Plane-Tree}. Further discussion about the metrics used for testing these algorithms is given in section \ref{metricsSection}. \\  


\section{Error metrics}
\label{metricsSection}

Several metrics are used to assess the set of proposed 3D reconstruction algorithms as well as the proposed Plane-Tree and other compression methods. These metrics are presented and defined mathematically here. Assessing both the 3D reconstruction algorithms as well as the lossy 3D data compression techniques requires comparing output 3D models. In the case of 3D reconstructions, we can compare the error between the registered frame and the ground truth, the larger the error, the worse the registration. Alternatively, a measurement may be used to measure the difference between two frames before and after they are registered. If the error is reduced or almost zero after registration, then the registration method may be deemed correct. If the error is larger or almost the same post registration, the registration maybe considered incorrect or there possibly was little camera movement. \\


In both cases, a robust way to measure the similarity/difference between two 3D objects must be used. To compute this error, we use nearest neighbour functions to measure the closest point in one model to the closest point in the next model. For example, given model A and model B, the closest point $B_j$ in $B$ for point $A_i$ in $A$, is used in summation of the total error. The closest point $B_j$ to $A_i$ is simply the nearest neighbour of $A_i$. This can be formally described as a function of the 3D point list $V$ and a given point $p$. This is described mathematically in equation \ref{eqn:NN}. Here, the result is a point $q$ in $V$, in which the distance between $q$ and $p$, according to function $Dist(x,y)$, is shorter for $q$ given $p$ than any other point $k$ in $V$.  \\

\begin{equation} \label{eqn:NN}
NN(p, V) =  \{ q \in V | (Dist(q, p) < Dist(k, p))  \forall k \in V \}
\end{equation}

For all purposes within this research, the function $Dist(x,y)$ is simply the Euclidean distance between the two input points. Using this definition of a nearest neighbour, several metrics may be used to compute the distance between two models. The one-way distance between two models is defined as the summation of distances for each point in one model to its nearest neighbour from the other model. The one-way mean error between two 3D models, $P$ and $Q$ is given in equation \ref{eqn:HDOW}. The full mean error between two objects is then computed as the average of the $Mean-Error_{one-way}$ function from model $P$ to $Q$ and model $Q$ to $P$. This is defined in equation \ref{eqn:MEANERRORMETRIC1}. \\


\begin{equation} \label{eqn:HDOW}
ME_{1-way}(P, Q) = \frac{1}{N}\sum_{k=0}^{N} Dist(P_k, NN(P_k, Q))
\end{equation}


\begin{equation} \label{eqn:MEANERRORMETRIC1}
ME(P, Q) = \frac{ME_{1-way}(P,Q) + ME_{1-way}(Q,P)}{2}
\end{equation}

Other metrics may also be used, for example the mean squared error can be used instead of mean error. The mean squared error (MSE) may then replace the mean error used in equation \ref{eqn:HDOW}. The one-way and full error functions based on the MSE are provided in equations \ref{eqn:MSEOW} and \ref{eqn:MEANSQERRORMETRIC1}. \\


\begin{equation} \label{eqn:MSEOW}
MSE_{1-way}(P, Q) = \frac{1}{N}\sum_{k=0}^{N} Dist(P_k, NN(P_k, Q))^2
\end{equation}


\begin{equation} \label{eqn:MEANSQERRORMETRIC1}
MSE(P,Q) = \frac{MSE_{1-way}(P,Q) + MSE_{1-way}(Q,P)}{2}
\end{equation}


The mean error based on the nearest neighbour is often referred to as the Hausdorff error. In this research, we used the Hausdorff error and the Mean Squared Error (MSE) based on the nearest neighbour technique, as well as a percentage of total matches. The one-way percentage of total matches is the computation of the percent of points from one model which have a nearest neighbour with a distance below a given threshold. This metric is defined in equation \ref{eqn:PERCMATCHOW}. \\

\begin{equation} \label{eqn:PERCMATCHOW}
PM_{1-way}(P, Q) = \frac{100}{N}\sum_{k=0}^{N} x, where
  \begin{cases}
    x=1       & \quad \text{if } Dist(P_k,NN(P_k,Q)) < \text{threshold}\\
    x=0  & \quad otherwise\\
  \end{cases}
\end{equation}

Following this, the full Percent-Match function may be defined as the average of the $PM_{1-way}$ function in both directions. These three metrics are used along with several others in 3D reconstruction experiments. Additionally, the camera tracking error is computed as the euclidean distance between the ground truth camera movement from one frame to another and the estimated camera movement computed via the FVR method. The voxel error is also measured. Since the camera distance metric measures the real positions moved by the camera, it theoretically has the accuracy of the real number system. Conversely, FVR uses voxel spaces (3D volumes) to find the translation and rotation (location and pose) between frames. Therefore, the result must be quantized. By quantizing the ground truth prior to comparison, the accuracy up to the resolution of the voxel grid may be computed. This is important since it measures the error of the FVR 3D reconstruction method without penalizing it based on the effects of quantization which reduce the larger the volume. \\

In experiments comparing the Plane-Tree with state-of-the-art algorithms, the root mean squared error is also used. The root mean squared error is simply the square root of the mean squared error function defined in equation \ref{eqn:MEANSQERRORMETRIC1}, that is $RMS(P,Q) = \sqrt{MSE(P,Q)}$. The Plane-Tree is also assessed in terms of 3D reconstruction occupancy grid compression. To assess the difference between the original 3D occupancy grid and the lossy compressed version, the Peak Signal to Noise Ratio metric (PSNR) is used. This is a function of the voxel-wise mean squared error function. Given the entire 3D volume, the mean squared error is computed between the original 3D reconstruction and the compressed 3D reconstruction. The mean squared error over two 3D volumes is defined in equation \ref{eqn:MEANSQERRORMETRIC2} as the summation of the voxel-wise squared error divided by the number of voxels.

\begin{equation} \label{eqn:MEANSQERRORMETRIC2}
MSE_{volume}(V_1,V_2) = \frac{1}{N^3}\sum_{z=0}^{N}\sum_{y=0}^{N}\sum_{x=0}^{N} \left(V_1(x,y,z) - V_2(x,y,z)\right)^2
\end{equation}

The PSNR metric is then defined as in equation \ref{eqn:PSNR1}. This metric is commonly used in image compression comparisons and since 3D images are being compared in this test, it is used here also. The PSNR metric is inversely related to error, so the larger the PSNR the lower the error between the two models.

\begin{equation} \label{eqn:PSNR1}
PSNR(V_1,V_2) = 10 \times \log10{\frac{255^2}{MSE_{volume}(V_1, V_2)}}
\end{equation}



%the basic tracking stuff
\section{Camera Translation Tracking}
\label{Sec:CamTransTrackExp}

Experiments measuring the error after moving the camera using the FVR based methods as well as other methods from the literature are presented in this section. For these experiments, one camera frame of an indoor environment was captured using the ASUS Xtion PRO LIVE active camera. The camera was then moved (translated) by different amounts including: 5 centimeters, 10 centimeters and 15 centimeters. The 2D and 3D Feature matching / Ransac methods (FM2D, FM3D), ICP and PCA methods were all tested. Additionally, results for the FVR, FFVR and FVR3D method are presented. FM3D, FVR and FVR3D require the 3D frames be quantized into $256\times 256\times 256$ volumes. \\

Different levels of noise were added to both frames prior to 3D registration in order to measure each method's ability to register frames with large amounts of noise. The Signal to Noise Ratio (SNR) metric is used to describe the noise added to both captured frames, prior to any registration. This noise effects any registration method's ability to accurately estimate the transformation separating two sets of data. In a given experiment, a noise range value of $x$ means random noise was added in the range [$\frac{-x}{2}$, $\frac{x}{2}$] to a signal within the range [0, 1]. Reconstruction error is measured in mean squared error (as presented in section \ref{metricsSection}). \\

Table \ref{table:trans} shows the results of these tests, they illustrate the FVR's robustness to noise whilst registering frames which are captured during different camera translation intervals. Each sub-table is labelled with a distance in centimeters in which frames were separated prior to registration. The first two columns represent the amount of noise added both in terms of noise-range and the subsequent Signal to Noise ratio computed from it. The rest of the columns represent the registration error using the Mean Squared Error metric.  \\

%translation
\begin{table}[!htb]
\centering
\scalebox{1.0}{
\begin{tabular}{ccccccccc}
\\ \textbf{5cm} &   &   &   &   &   &   &   &   \\ 
Noise & SNR & \textbf{FM2D} & \textbf{FM3D} & \textbf{ICP} & \textbf{PCA} & \textbf{FVR} & \textbf{FFVR} & \textbf{FVR3D} \\ \hline
0 & $\infty$ & 5.07 & 4.75 & 12.35 & 3.73 & 4.04 & fail & 6.24 \\
0.1 & 20db & 18.96 & fail & fail & 2.33 & 2.73 & 14.6 & 5.63 \\
0.25 & 12db & 5.12 & 4.3 & 2.03 & 4.2 & 4.13 & 5.51 & 2.3 \\
0.5 & 6db & 5.61 & 1.95 & 3.71 & 6.83 & 3.7 & 7.73 & 2.83 \\
0.75 & 2.5db & 5.02 & 8.57 & 6.41 & 20.94 & 2.24 & 6.5 & 4.33 \\
\\ \textbf{10cm} &   &   &   &   &   &   &   &   \\ 
Noise & SNR & \textbf{FM2D} & \textbf{FM3D} & \textbf{ICP} & \textbf{PCA} & \textbf{FVR} & \textbf{FFVR} & \textbf{FVR3D} \\ \hline
0 & $\infty$ & 15.35 & fail & 10.26 & 16.87 & 3.47 & fail & 7.07 \\
0.1 & 20db & 18.2 & 15.36 & 6.24 & fail & 3.89 & fail & 11.45 \\
0.25 & 12db & fail & 24.12 & 6.02 & 20.12 & 5.4 & fail & 7.53 \\
0.5 & 6db & fail & 4.42 & 5 & 3.96 & 4.65 & fail & 9.47 \\
0.7 & 2.5db & 7.27 & 8.57 & 3.98 & 7.85 & 15.45 & fail & 4.3 \\
\\ \textbf{15cm} &   &   &   &   &   &   &   &   \\ 
Noise & SNR & \textbf{FM2D} & \textbf{FM3D} & \textbf{ICP} & \textbf{PCA} & \textbf{FVR} & \textbf{FFVR} & \textbf{FVR3D} \\ \hline
0 & $\infty$ & fail & fail & 7.85 & fail & 25.87 & fail & fail \\
0.1 & 20db & fail & fail & 7.13 & fail & 12.27 & fail & fail \\
0.25 & 12db & 15.17 & fail & fail & fail & 8.1 & 12.52 & fail \\
0.5 & 6db & fail & fail & 11.11 & fail & 10.13 & 11.09 & fail \\
0.7 & 2.5db & fail & fail & 10.31 & 20.59 & 7.79 & fail & fail \\
\\
\end{tabular}}
\\
\caption{Translation Tracking}
\label{table:trans}
\end{table}


Results show that the basic FVR method is generally the most accurate in terms of registration across the range of camera translation magnitudes and noise levels. In comparison, the FFVR method reduces processing time at the expense of accuracy, results show that in general it is only capable of up to 5cm of translation registration. FVR3D and the 2D feature matching method are capable of registering 5-10cm with larger levels of noise. However, both methods fail to register the full 15cm of camera translation well. \\

ICP performed next best, capable of performing well up to 15cm of translation, but failing to register twice during the tests. PCA was able to handle up to 10cm of too but often had larger registration errors. For the 15cm translation tests, the FVR performed the best followed by ICP, which failed once but outperformed FVR within the two lowest noise brackets. \\


It can be shown that, despite being limited to a single axis of rotation, the FVR algorithm can consistently register camera movements up to 15cm better than other algorithms from the literature. To put these camera translations into perspective at video frame rates, a displacement of 10cm per frame equates to camera velocity of 3 meters per second, this is twice the average person's walking speed making both the FVR method and FVR3D suitable for a majority of applications. \\


\section{Camera Rotation Tracking}
\label{Sec:CamRoteTrackExp}

Table \ref{table:rote} shows results for camera rotation experiments. These experiments were captured with the ASUS Xtion PRO LIVE camera using the same scene as results in section \ref{Sec:CamTransTrackExp}. 3D frames of these scenes are separated by 10, 20 and 30 degrees of camera rotation about the y-axis. These degrees were chosen because they are significantly large enough to be difficult for these algorithms to register against.  \\

Again, different levels of noise were added to each frame prior to registration. This experiment was designed to test the robustness of the FVR based methods in registering camera pose. \\


%rotation
\begin{table}[!htb]
\centering
\scalebox{1.0}{
\begin{tabular}{ccccccccc}
\\ \textbf{10deg} &   &   &   &   &   &   &   &   \\ 
Noise & SNR & \textbf{FM2D} & \textbf{FM3D} & \textbf{ICP} & \textbf{PCA} & \textbf{FVR} & \textbf{FFVR} & \textbf{FVR3D} \\ \hline
0 & $\infty$ & 0.17 & 9.77 & 0.19 & 0.3 & 0.18 & 0.35 & 4.05 \\
0.1 & 20db & 0.21 & 0.41 & 0.17 & 0.67 & 0.15 & 0.21 & 10.54 \\
0.25 & 12 & 0.16 & 0.34 & 0.13 & 0.39 & 0.15 & 0.44 & 5.44 \\
\\ \textbf{20deg} &   &   &   &   &   &   &   &   \\ 
Noise & SNR & \textbf{FM2D} & \textbf{FM3D} & \textbf{ICP} & \textbf{PCA} & \textbf{FVR} & \textbf{FFVR} & \textbf{FVR3D} \\ \hline
0 & $\infty$ & 1.37 & 13.1 & 1.81 & 16.93 & 7.29 & 1.32 & 3.18 \\
0.1 & 20db & 1.35 & 14.32 & 15.29 & 2.97 & 0.7 & 2.03 & 1.92 \\
0.25 & 12db & 1.6 & 3.26 & 14.7 & 0.64 & 1.54 & 4.29 & 2.24 \\
\\ \textbf{30deg} &   &   &   &   &   &   &   &   \\ 
Noise & SNR & \textbf{FM2D} & \textbf{FM3D} & \textbf{ICP} & \textbf{PCA} & \textbf{FVR} & \textbf{FFVR} & \textbf{FVR3D} \\ \hline
0 & $\infty$ & 3.87 & 10.15 & 3.16 & 9.97 & 2.11 & 5.68 & 3.63 \\
0.1 & 20db & 3.77 & 16.65 & 39.37 & 77.6 & 3.84 & 2.82 & 1.39 \\
0.25 & 12db & 3.68 & 2.76 & 5.12 & 5.78 & 3.3 & 7.15 & 6.38 \\
\\
\end{tabular}}
\\
\caption{Rotation Tracking}
\label{table:rote}
\end{table}

In the first sub-table, registration errors for 10 degrees of rotation are presented. Results show that the FVR method outperforms the other methods. Here, ICP performs next best followed by the 2D feature matching method. This was expected as the FVR method was designed to be both robust to noise and to handle larger rotations. It can also be seen that in the 20 degree and 30 degree tests, the FVR method also beats both of these algorithms in terms of having to lower registration error. FFVR also worked well at different noise levels up to 30 degrees of rotation. FVR3D was found to be as robust as ICP at registering rotation but not at accurate as the 2D feature matching method. \\ 

It should be noted that twelve degrees of camera rotation per frame is almost a full rotation per second in video rates. This is so fast that most cameras would acquire too much motion blur for registration to be possible. Therefore this test indicates the robustness of the FVR method in comparison with the other algorithms within the context of camera pose estimation. \\


\section{Reconstructed Scenes}
\label{Sec:FVRQual1Exp}

\begin{figure}[!htb] 
        \centering
        \begin{subfigure}[b]{3.0in}
                \includegraphics[width=3.0in]{images/ch2/unit21}
                \caption{Apartment}
                \label{fig:RECON_UNIT}
        \end{subfigure}
        \begin{subfigure}[b]{3.0in}
                \includegraphics[width=3.0in]{images/ch2/officeA}
                \caption{Office}
                \label{fig:RECON_OFFICE}
        \end{subfigure}
        \begin{subfigure}[b]{3.0in}
                \includegraphics[width=3.0in]{images/ch2/outdoorA}
                \caption{Garden}
                \label{fig:RECON_GARDEN}
        \end{subfigure}
       \caption{Reconstructed Scenes.}
       \label{fig:RECONSTRUCTIONS}
\end{figure}

Qualitative experiments also show the ability of the FVR registration method to reconstruct 3D scenes. In these experiments, two indoor environments (Apartment and Office) as well as one outdoor environment (Garden) were reconstructed and are shown in figures \ref{fig:RECON_UNIT}, \ref{fig:RECON_OFFICE} and \ref{fig:RECON_GARDEN} respectively. \\

The Apartment reconstruction was recorded by moving the ASUS Xtion PRO LIVE active camera through a room and rotating the camera. Each frame was registered using the FVR algorithm. Some of the frames in the apartment scene contain walls which have few features. Between frames, walls also had colour contrast shifts. These shifts are due to the ASUS camera's automatic contrast feature which adjusts contrast based on colour histograms. Despite these setbacks, accurate 3D reconstruction was achieved by the FVR method as illustrated in figure \ref{fig:RECON_UNIT}. \\


The office reconstruction was also generated by rotating the ASUS Xtion PRO LIVE active camera about the y-axis while moving the camera around the room. This time, during rotation, the camera was focused on both foreground and background objects. Here, the entire video sequence was accurately registered. It can be seen that despite the foreground and background focus, the global reconstruction is accurate. This scene, as in the apartment scene has usable texture which should not cause large amounts of texture confusion. These qualitative experiments show that despite being a closed form solution, the FVR has reconstruction accuracy comparable to existing feature based SLAM methods. \\


Typical feature based methods work well with indoor environments where local features are readily distinguishable and easy to match. They do not tend to work as well in complex outdoor scenes where feature confusion is likely. To assess performance in such outdoor scenes, a garden scene containing bushes, plants and a ground covering of bark and rocks was captured for testing. Again, this scene was captured using the ASUS Xtion PRO LIVE active camera moving around the out-door garden. The proposed FVR method was able to produce a good quality reconstruction of this garden scene, as shown in figure \ref{fig:RECON_GARDEN}. This shows that reconstruction approaches which integrate or make use of the FVR registration method may have an advantage in performing 3D reconstructions in these types of scenes, scenes which are of common disturbance to many existing feature matching methods, as expressed in the literature.   



\section{Monocular Camera Translation Tracking}
\label{Sec:MonocularExperimentsSection}

\section{FVR vs. Existing Techniques}
\subsection{Algorithms} \label{AlgorithmsSection} Different 3D-registration algorithms were implemented to test the Fourier Volume Registration (FVR) method. Feature matching methods are important to compare with because they are still dominant and very successful in image processing and computer vision. In this research we show that FVR is competitive with feature matching methods whilst beating them in certain contexts (such as little textured scenes or scenes where texture confusion may occur). \\ 

We test with both 2D feature matching and 3D feature matching. In 2D feature matching, the features are found and matched between a pair of 2D-images, then RANSAC is used with the corresponding matches and true 3D point to compute pose. The pose is then used to reconstruct the scene. We found that SURF performed best out of the other feature matching methods, so SURF was used in experiments. The 2D feature matching method is limited as it cannot register frames which have too few features or frames which contain texture confusion. It is also not able to handle wide base-lines. \\

We also test 3D-feature matching using an implementation of SIFT in 3D. This algorithm was tested and written in C/C++ and like the 2D counterpart, is also susceptible to failed registration in scenes with too few features and texture confusion but it is able to handle wide base-lines since it works in 3D. //

Another algorithm used in experiments is Iterative Closest Point or ICP. This method has become very popular in 3D reconstruction and works well on most scene types. One disadvantage is that this method may get stuck in a local minima and fail to register correctly. This typically occurs when registering against wide-baselines. \\

Another algorithm present in the experiments is Principal Components Analysis (PCA). This algorithm is used to find the mean and principal components of a multi-dimensional data set. This is useful for registration purposes as it works on wide-baselines, is very fast and provides additional information about a scene. The downside is that it is very susceptible to noise and misaligned data. The proposed FVR method makes use of information from PCA so it is important to compare the two to find out what improvements are made by FVR over PCA. \\

The final algorithm tested is the proposed FVR algorithm, which uses both PCA and Fourier Phase Correlation to find the registration transformation between two 3D data-sets as described in \ref{FullRecovery3DSection}. This algorithm was proposed to handle general transformations in terms of rotation, scaling and translation. It was also designed to be able to handle noisy data, data with texture-confusion and data with little or no texture. This makes it a viable option in the 3D registration and pose estimation research areas.
\begin{figure}[!htb] 
        \centering
        \begin{subfigure}[b]{1.8in}
                \includegraphics[width=1.8in]{images/results/compression/bunnysota}
                \caption{Bunny Model}
                \label{fig:SA_BUNNY}
        \end{subfigure}%
        \begin{subfigure}[b]{1.8in}
                \includegraphics[width=1.8in]{images/results/compression/fandisksota}
                \caption{Fandisk Model}
                \label{fig:SA_FANDISK}
        \end{subfigure}
        
        \begin{subfigure}[b]{1.8in}
                \includegraphics[width=1.8in]{images/results/compression/horsesota}
                \caption{Horse Model}
                \label{fig:SA_HORSE}
        \end{subfigure}%
        \begin{subfigure}[b]{1.8in}
                \includegraphics[width=1.8in]{images/results/compression/rabbitsota}
                \caption{Rabbit Model}
                \label{fig:SA_RABBIT}
        \end{subfigure}
       \caption{Rate-Distortion graphs comparing the Plane-Tree to different state of the art codecs.}
       \label{fig:SOTAEXPS}
\end{figure}

In figures \ref{fig:SA_BUNNY}, \ref{fig:SA_FANDISK} and \ref{fig:SA_HORSE} we compare the Plane-Tree with the state of the art transform methods by Bayazit et al \cite{Bayazit103DMesh} and Khodakovsky et al \cite{Khodakovsky00Progressive}. \\

Experiment results comparing the Plane-Tree compression system proposed in this work with the transform-based method by Khodakovsky et al are shown in figure \ref{fig:SA_BUNNY}. This Rate-Distortion graph shows that the Plane-Tree has similar performance with the method by Khodakovsky et al. At low bit-rates the Plane-Tree is highly competitive the with performance matching Khodakovsky's. The competitiveness extends to higher bit-rates as well for the bunny model. \\

Figure \ref{fig:SA_FANDISK} shows comparisons between the Plane-Tree as well as methods by Khodakovsky et al and Bayazit et al. Here, both the Plane-Tree and the method by Khodakovsky et al perform similarly as in the bunny model experiment (figure \ref{fig:SA_BUNNY}). The Plane-Tree stays competitive with that state of the art method, and improves upon the compression performance in comparison to the spectral compression method by Bayazit et al. It can be seen that at low to mid bit-rates, the Plane-Tree method compressed the Fandisk model at a higher level of quality for a given bit-rate. \\

The Plane-Tree was also compared with Bayazit et al using another model, the horse model. Figure \ref{fig:SA_HORSE} shows the result of this experiment. The Plane-Tree method outperforms the transform based method of Bayazit et al whilst decreasing coding complexity compared to the complicated transform method. Overall, in this experiment, the proposed Plane-Tree method remains competitive at higher bitrates, and it outperforms the method by Bayazit et al at lower bitrates. \\

Finally the Plane-Tree is compared with the state of the art low-bitrate compression system FOLProM presented by Peng et al \cite{Peng10Feature}. Unlike the other experiments, the mean-error metric is used in this comparison, as the results presented by Peng et al used this metric. It can be seen that at low-bitrates (below 2 bits per vertex), the Plane-Tree method outperforms this state of the art low-bitrate compression system. \\


In conclusion, these experiments show that the Plane-Tree is extremely competitive with the state of the art compression systems at high bit-rates. At lower bit-rates it improves upon the results by these methods. This algorithm is also powerful in that it may be used to compress 3D volumetric data as well as point-cloud and mesh data. This makes it an interesting candidate for 3D reconstruction compression no matter the format output by a given 3D reconstruction method. \\

\subsection{Noise Robustness Comparisons}

Here we present results for noise robustness experiments. Tests were performed by transforming data by various transforms, and adding different amounts of noise. Noise here is measured in both SNR and noise range. A noise range of 1.0 means random noise within the range [$-0.5$, $0.5$] was added. SNR is measured in decibels. 

\begin{figure*}[t]
\centering
\includegraphics[width=6.0in]{images/results/noise/YRNoise0}
\caption{Percent Match for Y-Axis Registration with an Infinite SNR and noise range of $0$.}
\label{fig:YRNoise0}
\end{figure*}

In the first set of experiments, scanned 3D data was rotated about the Y-axis. Figure \ref{fig:YRNoise0} shows results with an infinite SNR. The 2D-FM method and FVR achieved perfect results. PCA performed next best, but failed to register degrees above 70. Since no noise was present, PCA was truly accurate, however some axes were flipped. This may be fixed by testing the data for flipped axes. FM-3D and ICP fell away much earlier. ICP is not good at registering significant transforms and FM-3D also does not perform well as quantization errors are higher since we use volume sizes of $128^3$.

\begin{figure*}[t]
\centering
\includegraphics[width=6.0in]{images/results/noise/YRNoise1}
\caption{Percent Match for Y-Axis Registration with an SNR of and noise range of $10300$ and noise range of $1.0$.}
\label{fig:YRNoise1}
\end{figure*}

For the results in figure \ref{fig:YRNoise1} we dropped the SNR to $10300$. Here we see similar results but FM-2D has begun to show some errors with larger rotations (remember rotations above 180 can be considered smaller but negative rotations). In figure \ref{fig:YRNoise2} we reduced the SNR to $2580$ which caused more error in FM-2D. These results suggest that for simple Y-axis rotation, FVR is superior in terms of noise robustness. 

\begin{figure*}[t]
\centering
\includegraphics[width=6.0in]{images/results/noise/YRNoise2}
\caption{Percent Match for Y-Axis Registration with an SNR of 2580 and noise range of $2.0$.}
\label{fig:YRNoise2}
\end{figure*}


\begin{figure*}[t]
\centering
\includegraphics[width=6.0in]{images/results/noise/TransNoise0}
\caption{Percent Match for Translation Registration with an infinite SNR and noise range of $0$.}
\label{fig:TNoise0}
\end{figure*}


\begin{figure*}[t]
\centering
\includegraphics[width=6.0in]{images/results/noise/TransNoise1}
\caption{Percent Match for Translation Registration with an SNR of and noise range of $10300$ and noise range of $1.0$.}
\label{fig:TNoise1}
\end{figure*}


\begin{figure*}[t]
\centering
\includegraphics[width=6.0in]{images/results/noise/TransNoise2}
\caption{Percent Match for Translation Registration with an SNR of and noise range of $2580$ and noise range of $2.0$.}
\label{fig:TNoise2}
\end{figure*}

%PTPTPT

plane tree is compared with sota mesh compression - proven by my paper published and in terms of compressing 3d reconstructions we compare it to the common OT representation. \\

Experiments reveal the Plane-Tree outperforms the original octree and outperforms some state of the art transform methods at low-bitrates. We compare our method with several state of the art algorithms which correspond to transform based methods \cite{Khodakovsky00Progressive,Bayazit103DMesh} and low-bitrate codecs \cite{Peng10Feature}. To compare algorithms quantitatively, we employ rate distortion graphs. Since our algorithm is lossy, this indicates the amount of distortion (in the decoded model compared with the original) given a particular bitrate. To measure distortion, we employ the mean error and root mean squared error, to measure bitrate we use the number of bits per vertex. All measurements are scaled by the main diagonal of the input model. We also provide qualitative results showing the effectiveness of our method. \\

In experiments we use 4 models which are popular amongst researchers in the area. These models are chosen both because they are readily available and because there are available results for these models generated by other state of the art methods. Figure \ref{fig:MODSUSED} shows each model along with its reference name and number of vertices. \\

\section{Plane-Tree vs. Octree}
Figure \ref{fig:OTEXPS} shows rate-distortion graphs comparing the Plane-Tree with the Octree compression method. In these rate-distortion graphs we use the mean error between the decoded and input model as the distortion metric. Results show that for the Bunny and Fandisk experiments (figures \ref{fig:OG_BUNNY} and \ref{fig:OG_FANDISK}) the Plane-Tree has better quality for a given bitrate compared to the Octree. It is also evident that the Octree is unable to reach the level of quality the Plane-Tree reaches. In both cases, due to the Plane-Tree's ability to prevent further tree decomposition via its more accurate representation method, the Plane-Tree has a much lower error rate for a given bit-rate compared with the Octree method. \\

In the Horse model graph in figure \ref{fig:OG_HORSE} there is some overlap in the model quality (error rates). This overlap ranges from around $0.0025$ to $0.004$ in which for these levels of quality, the Octree requires around 8 times more storage space compared with the Plane-Tree. These qualitative results show how much of an improvement the Plane-Tree model codec is over the Octree method. \\

\begin{figure}[!htb] 
        \centering
        \begin{subfigure}[b]{2.8in}
                \includegraphics[width=2.5in]{images/results/compression/OTbunny}
                \caption{Bunny Model}
                \label{fig:OG_BUNNY}
        \end{subfigure}%
        \begin{subfigure}[b]{2.8in}
                \includegraphics[width=2.5in]{images/results/compression/OTFandisk}
                \caption{Fandisk Model}
                \label{fig:OG_FANDISK}
        \end{subfigure}
        
        \begin{subfigure}[b]{2.8in}
                \includegraphics[width=2.5in]{images/results/compression/OTHorse}
                \caption{Horse Model}
                \label{fig:OG_HORSE}
        \end{subfigure}%

       \caption{Rate-distortion graphs comparing the Plane-Tree with the Octree.}
       \label{fig:OTEXPS}
\end{figure}
\section{Plane-Tree vs. Existing Techniques}
\begin{figure}[!htb] 
        \centering
        \begin{subfigure}[b]{1.8in}
                \includegraphics[width=1.8in]{images/results/compression/bunnysota}
                \caption{Bunny Model}
                \label{fig:SA_BUNNY}
        \end{subfigure}%
        \begin{subfigure}[b]{1.8in}
                \includegraphics[width=1.8in]{images/results/compression/fandisksota}
                \caption{Fandisk Model}
                \label{fig:SA_FANDISK}
        \end{subfigure}
        
        \begin{subfigure}[b]{1.8in}
                \includegraphics[width=1.8in]{images/results/compression/horsesota}
                \caption{Horse Model}
                \label{fig:SA_HORSE}
        \end{subfigure}%
        \begin{subfigure}[b]{1.8in}
                \includegraphics[width=1.8in]{images/results/compression/rabbitsota}
                \caption{Rabbit Model}
                \label{fig:SA_RABBIT}
        \end{subfigure}
       \caption{Rate-Distortion graphs comparing the Plane-Tree to different state of the art codecs.}
       \label{fig:SOTAEXPS}
\end{figure}

In figures \ref{fig:SA_BUNNY}, \ref{fig:SA_FANDISK} and \ref{fig:SA_HORSE} we compare the Plane-Tree with the state of the art transform methods by Bayazit et al \cite{Bayazit103DMesh} and Khodakovsky et al \cite{Khodakovsky00Progressive}. \\

Experiment results comparing the Plane-Tree compression system proposed in this work with the transform-based method by Khodakovsky et al are shown in figure \ref{fig:SA_BUNNY}. This Rate-Distortion graph shows that the Plane-Tree has similar performance with the method by Khodakovsky et al. At low bit-rates the Plane-Tree is highly competitive the with performance matching Khodakovsky's. The competitiveness extends to higher bit-rates as well for the bunny model. \\

Figure \ref{fig:SA_FANDISK} shows comparisons between the Plane-Tree as well as methods by Khodakovsky et al and Bayazit et al. Here, both the Plane-Tree and the method by Khodakovsky et al perform similarly as in the bunny model experiment (figure \ref{fig:SA_BUNNY}). The Plane-Tree stays competitive with that state of the art method, and improves upon the compression performance in comparison to the spectral compression method by Bayazit et al. It can be seen that at low to mid bit-rates, the Plane-Tree method compressed the Fandisk model at a higher level of quality for a given bit-rate. \\

The Plane-Tree was also compared with Bayazit et al using another model, the horse model. Figure \ref{fig:SA_HORSE} shows the result of this experiment. The Plane-Tree method outperforms the transform based method of Bayazit et al whilst decreasing coding complexity compared to the complicated transform method. Overall, in this experiment, the proposed Plane-Tree method remains competitive at higher bitrates, and it outperforms the method by Bayazit et al at lower bitrates. \\

Finally the Plane-Tree is compared with the state of the art low-bitrate compression system FOLProM presented by Peng et al \cite{Peng10Feature}. Unlike the other experiments, the mean-error metric is used in this comparison, as the results presented by Peng et al used this metric. It can be seen that at low-bitrates (below 2 bits per vertex), the Plane-Tree method outperforms this state of the art low-bitrate compression system. \\


In conclusion, these experiments show that the Plane-Tree is extremely competitive with the state of the art compression systems at high bit-rates. At lower bit-rates it improves upon the results by these methods. This algorithm is also powerful in that it may be used to compress 3D volumetric data as well as point-cloud and mesh data. This makes it an interesting candidate for 3D reconstruction compression no matter the format output by a given 3D reconstruction method. \\
\section{Plane-Tree: Qualitative Results}

Figure \ref{fig:FIGS} in Appendix \ref{QualitativeLargeImage} shows qualitative results for the Plane-Tree. The third row shows the bunny, rabbit and horse models along with the number of bytes and bpv required to store them uncompressed. The first row shows each model compressed by the Plane-Tree with a threshold of 8.0. Despite these being crude approximations of the originals, the models are still distinguishable with around 1000 $\times$ less storage space required. \\

In the second row, each model was compressed with the Plane-Tree at a threshold of 1.0. In these experiments, there is little detail missing compared with the physical model. When looking at the bunny's legs, the same ripples are present. In the rabbit model, the outlines inside the ears and on the eyes are still present. On the horse model, the creases on the body and shoulder of the horse are still present. Here, the bunny is compressed to around 70 $\times$ less storage space, the rabbit at around 265 $\times$ less and the horse around 90 $\times$ less storage space. \\


Figure \ref{fig:qualSOTA1} shows the Bunny model compressed using the Plane-Tree method and two state-of-the-art methods, the valence method \cite{touma98triangle} and the spectral method \cite{Karni00Spectral}. The original model in Figure \ref{fig:PT_SOTAQ1_ORIG} requires 2,258,902 bytes for storage. The valence method and the spectral method require ~17,852 bytes to store the model and the compression effects are shown in Figures \ref{fig:PT_SOTAQ1_TG} and \ref{fig:PT_SOTAQ1_KG}. Noticeable artefacts not present in the original model are produced by both codecs. The model coded by the valence method fairs worse than the one coded by the spectral method. The model compressed by the Plane-Tree requires just over half the amount of bytes than the other codecs. The Plane-Tree coded model is smoother than the other models (despite flat shading) too. This is likely due to the way the Plane-Tree compression system approximates the original model using planes. This process doesn't introduce these artefacts, allowing the Plane-Tree to represent the original model more accurately. \\   

\begin{figure}[H] 
        \begin{center}
 		\begin{subfigure}[b]{6cm}
 			   \centering
 			   \includegraphics[width=5.8cm]{images/experiments/pt_qual/original1}
 			   \captionsetup{justification=centering}
                \caption{Original\\518.78 bpv\\2,258,902 bytes}
                \label{fig:PT_SOTAQ1_ORIG}
        \end{subfigure}%
        \begin{subfigure}[b]{6cm}
                \includegraphics[width=5.8cm]{images/experiments/pt_qual/pt_11004}
                \captionsetup{justification=centering}
                \caption{Plane-Tree\\2.52 bpv\\11,003 bytes}
                \label{fig:PT_SOTAQ1_PLT}
        \end{subfigure}
        \begin{subfigure}[b]{6cm}
                \includegraphics[width=5.8cm]{images/experiments/pt_qual/tg}
                \captionsetup{justification=centering}
                \caption{Valence Method \cite{touma98triangle}\\4.1 bpv\\17,852 bytes}
                \label{fig:PT_SOTAQ1_TG}
        \end{subfigure}%
        \begin{subfigure}[b]{6cm}
                \includegraphics[width=5.8cm]{images/experiments/pt_qual/kg}
                \captionsetup{justification=centering}
                \caption{Spectral Method \cite{Karni00Spectral}\\4.1 bpv\\17,852 bytes}
                \label{fig:PT_SOTAQ1_KG}
        \end{subfigure}
       \caption{The Bunny Model Compressed Using the Plane-Tree, Valence and Spectral Methods.}
       \label{fig:qualSOTA1}
       \end{center}
\end{figure}

Figure \ref{fig:qualSOTA2} shows the bunny model coded using the Plane-Tree and the codec by Khodakovsky et al. Both the Plane-Tree and method by Khodakovsky et al. require just 1349 bytes of storage to represent these low bit-rate models. Both methods however, use different strategies and introduce different styles of noise into the compressed model. The method by Khodakovsky et al. shrinks the models and warps the shape, simplifying it. The Plane-tree approximates the shape and positions of the vertices more closely but does not produce smooth surfaces. We conclude that both methods may be beneficial in different situations. The method by Khodakovsky et al. may be useful in situations where visual appeal are important. The Plane-Tree method may be more useful in applications where model shape and size are more important. \\

\begin{figure}[!htb] 
        \begin{center}
 		\begin{subfigure}[b]{6cm}
 			   \centering
 			   \includegraphics[width=5.8cm]{images/experiments/pt_qual/original2}
				\captionsetup{justification=centering}
                \caption{Original\\518.78 bpv\\2,258,902 bytes}
                \label{fig:PT_SOTAQ2_ORIG}
        \end{subfigure}%
        \begin{subfigure}[b]{6cm}
                \includegraphics[width=5.8cm]{images/experiments/pt_qual/planetree2_shade}
                \captionsetup{justification=centering}
                \caption{Plane-Tree\\0.31 bpv\\1,349 bytes}
                \label{fig:PT_SOTAQ1_PLT}
        \end{subfigure}
        \begin{subfigure}[b]{6cm}
                \includegraphics[width=5.8cm]{images/experiments/pt_qual/khodakovsky_shade}
                \captionsetup{justification=centering}
                \caption{Khodakovsky et al. \cite{Khodakovsky00Progressive}\\0.31 bpv\\1,349 bytes}
                \label{fig:PT_SOTAQ1_KHKY}
        \end{subfigure}
       \caption{The Bunny Model Compressed Using the Plane-Tree and the Wavelet Compression System Khodakovsky et al.}
       \label{fig:qualSOTA2}
       \end{center}
\end{figure}


\section{Plane-Tree: Reconstruction Compression}

In this section, experiments comparing the Plane-Tree with the Octree in terms of 3D frame or 3D occupancy reconstruction grid compression. Essentially this experiments compares the performance of the Plane-Tree and Octree in 3D occupancy grid compression. \\

The Octree is used for comparison since it is the closest relative of the Plane-Tree and existing methods which use the Octree are presently used in 3D reconstruction research. There is also a 3D implementation of the Interpolating Leaf Quad-Tree \cite{Lincoln13Interpolating} compression algorithm used in the comparison. Being the 3D extension, it is therefore referred to as the Interpolating Leaf Octree algorithm (ILOT). \\

In this experiment, Rate-Distortion is measured in PSNR (Peak Signal to Noise Ratio) which is a measurement of the amount of accuracy between the original model and the compressed version for a given algorithm and level of compression. The larger the PSNR, the higher the quality in which the compression system produces for a given bitrate. \\

A Rate-Distortion graph comparing the Plane-Tree, Octree and ILOT is presented in Figure \ref{fig:3DReconCompression1}. Results show that for low bit-rates the ILOT outperforms the Octree method. However as higher quality is required, the Octree in turn outperforms the ILOT method. The Plane-Tree algorithm proposed in this work is shown to be dominant compared to both algorithms. For each algorithm, at any bitrate, the Plane-Tree has a much larger Peak-Signal-To-Noise ratio. \\

\begin{figure}[!htb]
\centering
\includegraphics[width=4.0in]{images/results/compression/psnr1}
\caption{PSNR vs Bitrate comparing the ILOT, OT and PT compression methods.}
\label{fig:3DReconCompression1}
\end{figure}

In conclusion, as in results comparing the Plane-Tree with state-of-the-art mesh compression algorithms, the Plane-Tree is shown to perform well and even outperform ILOT and the generic Octree data structures when compressing 3D data used in reconstructions. 




