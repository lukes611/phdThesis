\begin{savequote}[8cm]
  ``I just wondered how things were put together.''
  \qauthor{Claude Shannon}
\end{savequote}
\makeatletter
\chapter{Experiments}
\label{ch:Experiments}

\section{Experiments}
Several experiments were designed in order to evaluate both the FVR method as well as the Plane-Tree and 3D Shade-Tree method. The types of these experiments as well as their conditions and results are presented in this section. In the next section (section \ref{TestDataSection}) the test data used in experimentation is introduced. Following that (in section \ref{ToolsSection}), the tools used to run these experiments are also introduced along with various error metrics (section \ref{metricsSection}) which provide means for comparisons between algorithms. \\

Then different results for experiments analysing the performance of FVR and comparing it to methods used in the current literature are presented. These experiments compare the FVR's ability to track camera movements (sections \ref{Sec:CamTransTrackExp} and \ref{Sec:CamRoteTrackExp}), its qualitative performance (section \ref{Sec:FVRQual1Exp}) and its robustness to noise (section \ref{Sec:FVRMotionExp}). Experiments evaluating the performance of Monocular FVR are then presented in section \ref{Sec:MonocularExperimentsSection} followed by experiments comparing FVR-3D to commonly used procedures from the literature. After these experiments, the Plane-Tree is compared to the Octree. These experiments compare both method's efficacy in compressing 3D mesh data. The Plane-Tree is also compared to various state-of-the-art compression techniques from the literature. Next, qualitative compression results are presented comparing the Plane-Tree to several other methods. Finally, the Plane-Tree is compared to both the Octree and the 3D-ShadeTree in compressing 3D reconstructions. \\



%\section{Test Data}
%\label{TestDataSection}
%


In order to assess results using an active camera sensor, test data was generated using the ASUS Xtion PRO LIVE camera. Because these data sets were generated during experimentation, the opportunity was taken to generate scenes, each captured using a specific type of camera movement. Some scenes were captured by rotating the camera about the x or y-axes, others by translating the camera. By testing with different movements, future algorithms may be constructed by switching to different registration methods based on camera movement. The different camera transformations recorded in the test data include: translation (left and right), and rotation about different axes (x and y axis). \\

Some samples from these data sets are shown in appendix \ref{ActiveSensorDataSet}. The first scene: the Apartment Texture Rotate scene was taken by rotating the camera around the y-axis across an apartment. This scene contains a lot of texture information. The Apartment Texture X Axis scene is similar in terms of texture but contains both x and y axis rotation. This tests the FVR's ability to handle multiple axes of rotation. The Office textured blind-spot rotation scene is a textured office scene where the camera is rotated about the y-axis. The scene is focused on a large divider which separates two desks. The divider may confuse registration methods which rely too heavily on minimization by aligning the large divider as a priority rather than taking into account the smaller details within the scene. An example of such an algorithm would be ICP and its derivatives. The Office scenes contain a plenty of usable texture and different sets were created by translating, rotating about the y-axis and rotating about the x-axis. \\

%mvvr
The MVVR method was analysed using test data generated using a Microsoft passive RGB camera or the rgb component of the ASUS Xtion PRO LIVE active camera. The data output is in basic video format, and all depth information was generated implicitly by the MVVR algorithm. This data was captured by moving the camera whilst focusing on a set of textured boxes within an indoor environment. Several frames were captured at 30 frames per second and registered for analysis of the MVVR method qualitatively. Quantitative tests of the MVVR make use of the same data as tests for the FVR method (testing translation and rotational registration error). \\

To assess qualitative reconstruction, three scenes were captured and registered using the FVR method. These scenes include the apartment scene, office scene and the garden scene. All three scenes were captured using the Asus XTION PRO live active camera at 30 frames per second. For each scene, only one in every 30 frames were registered, constituting real time FVR performance. The apartment scene was generated by moving and rotating the camera around an apartment living room before moving towards the kitchen area. This scene contains an abundance of features and would be considered a basic test for most 3D reconstruction algorithms. The second scene is the Office scene. This scene was generated by rotating the camera whilst zooming in and out on different objects within the room. Again this scene was reconstructed by registering every 30th frame. Despite both of the scenes being trivial to reconstruct, most algorithms (especially feature matching based methods) would find registering large rotations (such as those present in this data) difficult. The Garden scene is a difficult scene to reconstruct regardless of the reconstruction algorithm or technique used for registration. This scene was captured by rotating the camera around a garden outside the university. The scene contains many textures which are similar at the local level but are located in totally different locations. Therefore the scene is difficult to reconstruct using local and feature based methods such as FM+RANSAC and ICP. \\


\begin{figure}[!htb]
\centering
\includegraphics[width=4.0in]{images/experiments/test_data/modelsused}
\caption{Models used to assess the Plane-Tree compression algorithm.}
\label{fig:MODELSUSEDA}
\end{figure}

To test the Plane-Tree algorithm, several 3D objects commonly used within the literature were used to compare with known state of the art compression methods in terms of 3D mesh compression. The models used for testing are shown in figure \ref{fig:MODELSUSEDA}. These include: the bunny model with 34835 vertices, the rabbit model with 67039 vertices, the fandisk model with 6475 vertices and the horse model with 19851 vertices. \\

Several 3D reconstructions generated from the data set in appendix \ref{AppendixA} were used also to assess the Plane-Tree in compressing 3D reconstructions for storage and transmission. \\



\section{Tools}
\label{ToolsSection}

In the experiments, various machines, devices and software were used to generate and perform tests. This equipment is discussed here. All experiments were performed on two machines. One is an Asus laptop running both Windows 10 and Ubuntu 16. This laptop has an Intel i7 CPU and an NVIDIA GeForce 840 M GPU with 4 GB of RAM. The other is a Dell desktop computer running Windows 7. This machine has an Intel i5 CPU and 4 GB of RAM. \\ 

Both the Visual Studio C/C++ compiler on Windows and the GCC C/C++ compiler on Ubuntu were used to write programs for testing purposes. C++ version 17 was primarily used to write programs. Libraries used include: the OPEN-CV 3 computer vision library used for capturing, writing and processing image data, and CUDA used for writing General Purpose GPU (GPGPU) programs. Both Visual Studio (12 and 15) and Code Blocks 16 were used to write programs using the libraries and compilers. The METRO 3D object comparison tool was used for comparing 3D objects in Plane-Tree experiments. \\

To capture depth (RGB-D) video frames, the Asus Xtion Pro Live active camera was used to capture $640\times480$ video frames at 30 frames per second. This camera can capture depth between 0.8 and 3.5 m and was used to capture most of the test-data used in FVR experiments. \\ 

All source code and experiments are available online. The link for the FVR method is \url{https://github.com/lukes611/phdThesis} \href{https://github.com/lukes611/phdThesis} {FVR} and the link for the Plane-Tree source code is \url{https://github.com/lukes611/PlaneTree} \href{https://github.com/lukes611/PlaneTree}{Plane-Tree}. Further discussion about the metrics used for testing these algorithms is given in section \ref{metricsSection}. \\  


\section{Error metrics}
\label{metricsSection}

Several metrics are used to assess the set of proposed 3D reconstruction algorithms as well as the proposed Plane-Tree and other compression methods. These metrics are presented and defined mathematically here. Assessing both the 3D reconstruction algorithms as well as the lossy 3D data compression techniques requires comparing output 3D models. In the case of 3D reconstructions, we can compare the error between the registered frame and the ground truth, the larger the error, the worse the registration. Alternatively, a measurement may be used to measure the difference between two frames before and after they are registered. If the error is reduced or almost zero after registration, then the registration method may be deemed correct. If the error is larger or almost the same post registration, the registration maybe considered incorrect or there possibly was little camera movement. \\


In both cases, a robust way to measure the similarity/difference between two 3D objects must be used. To compute this error, we use nearest neighbour functions to measure the closest point in one model to the closest point in the next model. For example, given model A and model B, the closest point $B_j$ in $B$ for point $A_i$ in $A$, is used in summation of the total error. The closest point $B_j$ to $A_i$ is simply the nearest neighbour of $A_i$. This can be formally described as a function of the 3D point list $V$ and a given point $p$. This is described mathematically in equation \ref{eqn:NN}. Here, the result is a point $q$ in $V$, in which the distance between $q$ and $p$, according to function $Dist(x,y)$, is shorter for $q$ given $p$ than any other point $k$ in $V$.  \\

\begin{equation} \label{eqn:NN}
NN(p, V) =  \{ q \in V | (Dist(q, p) < Dist(k, p))  \forall k \in V \}
\end{equation}

For all purposes within this research, the function $Dist(x,y)$ is simply the Euclidean distance between the two input points. Using this definition of a nearest neighbour, several metrics may be used to compute the distance between two models. The one-way distance between two models is defined as the summation of distances for each point in one model to its nearest neighbour from the other model. The one-way mean error between two 3D models, $P$ and $Q$ is given in equation \ref{eqn:HDOW}. The full mean error between two objects is then computed as the average of the $Mean-Error_{one-way}$ function from model $P$ to $Q$ and model $Q$ to $P$. This is defined in equation \ref{eqn:MEANERRORMETRIC1}. \\


\begin{equation} \label{eqn:HDOW}
ME_{1-way}(P, Q) = \frac{1}{N}\sum_{k=0}^{N} Dist(P_k, NN(P_k, Q))
\end{equation}


\begin{equation} \label{eqn:MEANERRORMETRIC1}
ME(P, Q) = \frac{ME_{1-way}(P,Q) + ME_{1-way}(Q,P)}{2}
\end{equation}

Other metrics may also be used, for example the mean squared error can be used instead of mean error. The mean squared error (MSE) may then replace the mean error used in equation \ref{eqn:HDOW}. The one-way and full error functions based on the MSE are provided in equations \ref{eqn:MSEOW} and \ref{eqn:MEANSQERRORMETRIC1}. \\


\begin{equation} \label{eqn:MSEOW}
MSE_{1-way}(P, Q) = \frac{1}{N}\sum_{k=0}^{N} Dist(P_k, NN(P_k, Q))^2
\end{equation}


\begin{equation} \label{eqn:MEANSQERRORMETRIC1}
MSE(P,Q) = \frac{MSE_{1-way}(P,Q) + MSE_{1-way}(Q,P)}{2}
\end{equation}


The mean error based on the nearest neighbour is often referred to as the Hausdorff error. In this research, we used the Hausdorff error and the Mean Squared Error (MSE) based on the nearest neighbour technique, as well as a percentage of total matches. The one-way percentage of total matches is the computation of the percent of points from one model which have a nearest neighbour with a distance below a given threshold. This metric is defined in equation \ref{eqn:PERCMATCHOW}. \\

\begin{equation} \label{eqn:PERCMATCHOW}
PM_{1-way}(P, Q) = \frac{100}{N}\sum_{k=0}^{N} x, where
  \begin{cases}
    x=1       & \quad \text{if } Dist(P_k,NN(P_k,Q)) < \text{threshold}\\
    x=0  & \quad otherwise\\
  \end{cases}
\end{equation}

Following this, the full Percent-Match function may be defined as the average of the $PM_{1-way}$ function in both directions. These three metrics are used along with several others in 3D reconstruction experiments. Additionally, the camera tracking error is computed as the euclidean distance between the ground truth camera movement from one frame to another and the estimated camera movement computed via the FVR method. The voxel error is also measured. Since the camera distance metric measures the real positions moved by the camera, it theoretically has the accuracy of the real number system. Conversely, FVR uses voxel spaces (3D volumes) to find the translation and rotation (location and pose) between frames. Therefore, the result must be quantized. By quantizing the ground truth prior to comparison, the accuracy up to the resolution of the voxel grid may be computed. This is important since it measures the error of the FVR 3D reconstruction method without penalizing it based on the effects of quantization which reduce the larger the volume. \\

In experiments comparing the Plane-Tree with state-of-the-art algorithms, the root mean squared error is also used. The root mean squared error is simply the square root of the mean squared error function defined in equation \ref{eqn:MEANSQERRORMETRIC1}, that is $RMS(P,Q) = \sqrt{MSE(P,Q)}$. The Plane-Tree is also assessed in terms of 3D reconstruction occupancy grid compression. To assess the difference between the original 3D occupancy grid and the lossy compressed version, the Peak Signal to Noise Ratio metric (PSNR) is used. This is a function of the voxel-wise mean squared error function. Given the entire 3D volume, the mean squared error is computed between the original 3D reconstruction and the compressed 3D reconstruction. The mean squared error over two 3D volumes is defined in equation \ref{eqn:MEANSQERRORMETRIC2} as the summation of the voxel-wise squared error divided by the number of voxels.

\begin{equation} \label{eqn:MEANSQERRORMETRIC2}
MSE_{volume}(V_1,V_2) = \frac{1}{N^3}\sum_{z=0}^{N}\sum_{y=0}^{N}\sum_{x=0}^{N} \left(V_1(x,y,z) - V_2(x,y,z)\right)^2
\end{equation}

The PSNR metric is then defined as in equation \ref{eqn:PSNR1}. This metric is commonly used in image compression comparisons and since 3D images are being compared in this test, it is used here also. The PSNR metric is inversely related to error, so the larger the PSNR the lower the error between the two models.

\begin{equation} \label{eqn:PSNR1}
PSNR(V_1,V_2) = 10 \times \log10{\frac{255^2}{MSE_{volume}(V_1, V_2)}}
\end{equation}


%sota results
\section{Experiments using different sensor types}
\label{Sec:FVRSOTA}
\subsection{Algorithms} 
\label{AlgorithmsSection}
Different 3D-registration algorithms were implemented to test the Fourier Volume Registration (FVR) method. Feature matching methods are important to compare with because they are still dominant and very successful in image processing and computer vision. In this research we show that FVR is competitive with feature matching methods whilst beating them in certain contexts (such as little textured scenes or scenes where texture confusion may occur). \\ 

We test with both 2D feature matching and 3D feature matching. In 2D feature matching, the features are found and matched between a pair of 2D-images, then RANSAC is used with the corresponding matches and true 3D point to compute pose. The pose is then used to reconstruct the scene. We found that SURF performed best out of the other feature matching methods, so SURF was used in experiments. The 2D feature matching method is limited as it cannot register frames which have too few features or frames which contain texture confusion. It is also not able to handle wide base-lines. \\

We also test 3D-feature matching using an implementation of SIFT in 3D. This algorithm was tested and written in C/C++ and like the 2D counterpart, is also susceptible to failed registration in scenes with too few features and texture confusion but it is able to handle wide base-lines since it works in 3D. //

Another algorithm used in experiments is Iterative Closest Point or ICP. This method has become very popular in 3D reconstruction and works well on most scene types. One disadvantage is that this method may get stuck in a local minima and fail to register correctly. This typically occurs when registering against wide-baselines. \\

Another algorithm present in the experiments is Principal Components Analysis (PCA). This algorithm is used to find the mean and principal components of a multi-dimensional data set. This is useful for registration purposes as it works on wide-baselines, is very fast and provides additional information about a scene. The downside is that it is very susceptible to noise and misaligned data. The proposed FVR method makes use of information from PCA so it is important to compare the two to find out what improvements are made by FVR over PCA. \\

The final algorithm tested is the proposed FVR algorithm, which uses both PCA and Fourier Phase Correlation to find the registration transformation between two 3D data-sets as described in \ref{FullRecovery3DSection}. This algorithm was proposed to handle general transformations in terms of rotation, scaling and translation. It was also designed to be able to handle noisy data, data with texture-confusion and data with little or no texture. This makes it a viable option in the 3D registration and pose estimation research areas.

\subsection{Sensor Types} 
\label{SensorTypesExpsSection}

Experiements show that the FVR based techniques are capable of registration across the three primary types of sensors used in 3D reconstruction and SLAM research. These sensors include: Stereo sensors, which are capable of generating dense depth data with the 

%stereo tests
\subsection{Stereo Camera}
\label{StereoSOTA}
	
In these experiments, several current techniques (2D Feature Matching (FM-2D), 3D Feature Matching (FM-3D), ICP and PCA) are compared to the FVR technique in terms of stereo camera registration accuracy. To this end, five data sets from the Kitti Vision Benchmark data set \cite{Geiger13Vision} were used. Each data set scene is a complicated outdoor environment filmed using an autonomous driving platform. The majority of frames contain moving objects which interfere with the registration process of several algorithms. \\

Each Kitti Vision Benchmark data set contains accurate laser scans, stereo grey-scale images, stereo RGB-images and GPS and IMU data. To simulate the theoretical situation in which stereo cameras generate the most accurate depth maps, the laser scans are used in place of depth data computed by a stereo disparity algorithm. Therefore, in these experiments, only the laser scans and stereo colour and greyscale images are used. \\

The five stereo data sets used in the experiments are shown in Figures \ref{fig:KT1DSS}, \ref{fig:KT2DSS}, \ref{fig:KT5DSS}, \ref{fig:KT91DSS} and \ref{fig:KT95DSS}. The first scene is the Kitti 0001 Sync Data Set, this scene has 107 frames. This data set contains 12 cars, two cyclists and a tram. The tram and cyclists are non-static objects within the scene, making it more difficult for the current set of reconstruction algorithms which rely on scenes being primarily static in nature. To the right in this data set, there is a tram-line and some trees and gardens, to the left is a residential street. The second scene is the Kitti 0002 Sync data set, this set is 76 frames long. It contains one car and two cyclists which are non-static objects. Within this scene there is a long brick wall hiding some tall trees, several cyclists ride along next to the wall. To the left, there are some grassy areas, some buildings and some parked cars. \\

At 153 frames, the Kitti 0005 Sync data set is also used. This scene contains nine cars, three vans, two pedestrians and one cyclist. Because of the size of the van and the other non-static objects within the scene, this appears to be one of the more difficult scenes to register. The camera winds through several different streets as opposed to the previous data sets which are mostly of one long road. The fourth data set used, the Kitti 0091 Sync data set is composed of 339 frames making it the largest data set used in these experiments. This data set contains two cars, a van, 42 pedestrians, 14 sitters, eight cyclists and one miscellaneous object. As it contains many non-static objects this scene is also considered difficult like the Kitti 0005 data set. \\

The last Kitti Vision Benchmark data set used is the Kitti 0095 data set. Unlike the previous data sets, all 267 frames only contain static objects, and the scenes are made up primarily of small winding streets. \\ 

Experiments are tabled in full at per frame intervals in Appendix \ref{StereoResultsRaw}. Here, registration errors are presented where frame $n$ is registered against frame $n+1$ and the registration error is reported for each algorithm. The error function used is the Mean Squared Error $MSE(P,Q)$. This error is computed between the consecutive frames after registration. This error value is computed as in equation \ref{eqn:msesota}. Here, the function $Register(x)$ is replaced by the registration method being tested. \\

\begin{equation} \label{eqn:msesota}
Error(frame_1, frame_2) =  \frac{Register(frame_1), frame2}{MSE(frame_1,frame_2)}
\end{equation}

\begin{figure*}[t]
\centering
\begin{subfigure}[b]{1.5in}
\includegraphics[width=1.5in]{{images/experiments/stereo/1.1}.png}
\caption{Frame 1}
\end{subfigure}%
\begin{subfigure}[b]{1.5in}
\includegraphics[width=1.5in]{{images/experiments/stereo/1.2}.png}
\caption{Frame 39}
\end{subfigure}%
\begin{subfigure}[b]{1.5in}
\includegraphics[width=1.5in]{{images/experiments/stereo/1.3}.png}
\caption{Frame 77}
\end{subfigure}%
\begin{subfigure}[b]{1.5in}
\includegraphics[width=1.5in]{{images/experiments/stereo/1.4}.png}
\caption{Frame 114}
\end{subfigure}%
\caption{Kitti 0001 Sync Data Set Sample}
\label{fig:KT1DSS}
\end{figure*}



%% sync 0001

\begin{figure}
\centering
\begin{tabular}{ccc}
\hline
\textbf{Algorithm} & \textbf{Median Error $\times$ 1000} & \textbf{\% best results}\\ \hline
FM2D	& 5.28 & 13.21\%\\
FM3D	& 9235.71 & 0.94\%\\
ICP	& 5.15 & 27.36\%\\
PCA	& 5.66 & 2.83\%\\
FVR	& 5.5 & 13.21\%\\
FFVR	& 5.59 & 7.55\%\\
FVR-3D	& 5.1 & 34.91\%\\
\end{tabular}
\caption{Statistics for the Kitti Data 0001 Sync Data Set}
\label{tab:kittidata0001sync}
\end{figure} 

The summary of these results is also tabled here for convenience. For each algorithm, the median registration error is provided. This is computed by listing and sorting the registration error values for a particular algorithm and selecting the value in the middle. Also included is the percent of best results metric. This measures in percentage, the frequency of times a particular algorithm achieved the best (lowest error) registration result compared to the other algorithms tested. An algorithm with a lower median error than another algorithm has performed better overall. Additionally, if an algorithm has a higher percentage of best results, it outperformed the other algorithms most of the time. If an algorithm achieved an average percentage of best results but a higher median error, this could be explained by outliers. If an algorithm achieved a lower median error but did not achieve the highest percentage of best results, it may be due to having a very competitive and consistent registration error. \\

Table \ref{tab:kittidata0001sync} presents results for the Kitti 0001 Sync data set, some example colour frames are shown in Figure \ref{fig:KT1DSS}. The road in which this scene was filmed contains a tram-line and moving tram as well as a garden area to the right and a line of parked cars and houses under cover of shadows to the left. Registration statistics were taken over the full length of this data set, which is 107 frames. Results show that FVR-3D achieved the lowest median registration error, ICP achieved the next lowest followed by FM2D and the FVR method. FVR-3D also achieved the highest percentage of best frame registration results at ~34.91 \% compared to ICP with ~27.36 \%. If the FVR methods were combined as a single hybrid method, they would have computed the best registration result 55.67 \% of the time. The FVR method alone outperformed FFVR, PCA and FM-3D. The FM-3D method (which had several frame registration failures, as is evident in its statistics) and the PCA method were the worst performers on this data set. The FFVR, whilst faster than the FVR method is slightly worse off in terms of performance. \\ 


%% sync 0002
\begin{figure}
\centering
\begin{tabular}{ccc}
\hline
\textbf{Algorithm} & \textbf{Median Error $\times$ 1000} & \textbf{\% best results}\\ \hline
FM2D	& 4.78 & 5.33\%\\
FM3D	& 4.85 & 6.67\%\\
ICP	& 4.43 & 30.67\%\\
PCA	& 4.86 & 6.67\%\\
FVR	& 4.67 & 6.67\%\\
FFVR	& 5.23 & 5.33\%\\
FVR-3D	& 4.27 & 38.67\%\\
\end{tabular}
\caption{Statistics for the Kitti Data 0002 Sync Data Set}
\label{tab:kittidata0002sync}
\end{figure} 



\begin{figure*}[t]
\centering
\begin{subfigure}[b]{1.5in}
\includegraphics[width=1.5in]{{images/experiments/stereo/2.1}.png}
\caption{Frame 1}
\end{subfigure}%
\begin{subfigure}[b]{1.5in}
\includegraphics[width=1.5in]{{images/experiments/stereo/2.2}.png}
\caption{Frame 28}
\end{subfigure}%
\begin{subfigure}[b]{1.5in}
\includegraphics[width=1.5in]{{images/experiments/stereo/2.3}.png}
\caption{Frame 56}
\end{subfigure}%
\begin{subfigure}[b]{1.5in}
\includegraphics[width=1.5in]{{images/experiments/stereo/2.4}.png}
\caption{Frame 83}
\end{subfigure}%
\caption{Kitti 0002 Sync Data Set Sample}
\label{fig:KT2DSS}
\end{figure*}


Table \ref{tab:kittidata0002sync} presents median error and percent best result statistics for the Kitti 0002 Sync Data Set, example colour frames are shown in Figure \ref{fig:KT2DSS}. The road in which this scene contains some parked cars to the left as well as a building, some grass and some large trees. To the right there is an orange brick wall and some trees and gardens as well as two moving cyclists. Results for the full 74 frames of the data set show that the FVR-3D method achieved the lowest median registration error at 4.27. The ICP algorithm achieved the next best result at 4.43 and the non-rotation-invariant FVR method achieved the third best result. Combined, the FVR based methods achieved the best result ~50.67 \% of the time compared to ICP at 30.67 \%. In this scene, the FM-3D method did not have as many registration failures and achieved a better result than the FFVR method and a comparable result to the PCA method. \\ 



%% 0005
\begin{figure}
\centering
\begin{tabular}{ccc}
\hline
\textbf{Algorithm} & \textbf{Median Error $\times$ 1000} & \textbf{\% best results}\\ \hline
FM2D	& 3.39 & 39.22\%\\
FM3D	& 3.83 & 0\%\\
ICP	& 3.49 & 25.49\%\\
PCA	& 4.06 & 0\%\\
FVR	& 3.7 & 5.88\%\\
FFVR	& 4.25 & 1.96\%\\
FVR-3D	& 3.42 & 27.45\%\\
\end{tabular}
\caption{Statistics for the Kitti Data 0005 Sync Data Set}
\label{tab:kittidata0005sync}
\end{figure} 

Statistics for the Kitti 0005 Sync Data Set are presented in Table \ref{tab:kittidata0005sync}, example colour frames are shown in Figure \ref{fig:KT5DSS}. The scene captured in this data set was more difficult than previous scenes as it contains two moving cyclists and one large van which are all moving around in the scene without any relation to camera movement. In other words, these non-static objects cause major difficulties in most registration algorithms. In the full 152 frames of the data set, FM2D performed best with the lowest median error and highest percentage of best results. Next, FVR-3D also performed well with the second best median error and percentage of best results measurement. ICP achieved the third best median error and percentage of best results. In the results for this data set, FVR outperformed the FFVR method, as well as PCA and FM3D. Combined, the FVR algorithms achieved the best registration result 35.29 \% of the time, which is still below the performance of FM2D on this data set. \\

\begin{figure*}[t]
\centering
\begin{subfigure}[b]{1.5in}
\includegraphics[width=1.5in]{{images/experiments/stereo/5.1}.png}
\caption{Frame 1}
\end{subfigure}%
\begin{subfigure}[b]{1.5in}
\includegraphics[width=1.5in]{{images/experiments/stereo/5.2}.png}
\caption{Frame 54}
\end{subfigure}%
\begin{subfigure}[b]{1.5in}
\includegraphics[width=1.5in]{{images/experiments/stereo/5.3}.png}
\caption{Frame 107}
\end{subfigure}%
\begin{subfigure}[b]{1.5in}
\includegraphics[width=1.5in]{{images/experiments/stereo/5.4}.png}
\caption{Frame 160}
\end{subfigure}%
\caption{Kitti 0005 Sync Data Set Sample}
\label{fig:KT5DSS}
\end{figure*}


%% kitti dataset 0091 Sync

\begin{figure}
\centering
\begin{tabular}{ccc}
\hline
\textbf{Algorithm} & \textbf{Median Error $\times$ 1000} & \textbf{\% best results}\\ \hline
FM2D	& 3.6 & 17.11\%\\
FM3D	& 4.04 & 0.88\%\\
ICP	& 3.61 & 15.63\%\\
PCA	& 4.1 & 1.18\%\\
FVR	& 3.57 & 20.94\%\\
FFVR	& 3.78 & 5.9\%\\
FVR-3D	& 3.43 & 38.35\%\\
\end{tabular}
\caption{Statistics for the Kitti Data 0091 Sync Data Set}
\label{tab:kittidata0091sync}
\end{figure} 

Table \ref{tab:kittidata0091sync} presents results for the Kitti 0091 Data Set, example colour frames from the data set are shown in Figure \ref{fig:KT91DSS}. This is the largest data set tested at 339 frames. The scene filmed in this data set is that of an outdoors inner city. It contains many moving agents making it a scene which is difficult to register for most registrations algorithms. Specifically, it contains two cars, a van, 42 pedestrians and eight cyclists. Registration results show FVR-3D outperformed all the other algorithms in terms of both median error and percentage of best results. The FVR algorithm achieved the next best results. The FVR-3D algorithm achieves the best result more than twice as often as ICP and FM2D. The FVR based methods performed well on this data set, a hybrid approach would have achieved the best results 65.19 \% of the time.  \\  	


\begin{figure*}[t]
\centering
\begin{subfigure}[b]{1.5in}
\includegraphics[width=1.5in]{{images/experiments/stereo/91.1}.png}
\caption{Frame 1}
\end{subfigure}%
\begin{subfigure}[b]{1.5in}
\includegraphics[width=1.5in]{{images/experiments/stereo/91.2}.png}
\caption{Frame 116}
\end{subfigure}%
\begin{subfigure}[b]{1.5in}
\includegraphics[width=1.5in]{{images/experiments/stereo/91.3}.png}
\caption{Frame 231}
\end{subfigure}%
\begin{subfigure}[b]{1.5in}
\includegraphics[width=1.5in]{{images/experiments/stereo/91.4}.png}
\caption{Frame 346}
\end{subfigure}%
\caption{Kitti 0091 Sync Data Set Sample}
\label{fig:KT91DSS}
\end{figure*}


%% sync 0095

\begin{figure}
\centering
\begin{tabular}{ccc}
\hline
\textbf{Algorithm} & \textbf{Median Error $\times$ 1000} & \textbf{\% best results}\\ \hline
FM2D	& 4.19 & 12.36\%\\
FM3D	& 5.18 & 0\%\\
ICP	& 4.4 & 13.48\%\\
PCA	& 5.32 & 0.37\%\\
FVR	& 4.12 & 22.85\%\\
FFVR	& 4.73 & 4.12\%\\
FVR-3D	& 3.96 & 46.82\%\\
\end{tabular}
\caption{Statistics for the Kitti Data 0095 Sync Data Set}
\label{tab:kittidata0095sync}
\end{figure} 

Results for the Kitti 00095 Data Set are presented in Table \ref{tab:kittidata0095sync} and Figure \ref{fig:KT95DSS} presents four example colour images from this data set. This scene was much more static than the previous scenes. It contains primarily parked cars and buildings in an inner-city environment. There are a few pedestrians and cyclists which are moving agents within the scene. This data set is 266 frames and results show that FVR-3D achieves both the best (lowest) median error score and the highest percentage of best results score at ~46.82 \%. The FVR algorithm achieved the second best results in terms of both the median error and percentage of best results metrics. ICP and FM2D were third and fourth, respectively. The FVR-3D algorithm achieved the best result three times more often than both the ICP and FM2D methods. \\



\begin{figure*}[t]
\centering
\begin{subfigure}[b]{1.5in}
\includegraphics[width=1.5in]{{images/experiments/stereo/95.1}.png}
\caption{Frame 1}
\end{subfigure}%
\begin{subfigure}[b]{1.5in}
\includegraphics[width=1.5in]{{images/experiments/stereo/95.2}.png}
\caption{Frame 92}
\end{subfigure}%
\begin{subfigure}[b]{1.5in}
\includegraphics[width=1.5in]{{images/experiments/stereo/95.3}.png}
\caption{Frame 183}
\end{subfigure}%
\begin{subfigure}[b]{1.5in}
\includegraphics[width=1.5in]{{images/experiments/stereo/95.4}.png}
\caption{Frame 274}
\end{subfigure}%
\caption{Kitti 0095 Sync Data Set Sample}
\label{fig:KT95DSS}
\end{figure*}


%active camera-tests
\subsection{Active Camera}
\label{ActiveSOTA}

Experiments evaluating the performance of the set of FVR related algorithms on active sensor camera input are presented here. These data-sets are all 25 frames long and captured of different environments using different camera movements. Due to the limitations of the ASUS Xtion PRO LIVE active camera used, indoor scenes were captured primarily. Similarly to the results presented in section \ref{StereoSOTA} statistics are presented for algorithms from the literature (FM2D, FM3D, ICP \& PCA) as well as FVR based algorithms (FVR, FFVR \& FVR-3D) in the form of the median registration error and the percentage of best results. 

%%Apartment Texture Rotate:
\begin{figure}
\centering
\begin{tabular}{ccc}
\hline
\textbf{Algorithm} & \textbf{Median Error $\times$ 1000} & \textbf{\% best results}\\ \hline
FM2D	& 2.13 & 36\%\\
FM3D	& 5.14 & 0\%\\
ICP	& 2.42 & 16\%\\
PCA	& 8.61 & 4\%\\
FVR	& 2.87 & 8\%\\
FFVR	& 2.7 & 8\%\\
FVR3D	& 2.05 & 28\%\\
\end{tabular}
\caption{Statistics for the Apartment Texture Rotate Data Set}
\label{tab:apartmenttexturerotate}
\end{figure} 



%%Apartment Texture X-Axis Rotation
\begin{figure}
\centering
\begin{tabular}{ccc}
\hline
\textbf{Algorithm} & \textbf{Median Error $\times$ 1000} & \textbf{\% best results}\\ \hline
FM2D	& 1.97 & 4\%\\
FM3D	& 2.58 & 0\%\\
ICP	& 1.78 & 36\%\\
PCA	& 3.81 & 0\%\\
FVR	& 1.99 & 4\%\\
FFVR	& 2.01 & 0\%\\
FVR3D	& 1.87 & 56\%\\
\end{tabular}
\caption{Statistics for the Apartment Texture X-Axis Rotation Data Set}
\label{tab:apartmenttexturex-axisrotation}
\end{figure} 


%% desk texture translation
\begin{figure}
\centering
\begin{tabular}{ccc}
\hline
\textbf{Algorithm} & \textbf{Median Error $\times$ 1000} & \textbf{\% best results}\\ \hline
FM2D	& 1.24 & 8\%\\
FM3D	& 2.48 & 12\%\\
ICP	& 1.59 & 28\%\\
PCA	& 1.51 & 4\%\\
FVR	& 1.16 & 16\%\\
FFVR	& 1.29 & 16\%\\
FVR3D	& 1.23 & 16\%\\
\end{tabular}
\caption{Statistics for the Desk Texture Translation Data Set}
\label{tab:desktexturetranslation}
\end{figure} 



%%Office Textured Blindspot Rotation
\begin{figure}
\centering
\begin{tabular}{ccc}
\hline
\textbf{Algorithm} & \textbf{Median Error $\times$ 1000} & \textbf{\% best results}\\ \hline
FM2D	& 1.39 & 24\%\\
FM3D	& 5.92 & 0\%\\
ICP	& 1.2 & 44\%\\
PCA	& 4.83 & 8\%\\
FVR	& 2.07 & 0\%\\
FFVR	& 2.92 & 0\%\\
FVR3D	& 1.1 & 24\%\\
\end{tabular}
\caption{Statistics for the Office Textured Blindspot Rotation Data Set}
\label{tab:officetexturedblindspotrotation}
\end{figure} 

%%Office Textured Items Translation
\begin{figure}
\centering
\begin{tabular}{ccc}
\hline
\textbf{Algorithm} & \textbf{Median Error $\times$ 1000} & \textbf{\% best results}\\ \hline
FM2D	& 2.89 & 24\%\\
FM3D	& 5.45 & 0\%\\
ICP	& 2.93 & 4\%\\
PCA	& 3.79 & 0\%\\
FVR	& 5.04 & 0\%\\
FFVR	& 3.06 & 28\%\\
FVR3D	& 2.83 & 44\%\\
\end{tabular}
\caption{Statistics for the Office Textured Items Translation Data Set}
\label{tab:officetextureditemstranslation}
\end{figure} 


%%office texture rotation
\begin{figure}
\centering
\begin{tabular}{ccc}
\hline
\textbf{Algorithm} & \textbf{Median Error $\times$ 1000} & \textbf{\% best results}\\ \hline
FM2D	& 4.36 & 26.92\%\\
FM3D	& 7.15 & 0\%\\
ICP	& 4.76 & 34.62\%\\
PCA	& 6.55 & 0\%\\
FVR	& 5.3 & 7.69\%\\
FFVR	& 4.74 & 7.69\%\\
FVR3D	& 4.35 & 23.08\%\\
\end{tabular}
\caption{Statistics for the Office Texture Rotation Data Set}
\label{tab:officetexturerotation}
\end{figure} 



%monocular-tests
\subsection{Monocular Camera}
\label{Sec:MonocularSOTA}

The Fourier Volume Reconstruction (FVR) method described in sections \ref{Sec:VolumeRegistrationSection} and \ref{Sec:AFVRApproach} was designed to be independent of RGB-D information input and thus sensors. As long as greyscale data with a depth component, RGB data with a depth component or depth data on their own are input, the FVR method should be able to compute accurate relative dense 3D reconstructions and optionally relative camera pose data. \\

Below, in section \ref{subsec:DSBR} both the advantages and limitations of 3D reconstruction via active depth sensors are discussed. It was discovered that the closer computed depth is to true depth (either via a stereo camera pair, active camera or monocular camera) the better the camera pose registration using the FVR method would be. Section \ref{subsec:SCBR} covers stereo input. Stereo pair computed depth has the potential to be the most accurate in terms of depth accuracy and resolution. Since the resolution only depends on the camera pair used, it can be high in contrast to active cameras. Budget active cameras such as the Microsoft Kinect and ASUS Xtion PRO LVIE camera, typically have a lower resolution than budget HD video cameras. \\

In section \ref{subsec:MVVRMethodology} the FVR is applied to monocular data. This technique is named Monocular View Volume Reconstruction (MVVR). The MVVR is discussed in terms of process and limitations in this section. Experiments in section \ref{Sec:MonocularSOTA} show that the MVVR method is capable of camera translation tracking using noisy depth data generated using only monocular methods. However, its accuracy greatly depends on the quality of the 3D frames generated. \\

\subsubsection{Depth Sensor Based Reconstruction}
\label{subsec:DSBR}

Depth sensor input is advantageous for several reasons. It is faster to compute depth compared with both stereo and monocular methods, and in practice is more robustness and reliable. In most cases it is more accurate than monocular based methods. The major disadvantage in using such sensors has historically been one of accessibility. Thanks to budget active cameras such as the Microsoft Kinect and the Asus Xtion Pro Live camera being available to the general public at low cost, this is changing. However, there is still a mass of legacy video data which does not have actively generated depth data provided, moreover there is still a wealth of content still being produced by simple RGB cameras. \\

Another drawback of this sensor is that of accuracy. The accuracy of both the Kinect and Asus Xtion Pro Live sensors is limited to the resolution of the device. To alleviate this, some sub-pixel resolution generating algorithms may be used. These methods can be used in an attempt to generate greater resolution for given depth data. Methods may work on frames in a standalone fashion, although additional data such as consecutive frames and multi-camera techniques may also be used. Both the Microsoft Kinect and Asus Xtion Live Pro have maximum resolutions of $640 \times 480$. Methods of depth generation based on stereo data input are capable of generating dense depth information at resolutions only limited by the resolution of the cameras used. As of 2017, most cameras, even those found on mobile devices, can generate video data at resolutions of $1024 \times 1080$. \\

Another possible drawback in using RGB-D sensors is that certain materials reflect the infra-red light used by such active cameras to compute the depth information. These sensors are also known to produce noisy depth data around the corners and edges of objects captured within the frame. Since most computer vision algorithms make use of such salient features such as edges and corners within the image data, this can lead to inaccurate 3D reconstructions, especially in cases where 2D feature matching is used with RANSAC to produce 3D camera pose information. These sensors are also limited to indoor environments. If their infra-red sensor components are saturated with sunlight, little to no useful depth data may be generated. This is a major drawback as much of the world is made up of complex outdoor environments, therefore this is an unfortunate restriction given the usefulness of such cameras in indoor environments and lower light environments. \\

Another major disadvantage is that these sensors cannot produce depth data over distances. This is to do with the range of the active infra-red projection component of these sensors. Because the projection cannot reach far distances, the infra-red sensor cannot detect depth. This is in contrast to stereo and monocular methods which are capable of distant depth estimates.  

RGB-D camera sensors are still a popular choice for 3D reconstruction frameworks as the depth data produced by these methods is faster and more reliable than other sensors and techniques. For scenes and environments which can be accurately scanned by such depth sensors (such as environments with low sunlight, short ranges as in office environments and scenes with few objects which reflect infra-red light), RGB-D cameras are typically the preferred choice. The RGB-D data used to test the FVR method comes from an Asus Xtion Pro LIVE sensor. This sensor produces a colour and depth image input pair, which is processed in order to generate 3D volume frames which are the required input to the FVR method. \\

The color and depth image pair are referred to by $f(u,v)$ and $g(u,v)$ where $f(u,v)$ refers to the color image and $g(u,v)$ refers to the depth image. The colour image data contains a red, green and blue component each between the range $[0,255]$. Depth information is processed within the range $[0,10000]$. In terms of resolution, $u$ ranges $[0..639]$ and $v$ ranges $[0..479]$ giving a resolution of $640 \times 480$. Examples of images generated by the Asus Xtion Pro LIVE sensor are shown in figures \ref{fig:COLEXAMPLE} and \ref{fig:DEPTHEXAMPLE}. Given depth value for a given coordinate $u,v$, $Z_{u,v}$ is equal to the raw depth data value $g(u,v)$. Given coordinate $u,v$ and depth value $Z_{u,v}$ obtained via image $g$, $f(u,v)$ is projected into 3D space using equation \ref{eqn:PC_PROJECTION} to obtain the x-axis coordinate $X_{u,v}$ and y-axis coordinate $Y_{u,v}$. For each pixel value where depth information may be accurately recovered, the 3D point $[X_{u,v}, Y_{u,v}, Z_{u,v}]^T$ may be recovered. \\

From equation \ref{eqn:PC_PROJECTION}, the $X_{u,v}$ and $Y_{u,v}$ values are first translated relative to the center of the screen and then multiplied by the depth and attenuated by the focal length parameter. This is a typical projection function. In the equation, $c_x$ and $c_y$ represent the point where the optical axis intersects the projection plane (which occurs right at the center of the image, $c_x = 319.5$, $c_y = 239.5$. Parameters $f_x$ and $f_y$ represent the focal length which is defined as $f_x$, $f_y = 525.0$. \\


\begin{equation} \label{eqn:PC_PROJECTION}
\begin{split}
X_{u,v} & = \frac{(u - c_x)Z_{u,v}}{f_x} \\
Y_{u,v} & = \frac{(v - c_y)Z_{u,v}}{f_y} \\
\end{split}
\end{equation}

The FVR method requires that the point cloud generated by 3D points $[X_{u,v}, Y_{u,v}, Z_{u,v}]^T$ and colour data $f(u,v)$ be quantized for integration within a 3D volume. The volumes sizes used for testing were of sizes $128^3$, $256^3$ and $384^3$. Primarily, sizes $256^3$ and $384^3$ were used. It is common to use sizes which are an exponent of 2 for various computational reasons. However, $512^3$ was too large for the GPU used for testing (an NVIDIA GeForce 840 M GPU) so the maximum size $384^3$ was used for testing. The 3D point clouds were scaled to fit the volume, then their coordinates were rounded and integrated into the volume prior to registration. Examples of a colour and depth image pair as well as the corresponding projection into different volume sizes ($128^3$ and $256^3$) are shown in figure \ref{fig:PROJECTED_FRAME}.  \\

Once the data has been projected into a volume, it may be used as input to the technique described in section \ref{Sec:AFVRApproach}. The algorithm may then compute the registration matrix required to merge two frames, the matrix is accumulated and used to integrate each frame in succession. Upon registration and integration of each frame, the final output 3D reconstruction may be obtained. \\

\begin{figure}[!htb] 
        \centering
        \begin{subfigure}[b]{1.8in}
                \includegraphics[width=1.7in]{images/ch2/colorF11}
                \caption{Color Image}
                \label{fig:COLEXAMPLE}
        \end{subfigure}%
        \begin{subfigure}[b]{1.8in}
                \includegraphics[width=1.7in]{images/ch2/depthF11}
                \caption{Depth Image}
                \label{fig:DEPTHEXAMPLE}
        \end{subfigure}
        
         \begin{subfigure}[b]{1.8in}
                \includegraphics[width=1.8in]{images/ch2/volumeF11128}
                \caption{Projected Volume $128^3$}
                \label{fig:VOLUMEEXAMPLE128}
        \end{subfigure}%
         \begin{subfigure}[b]{1.8in}
                \includegraphics[width=1.8in]{images/ch2/volumeF11256}
                \caption{Projected Volume $256^3$}
                \label{fig:VOLUMEEXAMPLE384}
        \end{subfigure}%
       \caption{A Projected Frame.}
       \label{fig:PROJECTED_FRAME}
\end{figure}

\subsubsection{Stereo Camera Based Reconstruction}
\label{subsec:SCBR}
As mentioned, the FVR method may take dense depth data from any type of input sensor, as long as dense 3D point clouds can be generated per frame or near per frame rates 3D reconstructions may be computed. Stereo methods are capable of producing depth data at much higher resolitions and accuracies but are not as fast or as reliably compared to active sensors such as RGB-D cameras. However, they can produce depth maps under many environmental conditions where active cameras cannot. Outdoor scenes with a lot of sunlight may be captured with a stereo pair and used to compute dense depth data per frame. This data can then be integrated into the FVR algorithm. Additionally stereo sensors are capable of generating depth data over larger ranges as compared to active camera approaches. In this way, they can be used to scan objects in far off distances or produce depth data from overhead aerial data. \\

A major disadvantage is that in practical stereo algorithms, depth maps can be noisy. Computing 3D reconstructions from depth data which have been produced via stereo methods is also a challenging research project in itself. Still the noise is much lower in comparison to monocular methods of depth estimation. In generating depth data using a stereo set-up, the stereo camera pair must first be calibrated. The camera pair may be calibrated using software techniques to compute intrinsic camera parameters. Alternatively, the stereo data may be computed without first calibrating the stereo pair, resulting in less accurate depth data. Section \ref{StereoMethodsSection} described some of the available techniques in generating stereo data. \\

\subsubsection{Monocular View Volume Reconstruction}
\label{subsec:MVVRMethodology}

Generating dense depth information from monocular view is challenging. There are methods however, which take advantage of the properties of the Fundamental and Essential matrices to pre-calibrate consecutive frames. This allows stereo methods (see section \ref{StereoMethodsSection}) to be used to compute dense depth data per frame. These stereo methods are typically split into three categories: global, semi-global and local. Global methods are the most complex, and typically make use of some optimization method which optimizes the depth map's smoothness, disparity and contouring of edges jointly. Semi-global methods also optimize the depth map with respect to these requirements, however they are restricted to optimizing scan lines. This makes them much faster than the global optimization methods, whilst still providing better results than the local methods. Local methods are fast and ideal for real time applications, they make use of block matching to compute the disparity per pixel. \\

These methods all require frame by frame rectification. To do this, the fundamental matrix must be estimated accurately. The computation of the Fundamental matrix requires the use of feature matching and RANSAC which often are unable to accurately compute the Fundamental matrix. To generate dense depth data per frame, the use of optical flow based on 2D block matching was explored. By using 2D block matching, dense depth approximations were able to be computed whilst circumventing the use of the fundamental matrix and frame rectification. The depth maps computed are very noisy and inaccurate. The use of 2D spatial filtering as well as temporal filtering reduces such noise. \\

The spatial filtering technique used is basic Gaussian convolution. Temporal filtering was performed on a per pixel basis, where the output depth map is computed based on the previous depth map and the most recent depth map. Each depth map value is set to a value of $\delta \times prior-depth + (1-\delta) \times current-depth$. The best value for $\delta$ was judged based on qualitative analysis of experiment data sets. This method, although simpler and less accurate than other stereo methods, was used to test the FVR methods robustness to noisy and inaccurate 3D frame input. \\


For best performance, RGB-D information from alternative sources may be used in the FVR 3D reconstruction algorithm such as depth computed via active camera (Kinect, Asus Xtion LIVE PRO sensors) or disparity computed via stereo device. An alternative method is proposed here named Monocular View Volume Registration ($MVVR$). This algorithm takes consecutive RGB frames as input from a monocular camera sensor and computes depth data from them using the simple block matching method. The depth data is not aligned and because of camera speeds projection may not be consistent across frames. The depth maps are computed under the assumption of temporal camera translation. By performing block matching on frames taken from different camera locations, depth map be computed but only if the camera displacement is large enough. Therefore a $skip$ variable is used in order to block match frames with a distance from each other. \\

Using the computed depth maps, the algorithm projects consecutively computed depth maps into frames, which are then voxelized. Once the two frames are voxelized, the FVR technique proposed in sections \ref{FVRSectionA} and \ref{Sec:VolumeRegistrationSection}. A registration matrix may then be used to update camera parameters and to register and integrate the 3D frames. 

The algorithm is detailed in listings \ref{algorithm:MVVRAlgorithm}. A queue is used named $frameQueue$ to keep track of input RGB frames in order to generate depth data using frames which have a predefined temporal separation. A stack, $depthMapStack$ is used to keep track of the next two consecutive depth maps to register. $GlobalReconstruction$ is an automatic dynamically resizeable occupancy grid. \\

The matrix $M$ is used to keep track of the accumulation of transforms used to register and integrate each depth map into the global reconstruction output. Both $Camera_{location}$ and $Camera_{pose}$ keep track of the camera's location and pose information on a per frame basis. The list, $Cameras$ keeps track of the list of camera location and pose information for each frame.\\

\begin{figure}
\begin{lstlisting}[language=c++,caption=Monocular View Volume Reconstruction,label=algorithm:MVVRAlgorithm,mathescape,basicstyle=\ttfamily]
$frameQueue$ = newQueue();
$depthMapStack$ = newStack();
$GlobalReconstruction$.clear();
$M$ = IdentityMatrix();
$Camera_{location}$ = $[0, 0, 0]^T$;
$Camera_{pose}$ = $[[1, 0, 0]^T,[0, 1, 0]^T,[0, 0, 1]^T]$;
$Cameras$ = $\left[\left[Camera_{location}, Camera_{pose}\right] \right]$;
while(more frames){
	$f$ = ReadRGBFrame();
	$frameQueue$.push($f$);
	if($frameQueue$.length >= skip)
		$depthMapStack$.push(BlockMatch($f$, $frameQueue$.pop()))
	if($depthMapStack$.length >= 2){
		$f_2$ = $depthMapStack$.pop();
		$f_1$ = $depthMapStack$.pop();
		$points_1$ = project($f_2$);
		$points_2$ = project($f_1$);
		$V_1$ = Voxelize($points_1$);
		$V_2$ = Voxelize($points_2$);
		$(\theta, \varphi, t_x, t_y, t_z) = FVR_{\theta \varphi t_x t_y t_z}(V_1, V_2)$;
		$Temp = $TransformMat($(\theta, \varphi, t_x, t_y, t_z)$);
		$M = M \times Temp$;
		$points_1$ = Transform($points_1$, $M$);
		$GlobalReconstruction$.integrate($points_1$);
		$Camera_{location}$ = $Temp^{-1} \times Camera_{location}$;
		$Camera_{pose}$ = $Temp^{-1} \times (Camera_{pose} + Camera_{location})$;
		$Camera_{pose}$ = $\frac{Camera_{pose} - Camera_{location}}{Camera_{pose} - Camera_{location}}$;
		$Cameras.add(\left[Camera_{location}, Camera_{pose}\right])$;
		$depthMapStack$.push($f_2$);
	}
}
\end{lstlisting}
\end{figure}


Then a loop processes the frames whilst there are still frames available for input. For each new frame $f$, $f$ is added to the queue of frames, $frameQueue$. If there are more frames in the queue than variable $skip$ (typically set to 4 to ensure enough camera translation is present to obtain measurable depth) then block matching is performed between the current frame and the oldest item in the queue which is removed. The corresponding RGB-D image obtained from the $BlockMatch$ function is inserted into the $depthMapStack$ stack. \\

Also during the loop, if there are two or more RGB-D images within the stack, they are popped out and projected into the $points_1$ and $points_2$ 3D point clouds. Both point clouds are then voxelized into two volumes for input into the FVR method. A matrix representing the transform between the two volumes is then computed. \\

The accumulation matrix $M$ is then updated using this information and $points_1$ is aligned with $points_2$ using it. The transformed $points_1$ may then be integrated into the global reconstruction occupancy grid. Next the camera pose and location are updated. Finally, the most recently computed depth map, $f_2$ is inserted back into the stack for comparison with the next depth map computed. \\ 

Using the concept of parallax, we know that the magnitude of the displacement vector computed using block matching is an approximation to the actual depth for each corresponding location in both images. However, due to noise the resulting displacement vectors can be very noisy. Figure \ref{fig:DepthGenerationExample} shows how noisy this estimated depth information can be compared with input from an RGB-D device. As mentioned, the noise can be largely eliminated using spatiotemporal filtering. \\


\begin{figure}[!htb]
\centering
\includegraphics[width=6cm]{images/methodology/FVR/home_depth_frame_mono}
\includegraphics[width=6cm]{images/methodology/FVR/home_depth_frame}
\caption{Comparison between block matching estimated depth (left) and RGBD device depth (right).}
\label{fig:DepthGenerationExample}
\end{figure}
 
\subsubsection{Limitations of MVVR}

During experiments it was discovered that due to the high levels of noise, the MVVR method was limited to registration of translation information only. This was due to the difficulty in computing depth data using basic methods such as stereo methods between frames and optical flow. Despite this limitation, the MVVR method was able to register camera translation (see experiments in section \ref{Sec:MonocularSOTA}).  

%the basic tracking stuff
\section{Camera Tracking \& Noise Robustness}
\label{Sec:CamTransTrackExp}
\section{Camera Translation Tracking}
\label{Sec:CamTransTrackExp}

Experiments measuring the error after moving the camera using the FVR based methods as well as other methods from the literature are presented in this section. For these experiments, one camera frame of an indoor environment was captured using the ASUS Xtion PRO LIVE active camera. The camera was then moved (translated) by different amounts including: 5 centimeters, 10 centimeters and 15 centimeters. The 2D and 3D Feature matching / Ransac methods (FM2D, FM3D), ICP and PCA methods were all tested. Additionally, results for the FVR, FFVR and FVR3D method are presented. FM3D, FVR and FVR3D require the 3D frames be quantized into $256\times 256\times 256$ volumes. \\

Different levels of noise were added to both frames prior to 3D registration in order to measure each method's ability to register frames with large amounts of noise. The Signal to Noise Ratio (SNR) metric is used to describe the noise added to both captured frames, prior to any registration. This noise effects any registration method's ability to accurately estimate the transformation separating two sets of data. In a given experiment, a noise range value of $x$ means random noise was added in the range [$\frac{-x}{2}$, $\frac{x}{2}$] to a signal within the range [0, 1]. Reconstruction error is measured in mean squared error (as presented in section \ref{metricsSection}). \\

Table \ref{table:trans} shows the results of these tests, they illustrate the FVR's robustness to noise whilst registering frames which are captured during different camera translation intervals. Each sub-table is labelled with a distance in centimeters in which frames were separated prior to registration. The first two columns represent the amount of noise added both in terms of noise-range and the subsequent Signal to Noise ratio computed from it. The rest of the columns represent the registration error using the Mean Squared Error metric.  \\

%translation
\begin{table}[!htb]
\centering
\scalebox{1.0}{
\begin{tabular}{ccccccccc}
\\ \textbf{5cm} &   &   &   &   &   &   &   &   \\ 
Noise & SNR & \textbf{FM2D} & \textbf{FM3D} & \textbf{ICP} & \textbf{PCA} & \textbf{FVR} & \textbf{FFVR} & \textbf{FVR3D} \\ \hline
0 & $\infty$ & 5.07 & 4.75 & 12.35 & 3.73 & 4.04 & fail & 6.24 \\
0.1 & 20db & 18.96 & fail & fail & 2.33 & 2.73 & 14.6 & 5.63 \\
0.25 & 12db & 5.12 & 4.3 & 2.03 & 4.2 & 4.13 & 5.51 & 2.3 \\
0.5 & 6db & 5.61 & 1.95 & 3.71 & 6.83 & 3.7 & 7.73 & 2.83 \\
0.75 & 2.5db & 5.02 & 8.57 & 6.41 & 20.94 & 2.24 & 6.5 & 4.33 \\
\\ \textbf{10cm} &   &   &   &   &   &   &   &   \\ 
Noise & SNR & \textbf{FM2D} & \textbf{FM3D} & \textbf{ICP} & \textbf{PCA} & \textbf{FVR} & \textbf{FFVR} & \textbf{FVR3D} \\ \hline
0 & $\infty$ & 15.35 & fail & 10.26 & 16.87 & 3.47 & fail & 7.07 \\
0.1 & 20db & 18.2 & 15.36 & 6.24 & fail & 3.89 & fail & 11.45 \\
0.25 & 12db & fail & 24.12 & 6.02 & 20.12 & 5.4 & fail & 7.53 \\
0.5 & 6db & fail & 4.42 & 5 & 3.96 & 4.65 & fail & 9.47 \\
0.7 & 2.5db & 7.27 & 8.57 & 3.98 & 7.85 & 15.45 & fail & 4.3 \\
\\ \textbf{15cm} &   &   &   &   &   &   &   &   \\ 
Noise & SNR & \textbf{FM2D} & \textbf{FM3D} & \textbf{ICP} & \textbf{PCA} & \textbf{FVR} & \textbf{FFVR} & \textbf{FVR3D} \\ \hline
0 & $\infty$ & fail & fail & 7.85 & fail & 25.87 & fail & fail \\
0.1 & 20db & fail & fail & 7.13 & fail & 12.27 & fail & fail \\
0.25 & 12db & 15.17 & fail & fail & fail & 8.1 & 12.52 & fail \\
0.5 & 6db & fail & fail & 11.11 & fail & 10.13 & 11.09 & fail \\
0.7 & 2.5db & fail & fail & 10.31 & 20.59 & 7.79 & fail & fail \\
\\
\end{tabular}}
\\
\caption{Translation Tracking}
\label{table:trans}
\end{table}


Results show that the basic FVR method is generally the most accurate in terms of registration across the range of camera translation magnitudes and noise levels. In comparison, the FFVR method reduces processing time at the expense of accuracy, results show that in general it is only capable of up to 5cm of translation registration. FVR3D and the 2D feature matching method are capable of registering 5-10cm with larger levels of noise. However, both methods fail to register the full 15cm of camera translation well. \\

ICP performed next best, capable of performing well up to 15cm of translation, but failing to register twice during the tests. PCA was able to handle up to 10cm of too but often had larger registration errors. For the 15cm translation tests, the FVR performed the best followed by ICP, which failed once but outperformed FVR within the two lowest noise brackets. \\


It can be shown that, despite being limited to a single axis of rotation, the FVR algorithm can consistently register camera movements up to 15cm better than other algorithms from the literature. To put these camera translations into perspective at video frame rates, a displacement of 10cm per frame equates to camera velocity of 3 meters per second, this is twice the average person's walking speed making both the FVR method and FVR3D suitable for a majority of applications. \\


\section{Camera Rotation Tracking}
\label{Sec:CamRoteTrackExp}

Table \ref{table:rote} shows results for camera rotation experiments. These experiments were captured with the ASUS Xtion PRO LIVE camera using the same scene as results in section \ref{Sec:CamTransTrackExp}. 3D frames of these scenes are separated by 10, 20 and 30 degrees of camera rotation about the y-axis. These degrees were chosen because they are significantly large enough to be difficult for these algorithms to register against.  \\

Again, different levels of noise were added to each frame prior to registration. This experiment was designed to test the robustness of the FVR based methods in registering camera pose. \\


%rotation
\begin{table}[!htb]
\centering
\scalebox{1.0}{
\begin{tabular}{ccccccccc}
\\ \textbf{10deg} &   &   &   &   &   &   &   &   \\ 
Noise & SNR & \textbf{FM2D} & \textbf{FM3D} & \textbf{ICP} & \textbf{PCA} & \textbf{FVR} & \textbf{FFVR} & \textbf{FVR3D} \\ \hline
0 & $\infty$ & 0.17 & 9.77 & 0.19 & 0.3 & 0.18 & 0.35 & 4.05 \\
0.1 & 20db & 0.21 & 0.41 & 0.17 & 0.67 & 0.15 & 0.21 & 10.54 \\
0.25 & 12 & 0.16 & 0.34 & 0.13 & 0.39 & 0.15 & 0.44 & 5.44 \\
\\ \textbf{20deg} &   &   &   &   &   &   &   &   \\ 
Noise & SNR & \textbf{FM2D} & \textbf{FM3D} & \textbf{ICP} & \textbf{PCA} & \textbf{FVR} & \textbf{FFVR} & \textbf{FVR3D} \\ \hline
0 & $\infty$ & 1.37 & 13.1 & 1.81 & 16.93 & 7.29 & 1.32 & 3.18 \\
0.1 & 20db & 1.35 & 14.32 & 15.29 & 2.97 & 0.7 & 2.03 & 1.92 \\
0.25 & 12db & 1.6 & 3.26 & 14.7 & 0.64 & 1.54 & 4.29 & 2.24 \\
\\ \textbf{30deg} &   &   &   &   &   &   &   &   \\ 
Noise & SNR & \textbf{FM2D} & \textbf{FM3D} & \textbf{ICP} & \textbf{PCA} & \textbf{FVR} & \textbf{FFVR} & \textbf{FVR3D} \\ \hline
0 & $\infty$ & 3.87 & 10.15 & 3.16 & 9.97 & 2.11 & 5.68 & 3.63 \\
0.1 & 20db & 3.77 & 16.65 & 39.37 & 77.6 & 3.84 & 2.82 & 1.39 \\
0.25 & 12db & 3.68 & 2.76 & 5.12 & 5.78 & 3.3 & 7.15 & 6.38 \\
\\
\end{tabular}}
\\
\caption{Rotation Tracking}
\label{table:rote}
\end{table}

In the first sub-table, registration errors for 10 degrees of rotation are presented. Results show that the FVR method outperforms the other methods. Here, ICP performs next best followed by the 2D feature matching method. This was expected as the FVR method was designed to be both robust to noise and to handle larger rotations. It can also be seen that in the 20 degree and 30 degree tests, the FVR method also beats both of these algorithms in terms of having to lower registration error. FFVR also worked well at different noise levels up to 30 degrees of rotation. FVR3D was found to be as robust as ICP at registering rotation but not at accurate as the 2D feature matching method. \\ 

It should be noted that twelve degrees of camera rotation per frame is almost a full rotation per second in video rates. This is so fast that most cameras would acquire too much motion blur for registration to be possible. Therefore this test indicates the robustness of the FVR method in comparison with the other algorithms within the context of camera pose estimation. \\


\section{Reconstructed Scenes}
\label{Sec:FVRQual1Exp}

\begin{figure}[!htb] 
        \centering
        \begin{subfigure}[b]{3.0in}
                \includegraphics[width=3.0in]{images/ch2/unit21}
                \caption{Apartment}
                \label{fig:RECON_UNIT}
        \end{subfigure}
        \begin{subfigure}[b]{3.0in}
                \includegraphics[width=3.0in]{images/ch2/officeA}
                \caption{Office}
                \label{fig:RECON_OFFICE}
        \end{subfigure}
        \begin{subfigure}[b]{3.0in}
                \includegraphics[width=3.0in]{images/ch2/outdoorA}
                \caption{Garden}
                \label{fig:RECON_GARDEN}
        \end{subfigure}
       \caption{Reconstructed Scenes.}
       \label{fig:RECONSTRUCTIONS}
\end{figure}

Qualitative experiments also show the ability of the FVR registration method to reconstruct 3D scenes. In these experiments, two indoor environments (Apartment and Office) as well as one outdoor environment (Garden) were reconstructed and are shown in figures \ref{fig:RECON_UNIT}, \ref{fig:RECON_OFFICE} and \ref{fig:RECON_GARDEN} respectively. \\

The Apartment reconstruction was recorded by moving the ASUS Xtion PRO LIVE active camera through a room and rotating the camera. Each frame was registered using the FVR algorithm. Some of the frames in the apartment scene contain walls which have few features. Between frames, walls also had colour contrast shifts. These shifts are due to the ASUS camera's automatic contrast feature which adjusts contrast based on colour histograms. Despite these setbacks, accurate 3D reconstruction was achieved by the FVR method as illustrated in figure \ref{fig:RECON_UNIT}. \\


The office reconstruction was also generated by rotating the ASUS Xtion PRO LIVE active camera about the y-axis while moving the camera around the room. This time, during rotation, the camera was focused on both foreground and background objects. Here, the entire video sequence was accurately registered. It can be seen that despite the foreground and background focus, the global reconstruction is accurate. This scene, as in the apartment scene has usable texture which should not cause large amounts of texture confusion. These qualitative experiments show that despite being a closed form solution, the FVR has reconstruction accuracy comparable to existing feature based SLAM methods. \\


Typical feature based methods work well with indoor environments where local features are readily distinguishable and easy to match. They do not tend to work as well in complex outdoor scenes where feature confusion is likely. To assess performance in such outdoor scenes, a garden scene containing bushes, plants and a ground covering of bark and rocks was captured for testing. Again, this scene was captured using the ASUS Xtion PRO LIVE active camera moving around the out-door garden. The proposed FVR method was able to produce a good quality reconstruction of this garden scene, as shown in figure \ref{fig:RECON_GARDEN}. This shows that reconstruction approaches which integrate or make use of the FVR registration method may have an advantage in performing 3D reconstructions in these types of scenes, scenes which are of common disturbance to many existing feature matching methods, as expressed in the literature.   



\subsection{Comparison}

\begin{table}
\resizebox{\textwidth}{!}{%
\begin{tabular}{ccccccccc}
\textbf{\textit{Algorithm}} & \textbf{translation} & \textbf{rotation} & \textbf{scale} & \textbf{object motion} & \textbf{wide baselines} & \textbf{monocular} & \textbf{stereo} & \textbf{RGB-D}\\ \hline
FVR & yes & 1 axis only & yes & yes & yes & no & yes & yes\\
FFVR & yes & 1 axis only & no & yes & no & no & yes & yes\\
FVR-3D & yes & yes & yes & yes & no & no & yes & yes\\
MMVR & yes & no & no & no & no & yes & no & no\\
\end{tabular}}
\label{tab:GridRT}
\caption{FVR Comparison Table}
\end{table}


Here we present a look-up table to assess the various abilities of each 3D reconstruction algorithm in terms or translation, rotation, scale, non-static object motion robustness and ability to handle various sensory types. Table \ref{tab:GridRT} presents these results. In terms of translation, each of the presented algorithms is able to register with respect to translation. In experiments, it was shown that the FVR algorithm outperforms others when this type of camera movement is present. Especially at wider baselines of 10 to 15 centimetres. In Registering rotation, the FVR and FFVR methods are only capable of a single axis of rotation. FVR-3D however, is capable of registering against all 3 axes of rotation. Results have shown that in the presence of rotation, the FVR-3D performs at or above the level of the more accurate algorithms from the literature: ICP and FM-2D. It was found that MVVR was not able to handle rotation well enough, this is due to the high levels of noise found in the depth maps produced by monocular based techniques. \\

In terms of registration against scale, only the FVR and FVR-3D method were capable of registering against it. From the literature, ICP and PCA are not capable of handling such transformations without some modification. Results have shown that these algorithms are capable of handling non-static object motion well. Results on the Kitti Vision benchmark show that FVR-3D outperforms the state of the art, even on data sets which have non-static moving agents. Results also showed that the FVR was the most accurate in the registration of wide baselines, that is translations above 5 centimetres and rotations above 5 degrees. It was show that the FVR algorithm outperformed the others, including the top methods used in the literature. Out of these methods, MVVR was shown to be able to register monocular sensor data against translation when noise levels were low. The closer the depth data computed via monocular methods approaches perfectly accurate depth images, the closer the MVVR algorithm approaches FVR performance. In terms of both stereo and active sensor input, FVR, FFVR and FVR-3D are all capable of handling this type of input data well. \\

In summary, results have shown that on stereo camera results and active camera results, FVR-3D method outperforms outperforms all of the other algorithms from the literature. For most of these data-sets the FVR algorithm was a strong second place contender. When registering wider baselines, the FVR algorithm was the top performer and FVR-3D fell behind.

%PTPTPT
\section{Plane-Tree Experiments}
Several experiments were designed in order to evaluate both the FVR method as well as the Plane-Tree and 3D Shade-Tree method. The types of these experiments as well as their conditions and results are presented in this section. In the next section (section \ref{TestDataSection}) the test data used in experimentation is introduced. Following that (in section \ref{ToolsSection}), the tools used to run these experiments are also introduced along with various error metrics (section \ref{metricsSection}) which provide means for comparisons between algorithms. \\

Then different results for experiments analysing the performance of FVR and comparing it to methods used in the current literature are presented. These experiments compare the FVR's ability to track camera movements (sections \ref{Sec:CamTransTrackExp} and \ref{Sec:CamRoteTrackExp}), its qualitative performance (section \ref{Sec:FVRQual1Exp}) and its robustness to noise (section \ref{Sec:FVRMotionExp}). Experiments evaluating the performance of Monocular FVR are then presented in section \ref{Sec:MonocularExperimentsSection} followed by experiments comparing FVR-3D to commonly used procedures from the literature. After these experiments, the Plane-Tree is compared to the Octree. These experiments compare both method's efficacy in compressing 3D mesh data. The Plane-Tree is also compared to various state-of-the-art compression techniques from the literature. Next, qualitative compression results are presented comparing the Plane-Tree to several other methods. Finally, the Plane-Tree is compared to both the Octree and the 3D-ShadeTree in compressing 3D reconstructions. \\

\section{Plane-Tree vs. Octree}
\label{SEC:PTVSOT}
Figure \ref{fig:OTEXPS} shows rate-distortion graphs comparing the Plane-Tree with the Octree compression method. In these rate-distortion graphs we use the mean error between the decoded and input model as the distortion metric. Results show that for the Bunny and Fandisk experiments (figures \ref{fig:OG_BUNNY} and \ref{fig:OG_FANDISK}) the Plane-Tree has better quality for a given bitrate compared to the Octree. It is also evident that the Octree is unable to reach the level of quality the Plane-Tree reaches. In both cases, due to the Plane-Tree's ability to prevent further tree decomposition via its more accurate representation method, the Plane-Tree has a much lower error rate for a given bit-rate compared with the Octree method. \\

In the Horse model graph in figure \ref{fig:OG_HORSE} there is some overlap in the model quality (error rates). This overlap ranges from around $0.0025$ to $0.004$ in which for these levels of quality, the Octree requires around 8 times more storage space compared with the Plane-Tree. These qualitative results show how much of an improvement the Plane-Tree model codec is over the Octree method. \\

\begin{figure}[!htb] 
        \centering
        \begin{subfigure}[b]{2.8in}
                \includegraphics[width=2.5in]{images/results/compression/OTbunny}
                \caption{Bunny Model}
                \label{fig:OG_BUNNY}
        \end{subfigure}%
        \begin{subfigure}[b]{2.8in}
                \includegraphics[width=2.5in]{images/results/compression/OTFandisk}
                \caption{Fandisk Model}
                \label{fig:OG_FANDISK}
        \end{subfigure}
        
        \begin{subfigure}[b]{2.8in}
                \includegraphics[width=2.5in]{images/results/compression/OTHorse}
                \caption{Horse Model}
                \label{fig:OG_HORSE}
        \end{subfigure}%

       \caption{Rate-distortion graphs comparing the Plane-Tree with the Octree.}
       \label{fig:OTEXPS}
\end{figure}
\section{Plane-Tree vs. Existing Techniques}
\label{SEC:PTVSSOTA}
\begin{figure}[!htb] 
        \centering
        \begin{subfigure}[b]{1.8in}
                \includegraphics[width=1.8in]{images/results/compression/bunnysota}
                \caption{Bunny Model}
                \label{fig:SA_BUNNY}
        \end{subfigure}%
        \begin{subfigure}[b]{1.8in}
                \includegraphics[width=1.8in]{images/results/compression/fandisksota}
                \caption{Fandisk Model}
                \label{fig:SA_FANDISK}
        \end{subfigure}
        
        \begin{subfigure}[b]{1.8in}
                \includegraphics[width=1.8in]{images/results/compression/horsesota}
                \caption{Horse Model}
                \label{fig:SA_HORSE}
        \end{subfigure}%
        \begin{subfigure}[b]{1.8in}
                \includegraphics[width=1.8in]{images/results/compression/rabbitsota}
                \caption{Rabbit Model}
                \label{fig:SA_RABBIT}
        \end{subfigure}
       \caption{Rate-Distortion graphs comparing the Plane-Tree to different state of the art codecs.}
       \label{fig:SOTAEXPS}
\end{figure}

In figures \ref{fig:SA_BUNNY}, \ref{fig:SA_FANDISK} and \ref{fig:SA_HORSE} we compare the Plane-Tree with the state of the art transform methods by Bayazit et al \cite{Bayazit103DMesh} and Khodakovsky et al \cite{Khodakovsky00Progressive}. \\

Experiment results comparing the Plane-Tree compression system proposed in this work with the transform-based method by Khodakovsky et al are shown in figure \ref{fig:SA_BUNNY}. This Rate-Distortion graph shows that the Plane-Tree has similar performance with the method by Khodakovsky et al. At low bit-rates the Plane-Tree is highly competitive the with performance matching Khodakovsky's. The competitiveness extends to higher bit-rates as well for the bunny model. \\

Figure \ref{fig:SA_FANDISK} shows comparisons between the Plane-Tree as well as methods by Khodakovsky et al and Bayazit et al. Here, both the Plane-Tree and the method by Khodakovsky et al perform similarly as in the bunny model experiment (figure \ref{fig:SA_BUNNY}). The Plane-Tree stays competitive with that state of the art method, and improves upon the compression performance in comparison to the spectral compression method by Bayazit et al. It can be seen that at low to mid bit-rates, the Plane-Tree method compressed the Fandisk model at a higher level of quality for a given bit-rate. \\

The Plane-Tree was also compared with Bayazit et al using another model, the horse model. Figure \ref{fig:SA_HORSE} shows the result of this experiment. The Plane-Tree method outperforms the transform based method of Bayazit et al whilst decreasing coding complexity compared to the complicated transform method. Overall, in this experiment, the proposed Plane-Tree method remains competitive at higher bitrates, and it outperforms the method by Bayazit et al at lower bitrates. \\

Finally the Plane-Tree is compared with the state of the art low-bitrate compression system FOLProM presented by Peng et al \cite{Peng10Feature}. Unlike the other experiments, the mean-error metric is used in this comparison, as the results presented by Peng et al used this metric. It can be seen that at low-bitrates (below 2 bits per vertex), the Plane-Tree method outperforms this state of the art low-bitrate compression system. \\


In conclusion, these experiments show that the Plane-Tree is extremely competitive with the state of the art compression systems at high bit-rates. At lower bit-rates it improves upon the results by these methods. This algorithm is also powerful in that it may be used to compress 3D volumetric data as well as point-cloud and mesh data. This makes it an interesting candidate for 3D reconstruction compression no matter the format output by a given 3D reconstruction method. \\
\section{Plane-Tree: Qualitative Results}
\label{SEC:PTQUALEVAL}

Figure \ref{fig:FIGS} in Appendix \ref{QualitativeLargeImage} shows qualitative results for the Plane-Tree. The third row shows the bunny, rabbit and horse models along with the number of bytes and bpv required to store them uncompressed. The first row shows each model compressed by the Plane-Tree with a threshold of 8.0. Despite these being crude approximations of the originals, the models are still distinguishable with around 1000 $\times$ less storage space required. \\

In the second row, each model was compressed with the Plane-Tree at a threshold of 1.0. In these experiments, there is little detail missing compared with the physical model. When looking at the bunny's legs, the same ripples are present. In the rabbit model, the outlines inside the ears and on the eyes are still present. On the horse model, the creases on the body and shoulder of the horse are still present. Here, the bunny is compressed to around 70 $\times$ less storage space, the rabbit at around 265 $\times$ less and the horse around 90 $\times$ less storage space. \\


Figure \ref{fig:qualSOTA1} shows the Bunny model compressed using the Plane-Tree method and two state-of-the-art methods, the valence method \cite{touma98triangle} and the spectral method \cite{Karni00Spectral}. The original model in Figure \ref{fig:PT_SOTAQ1_ORIG} requires 2,258,902 bytes for storage. The valence method and the spectral method require ~17,852 bytes to store the model and the compression effects are shown in Figures \ref{fig:PT_SOTAQ1_TG} and \ref{fig:PT_SOTAQ1_KG}. Noticeable artefacts not present in the original model are produced by both codecs. The model coded by the valence method fairs worse than the one coded by the spectral method. The model compressed by the Plane-Tree requires just over half the amount of bytes than the other codecs. The Plane-Tree coded model is smoother than the other models (despite flat shading) too. This is likely due to the way the Plane-Tree compression system approximates the original model using planes. This process doesn't introduce these artefacts, allowing the Plane-Tree to represent the original model more accurately. \\   

\begin{figure}[H] 
        \begin{center}
 		\begin{subfigure}[b]{6cm}
 			   \centering
 			   \includegraphics[width=5.8cm]{images/experiments/pt_qual/original1}
 			   \captionsetup{justification=centering}
                \caption{Original\\518.78 bpv\\2,258,902 bytes}
                \label{fig:PT_SOTAQ1_ORIG}
        \end{subfigure}%
        \begin{subfigure}[b]{6cm}
                \includegraphics[width=5.8cm]{images/experiments/pt_qual/pt_11004}
                \captionsetup{justification=centering}
                \caption{Plane-Tree\\2.52 bpv\\11,003 bytes}
                \label{fig:PT_SOTAQ1_PLT}
        \end{subfigure}
        \begin{subfigure}[b]{6cm}
                \includegraphics[width=5.8cm]{images/experiments/pt_qual/tg}
                \captionsetup{justification=centering}
                \caption{Valence Method \cite{touma98triangle}\\4.1 bpv\\17,852 bytes}
                \label{fig:PT_SOTAQ1_TG}
        \end{subfigure}%
        \begin{subfigure}[b]{6cm}
                \includegraphics[width=5.8cm]{images/experiments/pt_qual/kg}
                \captionsetup{justification=centering}
                \caption{Spectral Method \cite{Karni00Spectral}\\4.1 bpv\\17,852 bytes}
                \label{fig:PT_SOTAQ1_KG}
        \end{subfigure}
       \caption{The Bunny Model Compressed Using the Plane-Tree, Valence and Spectral Methods.}
       \label{fig:qualSOTA1}
       \end{center}
\end{figure}

Figure \ref{fig:qualSOTA2} shows the bunny model coded using the Plane-Tree and the codec by Khodakovsky et al. Both the Plane-Tree and method by Khodakovsky et al. require just 1349 bytes of storage to represent these low bit-rate models. Both methods however, use different strategies and introduce different styles of noise into the compressed model. The method by Khodakovsky et al. shrinks the models and warps the shape, simplifying it. The Plane-tree approximates the shape and positions of the vertices more closely but does not produce smooth surfaces. We conclude that both methods may be beneficial in different situations. The method by Khodakovsky et al. may be useful in situations where visual appeal are important. The Plane-Tree method may be more useful in applications where model shape and size are more important. \\

\begin{figure}[!htb] 
        \begin{center}
 		\begin{subfigure}[b]{6cm}
 			   \centering
 			   \includegraphics[width=5.8cm]{images/experiments/pt_qual/original2}
				\captionsetup{justification=centering}
                \caption{Original\\518.78 bpv\\2,258,902 bytes}
                \label{fig:PT_SOTAQ2_ORIG}
        \end{subfigure}%
        \begin{subfigure}[b]{6cm}
                \includegraphics[width=5.8cm]{images/experiments/pt_qual/planetree2_shade}
                \captionsetup{justification=centering}
                \caption{Plane-Tree\\0.31 bpv\\1,349 bytes}
                \label{fig:PT_SOTAQ1_PLT}
        \end{subfigure}
        \begin{subfigure}[b]{6cm}
                \includegraphics[width=5.8cm]{images/experiments/pt_qual/khodakovsky_shade}
                \captionsetup{justification=centering}
                \caption{Khodakovsky et al. \cite{Khodakovsky00Progressive}\\0.31 bpv\\1,349 bytes}
                \label{fig:PT_SOTAQ1_KHKY}
        \end{subfigure}
       \caption{The Bunny Model Compressed Using the Plane-Tree and the Wavelet Compression System Khodakovsky et al.}
       \label{fig:qualSOTA2}
       \end{center}
\end{figure}


\section{Plane-Tree: Reconstruction Compression}
\label{SEC:PTONRECON}

In this section, experiments comparing the Plane-Tree with the Octree in terms of 3D frame or 3D occupancy reconstruction grid compression. Essentially this experiments compares the performance of the Plane-Tree and Octree in 3D occupancy grid compression. \\

The Octree is used for comparison since it is the closest relative of the Plane-Tree and existing methods which use the Octree are presently used in 3D reconstruction research. There is also a 3D implementation of the Interpolating Leaf Quad-Tree \cite{Lincoln13Interpolating} compression algorithm used in the comparison. Being the 3D extension, it is therefore referred to as the Interpolating Leaf Octree algorithm (ILOT). \\

In this experiment, Rate-Distortion is measured in PSNR (Peak Signal to Noise Ratio) which is a measurement of the amount of accuracy between the original model and the compressed version for a given algorithm and level of compression. The larger the PSNR, the higher the quality in which the compression system produces for a given bitrate. \\

A Rate-Distortion graph comparing the Plane-Tree, Octree and ILOT is presented in Figure \ref{fig:3DReconCompression1}. Results show that for low bit-rates the ILOT outperforms the Octree method. However as higher quality is required, the Octree in turn outperforms the ILOT method. The Plane-Tree algorithm proposed in this work is shown to be dominant compared to both algorithms. For each algorithm, at any bitrate, the Plane-Tree has a much larger Peak-Signal-To-Noise ratio. \\

\begin{figure}[!htb]
\centering
\includegraphics[width=4.0in]{images/results/compression/psnr1}
\caption{PSNR vs Bitrate comparing the ILOT, OT and PT compression methods.}
\label{fig:3DReconCompression1}
\end{figure}

In conclusion, as in results comparing the Plane-Tree with state-of-the-art mesh compression algorithms, the Plane-Tree is shown to perform well and even outperform ILOT and the generic Octree data structures when compressing 3D data used in reconstructions. 




