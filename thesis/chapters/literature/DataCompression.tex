\subsection{Model Compression}

The Feature-Oriented Geometric Progressive Lossless Mesh coder (FOLProM) \cite{Peng10Feature} is a state of the art codec which is progressive. It also aims to be an effective low-bitrate codec. It classifies segments of the mesh as being visually salient or not. Salient segments are preserved more during compression compared to non-salient ones. \\

\subsection{Spectral Compression}

Karni and Gotsman \cite{Karni00Spectral} proposed a lossy method which compresses a spectral representation of a mesh. This algorithm generally partitions the mesh and compresses each partition separately since it does not work on large meshes. Encoding a basis function for each partition, coefficients are quantized, truncated and entropy coded. Results show this method outperforms the valence method \cite{touma98triangle} at coarse quantization levels. Bayazit et al. \cite{Bayazit103DMesh} also developed a progressive method based on spectral compression. This method is based on the region adaptive transform in the spectral domain and is advertised as a current state of the art lossy 3D data compression method. \\

\subsection{Wavelet Methods}

A lossy wavelet based compression system was proposed by Khodakovsky et al \cite{Khodakovsky00Progressive}. This technique samples the mesh, and uses the wavelet transform to decorrelate the data. Coefficients are quantized and stored in a structure called a zero tree which increases compression performance. This method is shown to outperform the valence method. Other wavelet approaches \cite{Guskov00Normal,Khodakovsky04Normalmesh} also sample the mesh and use a multi-resolution representation in which the data is described using local normal directions on the mesh surface.

Gu et al \cite{Gu02Geometry} devised a solution for representing 3D models as 2D images which are then compressed using state of the art image compression methods (based on wavelets). To form this representation, the mesh is cut along a network of edge paths, opening the mesh into a topological disk, which is then sampled onto a 2D grid. Each pixel in the image has a corresponding coordinate in the model, with pixel neighbourhoods describing connectivity. Comparisons with the method by Khodakovsky et al reveal the geometry image codec does not have as high compression performance.

