
\subsection{$G^2$o}

The Graph SLAM algorithm sets relationships between camera poses and locations as nodes in a graph. Rather than defining camera pose estimations as truth, they are represented using a Gaussian in which the peak probability is the estimated value. Then using the visible landmarks (visible from multiple views) and the functions for the camera poses the joint graph is optimized for the improved pose estimates. Once this is complete, a 3D reconstruction map is produced. \\

Typically an occupancy grid or octree is used to represent the map \cite{Wurm10Octomap}. Further detail about the Graph SLAM method is outside the scope of this work, more detail can be found in the original paper on Graph SLAM \cite{Kummerle11G}. \\

\subsection{Bundle Adjustment}

Bundle Adjustment is another optimization technique used in SLAM and 3D reconstruction \cite{Lourakis09Sba}. In this context bundle refers to the light leaving each camera projected into world coordinates. This techniques refines both camera pose and visual reconstruction. It requires some estimates for camera pose, 3D world point positions and image point locations for each view. \\

Using these data and any non-linear optimization method (such as Gauss-Newton or the Levenberg–Marquardt algorithm) the rigid motion based camera movement and the 3D world-coordinates of the points are optimized. \\

In the literature, Fioraio \cite{Fioraio11Realtime} presented a system which uses bundle adjustment to align rgb-d data. It has also been used in conjunction with feature matching to produce sparse 3D reconstructions \cite{Klein07Parallel,Agarwal09Building}.

[END BUNDLE ADJUSTMENT]



[BG Bylow]

? not sure where to put this yet [maybe icp...]
whelan [24 \cite{Whelan13Robust}] estended with rolling reconstruction volume and color fusion, evaluated alternative methods for visual odometry estimation
pure visual odometry induces significant drift, so matching with global model is their preference
whelan integrated photometric + icp methods for kinect fusion [24 \cite{Whelan13Robust}]
?






\ref{Newcombe11Kinectfusion}

\subsection{RGB-D Sensor Feature Based Systems}
RGB-D SLAM systems use both depth and image data and are capable of generating dense 3D reconstructions. Many of these methods rely on feature matching techniques \cite{Engelhard11Real,Henry10Rgb,Endres12Evaluation}. RANSAC is often used to filter outliers for the estimation of camera parameters\cite{Engelhard11Real,Henry10Rgb,Endres12Evaluation}. Another method which has also been used extensively in the area is Iterative Closest Point (ICP) \cite{Engelhard11Real,Henry10Rgb,Bylow13Real,Newcombe11Kinectfusion,Stuckler12Robust,Izadi11Kinectfusion}. ICP iteratively registers point cloud data, and is used to refine camera parameter estimates. A method named KinectFusion was proposed by Newcombe et al \cite{Newcombe11Kinectfusion} which uses RANSAC and a GPU implementation of IPC. Whelan et al \cite{Whelan12Kintinuous} extended this method allowing it to map larger areas using Fast Odometry From Vision (FOVIS) over ICP. Bylow et al \cite{Bylow13Real} improved the ICP approach by registering data using a signed distance function.
\subsection{Non-Feature Based Methods}
Several RGB-D SLAM systems are also non-feature based \cite{Weikersdorfer14Event,Izadi11Kinectfusion,Kerl13Dense}. Weikersdorfer et al \cite{Weikersdorfer14Event} presented a novel sensor system named D-eDVS along with an event based SLAM algorithm. The D-eDVS sensor combines depth and event driven contrast detection. Rather than using features, it uses all detected data for registration. Kerl et al \cite{Kerl13Dense} proposed a dense RGB-D SLAM system which uses a probabilistic camera parameter estimation procedure. It uses the entire image rather than features to perform SLAM.
\subsection{Summary}
As is evident from the current literature, SLAM typically relies on feature matching and RANSAC. However, these approaches fail when there are too few features, when feature confusion occurs or, when features are non-stationary due to object motion. As the extent of random feature displacement becomes more global the effectiveness of these approaches diminishes. Feature matching also dominates in image registration. However, Fourier based methods have been shown to work well under larger rotations and scales \cite{Gonzalez11Improving} whilst being closed form, insensitive to object motion and scaling naturally to GPU implementations. Accordingly, we propose a novel, closed form Fourier based SLAM method.

Simultaneous localization and mapping (SLAM) has applications in many fields including: robotics, business, architecture and engineering, and science. Its goal is to generate a map (2D birds-eye view, or 3D) of an environment captured by camera and/or other means. In this work we focus on monocular systems, or systems which generate location and mapping data using information generated by a single basic video camera. To this end, current methods rely on the computation of the fundamental and essential matrices. These feature matching techniques fail in cases where features are not stable or where feature confusion occurs. 

It has been shown [1] that using volume registration to compute dense 3D maps is not only independent of feature matching, but it is a closed form solution and is robust to noise and object motion. However, this method requires RGB-D video input provided by special hardware. In this paper we present preliminary results in applying volume registration to generate dense 3D maps from monocular video data. To achieve this, disparity maps are generated between video frames. This data is then used as input for the RGB-D volume registration method.
