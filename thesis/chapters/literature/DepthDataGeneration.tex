\subsection{Introduction}

In 3D reconstruction, dense depth data is very important, without it there is no dense 3D reconstruction, only sparse mapping. This section introduces some techniques and research on different methods (both hardware and software) of depth data generation. In the first section the use of sensors which are capable of generating depth data is introduced. Next, methods using stereo camera pairs are discussed. Finally, monocular techniques are discussed. 


\subsection{Sensors}

In 3D reconstruction, it is often ideal to use specialized sensors which capture reliable and dense depth data on a per pixel basis. One such camera is the RGB-D (Red, Green, Blue \& Depth) camera. These sensors are becoming more accurate and less expensive and are now found in mobile technologies. \\

Research by Zhang et al \cite{Zhang12Microsoft} used an RGB-D camera to generate smooth, continuously updating dense 3D reconstructions using only depth data. Here, using the data, 6 degrees of freedom were tracked. This techniques uses depth information only, and as such it works in the absence of visual light (it works in the dark) unlike passive camera approaches \cite{Klein07Parallel, Newcombe10Live,Stuhmer10Real} and techniques which use color data along with depth data \cite{Henry10Rgb}. \\

The Kinect is one such rgb-d camera. It uses a structured light based depth sensor along with an application specific integrated circuit to generate an 11-bit $640\times 480$ depth map at 30 times per second (real-time). \\

There is no doubt that these sensors are the easiest way to compute ready dense depth data for 3D reconstruction, but there are several drawbacks. Depth images contain holes. This is caused by a lack of structured light on a captured surface. Some materials simply do not reflect infra-red light (surfaces at steep angles or very thin objects). Also, when moving fast, the device easily experiences motion blur which leads to missing and incorrect data. \\

Some 3D reconstruction techniques use the approach of dense mapping and tracking via depth sensors and lasers. These approaches usually compute feature matches to align the frames or some sort of error minimization function.


\subsection{Stereo Cameras}

\label{StereoMethodsSection}

Stereo camera set-ups employ two cameras capturing a singular scene. Using a variety of techniques, depth data is computed from the stereo pair by measuring parallax. Stereo cameras may be calibrated or un-calibrated. Un-Calibrated stereo pairs require calibration using the Fundamental or Essential matrix, whilst pre-calibrated cameras require no further action before parallax may be computed. Using a variety of pose estimation techniques, the dense depth data may be integrated forming a 3D reconstruction. \\

In this section, some important techniques proposed within the area of depth data generation via stereo algorithms. Readers wanting to research techniques prior to 2005 are encourages to seed a survey by Scharstein and Szeliski \cite{Scharstein02Taxonomy}. \\


Sun et al \cite{Sun05Symmetric} presented an occlusion handling stereo matching algorithm. Their method incorporates a visibility constraint into the energy function for the belief propagation global optimisation method. Klaus et al \cite{Klaus06Segment} devised a stereo correspondence algorithm which uses colour segmentation combined with a self adapting matching score which minimizes the number of reliable occurrences. This method also uses belief propagation to assign a single disparity to each segmented region. Hirschmuller \cite{Hirschmuller05Accurate} proposed a semi-global matching method based on mutual information and approximation of a global smoothness constraint. \\

Yoon \cite{Yoon06Adaptive} presented work on a local method for computing disparity. In this method, the window is weighted based on geometric proximity and colour similarity. Darabiha et al \cite{Darabiha06Reconfigurable} presented an FPGA based local window stereo algorithm. This method obtains sub-pixel accuracy for 256$\times$360 images at 30 frames per second. Their window matching method uses correlation. Klaus et al. \cite{Klaus06Segment} devised a stereo method which segments the image before fitting regions with disparities. This fitting is based on interpolating depth values along each region. A self adapting matching score for segments is also used to maximize correspondences. Belief propagation is then used to optimize depth among the regions. \\


Yang et al \cite{Yang07Spatial} came up with a method which uses super pixel resolution to improve disparity images. First a depth map is computed at a lower resolution, it is then up-sampled to the same resolution as the corresponding colour image. The depth map is then used as a hypothesis to compute a cost volume, which is bilateral filtered. Then, a winner take all and a sub pixel estimation procedure are performed to estimate the full resolution depth map. This method is shown to improve sub-pixel resolution by up to 100 times compared to previous methods.\\


Sarkis et al \cite{Sarkis07Fast} improved the efficiency of the graph cut global optimisation based disparity algorithm whilst maintaining similar accuracy. Their method involves splitting the global space up using a quad-tree. Each space has an adapted energy function which is minimized using the graph cuts algorithm. Results show this method improves efficiency by up to 3 times over similar works. Wang and Zheng \cite{Wang08Region} came up with an inter-regional cooperative optimization based stereo correspondence technique. This method uses a local adaptive window based approach to compute an initial depth map. The original image is also segmented using a colour based mean shift technique. The segmented image and the initial depth map are then input into a region based optimization method.  \\


Yang et al. \cite{Yang08Near} attempted to solve the problem of stereo mapping for texture-less image regions. These regions are known to be difficult because there is less texture to correlate with. They use colour segmentation and plane-fitting with loopy belief propagation for error correction. Wang and Zheng \cite{Wang08Region} used a stereo method which uses image regions, plane fitting and segmentation as well as a novel region stereo optimization method. Ernst and Hirshmuller \cite{Ernst08Mutual} presented a GPU implementation of the semi-global matching technique (scan-line optimization) for stereo correspondence. Bleyer et al \cite{Bleyer09Stereo} devised a method which extracts disparities and alpha matting information at the same time. Alpha matting is used to generate artificial views for viewpoint effects. This method divides the image up into segments and computes the depth and alpha value for each of these segments. \\


Bleyer and Gelautz \cite{Bleyer09Temporally} presented a method for generating stereo disparity information from an un-calibrated stereo video. First, the video is segmented into scenes. Next, the two camera shots are calibrated before dynamic programming is used to optimize the disparity calculation. Then the disparity estimates are smoothed temporally in order to achieve robustness to disparity flickering. Hirschmuller and Sharstein \cite{Hirschmuller09Evaluation} presented a survey on stereo depth mapping with respect to radiometric differences. They investigated different metrics used in local matching methods as well as their relationship with radiometric variations. Also investigated were the effects of different filters: Laplacian of Gaussian (LOG), bilateral background subtraction, rank, SoftRank, census and ordinal. \\


In his masters thesis, Olofsson \cite{Olofsson10Modern} presented a survey and evaluation of various global and local stereo vision methods. He also presented a novel and efficient local method. This technique is shown to achieve state of the art results with some datasets. Bleyer et al \cite{Bleyer10Surface} introduced a method which models stereo images using smooth surfaces. This technique is based on the assumption that the entire scene is composed of a few smooth surfaces, and each pixel is a part of a surface. Colour segmentation is used to estimate different surfaces. Bleyer et al \cite{Bleyer11Patchmatch} presented a variation on the Patchmatch algorithm (PMA). PMA is a global optimisation method which models each pixel as a plane in 3D space. Each pixel is set to a random value within a larger region, as long as there is at least one close guess for the planes of one of the pixels, this information can be propagated. The default PMA performs spatial propagation, and uses adaptive support weighting to improve correspondence around the border. The method by Bleyer et al introduced view and temporal propagation to the original PMA. \\


Bleyer et al \cite{Bleyer11Object} later presented a joint stereo matching and segmentation algorithm. This method models segmented regions as objects having colour and a disparity distribution, they also use a novel 3D connectivity property for each object region. Lu et al \cite{Lu11Revisit} presented a method for increasing depth map quality given a colour original of the same scene at higher resolution. This allows disparity maps to be computed quickly by computing a lower resolution depth map before scaling-back the resolution. This method makes use of markov random fields and is unique in applying the technique to super pixel resolution in terms of depth mapping. This method is also uses state of the art disparity computation algorithms and so improves efficiency whilst retaining accuracy. \\ 


De-Maeztu et al \cite{De11Linear} presented a novel cost aggregation step for computing disparity images from a stereo pair. Their method works similarly to weighted pixel and scalable window routines but has complexity independent of window size. Unlike other cost aggregation methods, it can be used with colour and includes a novel disparity refinement pipeline. This method effectively filters the image so the pixels are weighted prior to performing some local disparity cost computation. Mei et al \cite{Mei11Building} presented a GPU based stereo correspondence algorithm which makes use of cost aggregation followed by scan-line optimization. Results show this GPU based algorithm is among the state of the art. \\


Mizukami et al \cite{Mizukami12Sub} described a novel method to reliably compute disparity cost volumes for sub-pixel depth mapping. It uses a combination of interpolation and an edge preserving filter. First a sub-pixel cost volume is computed, then this volume is filtered. Finally a two step sub-pixel disparity search is performed. Later, Zhu et al \cite{Zhu12Locally} presented a novel regularization method for stereo matching. This regularization method overcomes noise and captures general disparity at higher octaves and between regions. Lee et al \cite{Lee13Local} introduced a non-iterative one pass method for improving local stereo methods. To this end, a novel three mode cross census transform with a noise buffer is introduced. This method is used for both stereo image and video calculation. Stereo Video computation also makes use of optical flow. \\


Chen et al \cite{Chen13Novel} presented a local windowing based method which uses an adaptive support weight to achieve state of the art local method results. This method uses a novel trilateral filter as a weighting function. The trilateral filter extends the bilateral filter by adding in a component measuring boundary strength. Lu et al \cite{Lu13Patch} presented a super pixel variation of the Patchmatch stereo correspondence technique. Mei et al \cite{Mei13Segment} proposed a tree based approach for optimizing cost volumes. Their method first segments the image based on colour, instead of forming a graph between segments, and calculating the minimal spanning tree. A tree graph is created for each segment, then these graphs are linked with the optimization algorithm.\\


Tan et al \cite{Tan14Stereo} made an improvement to segmentation for use in disparity mapping. Since under-segmented regions contain disparity discontinuities (many methods assume regions have a global or planar based disparity value) and over-segmented regions contain noise, Tan et al proposed a new segmentation based stereo algorithm. This method makes use of a cost volume watershed algorithm and a new region merging strategy. It detects when regions are under-segmented and fixes the situation accordingly. Their method first computes information from an arbitrary segmentation method as well as an arbitrary local windowing disparity algorithm. This information is fed forward into their cost volume watershed. Using discontinuities in the cost volume, they further segment the regions, then a novel region merging method is performed, this final segmentation information is used for the global belief propagation disparity mapping method. \\


Yang \cite{Yang14Pattern} first developed a non-local disparity calculation algorithm based on using the minimal spanning tree to find an optimal solution using the cost volume. This method is supposedly improved upon by Vu et al \cite{Vu14Efficient}. This other method was designed for robustness to texture-less regions. It formulates the cost volume as a minimal spanning tree search problem. Yang et al also contributed some software for interactive depth of field effects called scribble2focus. Tan et al \cite{Tan14Soft} devised a multi-resolution based approach to disparity selection using cost aggregation over a cost volume. Results show this method performs close to global methods for reduced complexity. Lie et al \cite{Liu143d} posed the stereo correspondence problem in terms of interacting 3D entities. Their solution is aimed at curved feature depth estimation, and so they formulate their cost functions according to this constraint. \\



\subsection{Monocular Approaches}

Monocular depth generation includes any and all systems which use a single colour or grey-scale camera to generate dense depth data. Dense depth data is often computed using consecutive frames of a video sequence. If a relationship between frames is known, then the depth data may be either triangulated (given camera pose) or computed as in other stereo methods by calibrating image pairs first. The most popular technique is the computation of the essential matrix in order to perform stereo calibration. Once stereo calibration is performed, generic disparity computation methods (reviewed in \ref{StereoMethodsSection}. \\

An overview of computing the essential and fundamental matrix and the role they play in not only computing dense depth but computing camera pose is discussed in section \ref{FundamentalMatrixSection}. 

