
In this section we describe the general technique of recovering pose estimation via Fourier volume registration techniques. Several methods may be used and each has its own advantages and disadvantages and suitability depends on pose restriction, camera accuracy, noise levels and input data. 


overall description

computing camera pose, computing error

integration into volumetric cube

advantages / disadvantages



Figure \ref{fig:PIPELINE} shows a functional block diagram of our method. The input data are two 3D volumes ($Volume_1$ and $Volume_2$) and the output is the transformation matrix required to register the two volumes. The volumes are first Hanning windowed. Next, a translation independent representation is obtained for each by taking the magnitude of their 3D FFTs. Then a log function is applied to the resulting magnitude values, improving scale and rotation estimation \cite{Gonzalez11Improving}. Following a log-spherical transformation, 3D phase correlation is performed to find the global rotation and scale relationship between $Volume_1$ and $Volume_2$. $Volume_1$ is then inversely transformed by the rotation and scale parameters, leaving only the translation to be resolved. This is found by applying phase correlation again between the transformed $Volume_1$ and $Volume_2$. 