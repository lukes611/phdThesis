\subsection{Recovery of Translation Values}

Given a volume $V_1$ and a spatially shifted version of it $V_2$, the offset along each axis, $(x,y,z)$ may be recovered if a suitable correlation between the two volumes can be found. \\

The measure of correlation between $V_1$ and $V_2$ can be found by shifting $V_1$ and $V_2$'s mean values to zero, then summing the element-wise multiplication of $V_1$ by $V_2$. Equation \ref{eqn:CorrelationEquation} computes this correlation measure.

\begin{equation} \label{eqn:CorrelationEquation}
\sum_{z=0}^{N}\sum_{y=0}^{N}\sum_{x=0}^{N}(V_1(x,y,z)-avg(V_1)) \times (V_2(x,y,z)-avg(V_2))
\end{equation}

Using this measurement, two volumes which are similar in signal shape (element-wise-value to location correspondence) will give a larger measure of correlation than two volumes with a differing signal shape. If the volumes are first normalized we can regard volume $x$ is more aligned with volume $y$ than with volume $z$ given $x$ correlated with $y$ gives a larger value than $x$ correlated with $z$. \\

Cross-correlation searches over a space of translation parameters and outputs the optimal translation to align two volumes. It is optimal in the sense of correlation measurement. This is used as a best guess in terms of the alignment of the two volumes. It can be thought of as the optimization of parameters $x,y,z$ in equation \ref{eqn:CrossCorrelationEquation}.

\begin{equation} \label{eqn:CrossCorrelationEquation}
CrossCorrelate(Transform(V_1, x,y,z), V_2)
\end{equation}

Since we typically do not know the range of translation values $x,y,z$ to optimize for, we take into account the range from $[0,N]$ where $N$ is the width/height/depth of the volume. This gives a complexity of $N^6$. This is too computationally complex for practical volume sizes. Therefore, we use the properties of the Fourier Transform to reduce computational complexity.


 

Using the properties of the Frequency domain, we can efficiently recover the optimal alignment translation parameters using a function called $PhaseCorrelation$ (Eq. \ref{eqn:PC_basic}). This function takes two volumes as input and returns the best alignment translation between them in terms of maximizing correlation.
\begin{equation} \label{eqn:PC_basic}
(x, y, z) = PhaseCorrelation(V_x, V_y)
\end{equation}
The $PhaseCorrelation$ function first applies 3D FFTs to volumes, $V_1$ and $V_2$, converting them into the frequency domain, i.e. $F_{1_{x,y,z}} = FFT(V_1)$ and $F_{2_{x,y,z}} = FFT(V_2)$. Taking the normalised cross power spectrum using Eq. \ref{eqn:PHCOR_eq} based on these frequency domain volumes computes the frequency domain of a new volume called the phase correlation volume. \\


\begin{equation} \label{eqn:PHCOR_eq}
F_{3_{x,y,z}} = \frac{F_{1_{x,y,z}} \circ F_{2_{x,y,z}}^*}{ | F_{1_{x,y,z}} \circ F_{2_{x,y,z}}^* | }
\end{equation}

Here, $\circ$ is an element-wise multiplication and $|x|$ is the magnitude function. Taking the inverse FFT of the frequency domain of the phase correlation volume, $F_3$ gives the phase correlation volume itself, $V_3$ ($V_3 = FFT^{-1}(F_3)$). The location of the peak value in the phase correlation volume $V_3$, $(x_1, y_1, z_1)$ gives the shift between the $V_1$ and $V_2$. The phase correlation volume is typically noisy making the peak difficult to locate. Each in the phase correlation volume evaluates the correlation between $V_1$ (translated by the location of the peak) and $V_2$.


\subsection{Recovery of Y-Axis Rotation}

\subsection{Recovery of Scale}

\subsection{Full Recovery of 3D Rotation}
