\subsection{Recovery of Translation Values}

Given a volume $V_1$ and a spatially shifted version of it $V_2$, the offset along each axis, $(x,y,z)$ may be recovered if a suitable correlation between the two volumes can be found. \\

The measure of correlation between $V_1$ and $V_2$ can be found by shifting $V_1$ and $V_2$'s mean values to zero, then summing the element-wise multiplication of $V_1$ by $V_2$. Equation \ref{eqn:CorrelationEquation} computes this correlation measure.

\begin{equation} \label{eqn:CorrelationEquation}
\sum_{z=0}^{N}\sum_{y=0}^{N}\sum_{x=0}^{N}(V_1(x,y,z)-avg(V_1)) \times (V_2(x,y,z)-avg(V_2))
\end{equation}

Using this measurement, two volumes which are similar in signal shape (element-wise-value to location correspondence) will give a larger measure of correlation than two volumes with a differing signal shape. If the volumes are first normalized we can regard volume $x$ is more aligned with volume $y$ than with volume $z$ given $x$ correlated with $y$ gives a larger value than $x$ correlated with $z$. \\

Cross-correlation searches over a space of translation parameters and outputs the optimal translation to align two volumes. It is optimal in the sense of correlation measurement. This is used as a best guess in terms of the alignment of the two volumes. It can be thought of as the optimization of parameters $x,y,z$ in equation \ref{eqn:CrossCorrelationEquation}.

\begin{equation} \label{eqn:CrossCorrelationEquation}
CrossCorrelate(Transform(V_1, x,y,z), V_2)
\end{equation}

Since we typically do not know the range of translation values $x,y,z$ to optimize for, we take into account the range from $[0,N]$ where $N$ is the width/height/depth of the volume. This gives a complexity of $N^6$. This is too computationally complex for practical volume sizes. Therefore, we use the properties of the Fourier Transform to reduce computational complexity. \\



Using the properties of the Frequency domain, we can efficiently recover the optimal alignment translation parameters using a function called $PhaseCorrelation$ (Eq. \ref{eqn:PC_basic}). This function takes two volumes as input and returns the best alignment translation between them in terms of maximizing correlation.
\begin{equation} \label{eqn:PC_basic}
(x, y, z) = PhaseCorrelation(V_x, V_y)
\end{equation}
The $PhaseCorrelation$ function first applies 3D FFTs to volumes, $V_1$ and $V_2$, converting them into the frequency domain, i.e. $F_{1_{x,y,z}} = FFT(V_1)$ and $F_{2_{x,y,z}} = FFT(V_2)$. Taking the normalised cross power spectrum using Eq. \ref{eqn:PHCOR_eq} based on these frequency domain volumes computes the frequency domain of a new volume called the phase correlation volume. \\


\begin{equation} \label{eqn:PHCOR_eq}
F_{3_{x,y,z}} = \frac{F_{1_{x,y,z}} \circ F_{2_{x,y,z}}^*}{ | F_{1_{x,y,z}} \circ F_{2_{x,y,z}}^* | }
\end{equation}

Here, $\circ$ is an element-wise multiplication and $|x|$ is the magnitude function. Taking the inverse FFT of the frequency domain of the phase correlation volume, $F_3$ gives the phase correlation volume itself, $V_3$ ($V_3 = FFT^{-1}(F_3)$). The location of the peak value in the phase correlation volume $V_3$, $(x_1, y_1, z_1)$ gives the shift between the $V_1$ and $V_2$. The phase correlation volume is typically noisy making the peak difficult to locate. Each in the phase correlation volume evaluates the correlation between $V_1$ (translated by the location of the peak) and $V_2$.


\subsection{Recovery of Y-Axis Rotation and Scale}

Using the phase correlation procedure along with some spatial transformation functions allows the computation of a single axis of rotation (out of 3) along with the scale difference between two volumes. If $V_1$ and $V_2$ are translated, rotated and scaled versions of the same volume, such that they are related by some translation $(t_x, t_y, t_z)$, y-axis rotation $\theta$, and scale $\varphi$.\\


Here we describe how to compute the rotation and scale parameters, further action is required to recover translation. The first step, given two volumes $V_1$ and $V_2$ of size $N^3$ is to apply a Hanning windowing function (Eq. \ref{eqn:Hann}). This function assists in filtering the volumes prior to a discrete Fourier transform being applied to them. Since the discrete fourier transform relies on a continuous signal (volume) and this cannot be used in the calculation, by reducing the signal strength near the volume borders, we can improve the accuracy of the discrete fourier transform.
\begin{equation} \label{eqn:Hann}
\scriptstyle
HW_{x,y,z} = \frac{1}{2}\left(
1 - cos \left(
\frac{2\pi
\left(
\sqrt{\left(\frac{N}{2}\right)^3} -
\sqrt{
\left(x-\frac{N}{2}\right)^2 + \left(y-\frac{N}{2}\right)^2 + \left(z-\frac{N}{2}\right)^2
}
\right)
}
{2\sqrt{\left(\frac{N}{2}\right)^3} - 1}
\right)
\right)
\end{equation}
The rotation and scale factors are recovered first using a translation independent representation of the volumes using the Fourier shift theory. For this, the magnitude of the FFT of the volumes is taken, $M_1 = |FFT(V_1)|$, $M_2 = |FFT(V_2)|$. Since the magnitude values do not contain phase information they do not, by nature, contain location information of the volume (translation information). The zero-frequency of both $M_1$ and $M_2$ is next shifted to the center of the volume and the log of the result is taken $M'_1 = Log(M_1)$, $M'_2 = Log(M_2)$. By taking the log we can reduce spread out the contribution of the frequencies to take into account high frequencies as much as lower frequencies which have larger magnitudes. A log-spherical transform is then used to turn rotation and scaling into translation for both $M'_1$ and $M'_2$. Eq. \ref{eqn:Log_Spherical} shows the corresponding log-spherical space coordinate $(X_{log-spherical}, Y_{log-spherical}, Z_{log-spherical})$ for a given $(x,y,z)$ euclidean space coordinate.
\begin{equation} \label{eqn:Log_Spherical}
\begin{split}
X_{log-spherical} & = \frac{atan\left(
\left(\frac{x-\frac{N}{2}}{\sqrt{x^2+y^2+z^2}}\right)
\left(\frac{y-\frac{N}{2}}{\sqrt{x^2+y^2+z^2}}\right)^{-1}
\right)N}{360}\\
Y_{log-spherical} & = \frac{acos\left(
\frac{y}{\sqrt{x^2+y^2+z^2}}
\right)N}
{180} \\
Z_{log-spherical} & =\frac{log\left(\sqrt{x^2+y^2+z^2}\right)N}{log\left( \frac{N}{2.56} \right)} \\
\end{split}
\end{equation}
The log-spherical transforms of $M'_1$ and $M'_2$ are then phase correlated to find the shift between them, $(x_{M'},y_{M'},z_{M'}) = PhaseCorrelation(M'_1, M'_2)$. The rotation $\theta$ and scale $\varphi$ factors between $V_1$ and $V_2$ can then be found from the shift parameters using Eq. \ref{eqn:ROTATIONSCALEFROMXM} . 
\begin{equation} \label{eqn:ROTATIONSCALEFROMXM}
\begin{split}
\theta & = \frac{-360x_{M'}}{N}\\
\varphi & = e^{
-\left(
2.56^{-1}N
\right)z_{M'}N^{-1}
}
\end{split}
\end{equation}
Using $\theta$ and $\varphi$, $V_1$ can now be inverse transformed (using $(\frac{N}{2}, \frac{N}{2}, \frac{N}{2})$ as the origin). This aligns $V_1$ and $V_2$ with respect to scale and y-axis rotation. The translation parameters $(t_x, t_y, t_z)$ can then be found using phase correlation as given in Eq. \ref{eqn:FINALTRANS}.
\begin{equation} \label{eqn:FINALTRANS}
(t_x, t_y, t_z) = PhaseCorrelation(scale(rotate(V_1,\theta),\varphi), V_2)
\end{equation}
The complete function to recover translation, rotation and scaling, combining equations \ref{eqn:PHCOR_eq}-\ref{eqn:FINALTRANS} as is denoted in \ref{algorithm:PCSLAM} is \ref{eqn:FULLPC}.
\begin{equation} \label{eqn:FULLPC}
(\theta, \varphi, t_x, t_y, t_z) = PhaseCorrelation_{\theta \varphi t_x t_y t_z}(V_m, V_n)
\end{equation}

\subsection{Full Recovery of 3D Rotation}
