\begin{savequote}[8cm]
  ``I have not failed. I've just found 10,000 ways that won't work.''
  \qauthor{Thomas Edison}
\end{savequote}
\makeatletter
\chapter{Methodology}

\section{Introduction}

Here we present several techniques used to extract dense 3D reconstruction from image/video data. In the first section we describe techniques in addition to Fourier volume registration. These include the recovery of translation, y-axis rotation and scale information as well as a technique for the recovery of a so called 7 degrees of freedom transform. \\

We also list notable speed-ups for volume registration which reduce the amount of processing by over a third. After this, a full 3D reconstruction technique based on the principles of phase correlation is presented. Then we also present two novel reconstruction data representations which may be used to efficiently represent and fuse 3D reconstructions. Lastly, different sensor inputs are discussed and some conclusions are given about the usefulness of this technique in the context of these sensors is presented.

\section{Fourier Volume Registration} 

\subsection{Recovery of Translation Values}

Given a volume $V_1$ and a spatially shifted version of it $V_2$, the offset along each axis, $(x,y,z)$ may be recovered if a suitable correlation between the two volumes can be found. \\

The measure of correlation between $V_1$ and $V_2$ can be found by shifting $V_1$ and $V_2$'s mean values to zero, then summing the element-wise multiplication of $V_1$ by $V_2$. Equation \ref{eqn:CorrelationEquation} computes this correlation measure.

\begin{equation} \label{eqn:CorrelationEquation}
\sum_{z=0}^{N}\sum_{y=0}^{N}\sum_{x=0}^{N}(V_1(x,y,z)-avg(V_1)) \times (V_2(x,y,z)-avg(V_2))
\end{equation}

Using this measurement, two volumes which are similar in signal shape (element-wise-value to location correspondence) will give a larger measure of correlation than two volumes with a differing signal shape. If the volumes are first normalized we can regard volume $x$ is more aligned with volume $y$ than with volume $z$ given $x$ correlated with $y$ gives a larger value than $x$ correlated with $z$. \\

Cross-correlation searches over a space of translation parameters and outputs the optimal translation to align two volumes. It is optimal in the sense of correlation measurement. This is used as a best guess in terms of the alignment of the two volumes. It can be thought of as the optimization of parameters $x,y,z$ in equation \ref{eqn:CrossCorrelationEquation}.

\begin{equation} \label{eqn:CrossCorrelationEquation}
CrossCorrelate(Transform(V_1, x,y,z), V_2)
\end{equation}

Since we typically do not know the range of translation values $x,y,z$ to optimize for, we take into account the range from $[0,N]$ where $N$ is the width/height/depth of the volume. This gives a complexity of $N^6$. This is too computationally complex for practical volume sizes. Therefore, we use the properties of the Fourier Transform to reduce computational complexity.
 

 can be recovered using $PhaseCorrelation$ (Eq. \ref{eqn:PC_basic}). This function takes two volumes as input and returns the translation between them.
\begin{equation} \label{eqn:PC_basic}
(x, y, z) = PhaseCorrelation(V_m, V_n)
\end{equation}
The $PhaseCorrelation$ function first applies 3D FFTs to volumes, $V_1$ and $V_2$, converting them into the frequency domain, i.e. $F_{1_{x,y,z}} = FFT(V_1)$ and $F_{2_{x,y,z}} = FFT(V_2)$. Taking the normalised cross power spectrum using Eq. \ref{eqn:PHCOR_eq} completes the Phase correlation function. 
\begin{equation} \label{eqn:PHCOR_eq}
F_{3_{x,y,z}} = \frac{F_{1_{x,y,z}} \circ F_{2_{x,y,z}}^*}{ | F_{1_{x,y,z}} \circ F_{2_{x,y,z}}^* | }
\end{equation}
Here, $\circ$ is an element-wise multiplication and $|x|$ is the magnitude function. Taking the inverse FFT of $F_3$, gives the phase correlation volume $V_3$ ($V_3 = FFT^{-1}(F_3)$). The location of the peak value in $V_3$, $(x_1, y_1, z_1)$ gives the shift between the $V_1$ and $V_2$. The phase correlation volume is typically noisy making the peak difficult to locate. 


\subsection{Recovery of Y-Axis Rotation}

\subsection{Recovery of Scale}

\subsection{Full Recovery of 3D Rotation}


\input{chapters/chapter2/fvr2}



\section{Fourier Volume Registration based 3D Reconstruction}


In this section we describe the general technique of recovering pose estimation via Fourier volume registration techniques. Several methods may be used and each has its own advantages and disadvantages and suitability depends on pose restriction, camera accuracy, noise levels and input data. 


overall description

computing camera pose, computing error

integration into volumetric cube

advantages / disadvantages


\section{3D Reconstruction Data Representation}

\subsection{3D ShadeTree Coding}

\subsection{PlaneTree Coding}

\section{Sensor Input Techniques}


\subsection{Depth Sensor Based Reconstruction}

In this section we describe the process of projecting, registering and integrating 3D reconstructions from depth sensor based input. Depth Sensor based input is fast and reliable but such date not only requires specialized hardware, it also has several draw-backs discussed here. 

\subsection{Stereo Camera Based Reconstruction}

\subsection{Monocular Sensor Based 3D Reconstruction}

\section{Conclusion}