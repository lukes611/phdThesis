In this chapter experiments and results were presented which assist with answering the research questions from chapter \ref{ch:IntroIntroduction}. Experiment details were presented including: which type of experiments (qualitative and quantitative) were presented, which algorithms were tested, which tools were used to perform the experiments and which metrics were used in experiments. \\

Several experiments were designed to evaluate and compare the proposed Fourier based techniques (FVR, MVVR, FFVR and FVR-3D) with the current algorithms from the literature. These experiments included measuring 3D reconstruction error given stereo, active and monocular camera input, measuring error for specific camera tracking and evaluating the effects of noise to see which algorithms were most robust. \\

Results were also presented for the Plane-Tree. The Plane-Tree was evaluated and compared with the basic Octree quantitatively. It was also compared with current state-of-the-art methods both quantitatively and qualitatively. Results were also presented evaluating the Plane-Tree in regard to 3D reconstruction performance. \\

Results show that the FVR and FVR-3D methods outperform the other methods from the literature in 3D reconstruction in terms of: performance, accuracy and robustness to noise. Results for the Plane-Tree indicate it is capable of better lossy compression than the current state-of-the-art algorithms, especially at low bit-rates. In the next and final chapter (Chapter \ref{ch:Conclusion}) these results are analysed and used to answer the research questions. \\


