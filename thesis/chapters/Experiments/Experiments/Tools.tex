
In the experiments, various machines, devices and software were used to generate and perform tests. This equipment is discussed here. All experiments were performed on two machines. One is an Asus laptop running both Windows 10 and Ubuntu 16. This laptop has an Intel i7 CPU and an NVIDIA GeForce 840 M GPU with 4 GB of RAM. The other is a Dell desktop computer running Windows 7. This machine has an Intel i5 CPU and 4 GB of RAM. \\ 

Both the Visual Studio C/C++ compiler on Windows and the GCC C/C++ compiler on Ubuntu were used to write programs for testing purposes. C++ version 17 was primarily used to write programs. Libraries used include: the OPEN-CV 3 computer vision library used for capturing, writing and processing image data, and CUDA used for writing General Purpose GPU (GPGPU) programs. Both Visual Studio (12 and 15) and Code Blocks 16 were used to write programs using the libraries and compilers. The METRO 3D object comparison tool was used for comparing 3D objects in Plane-Tree experiments. \\

The Microsoft LifeCam HD-3000 passive camera was used to capture RGB video frames for testing the MVVR method. This camera was used to capture  $640 \times 480$ sized images at 30 frames per second. To capture depth (RGB-D) video frames, the Asus Xtion Pro Live active camera was used to capture $640\times480$ video frames at 30 frames per second. This camera can capture depth between 0.8 and 3.5 m and was used to capture most of the test-data used in FVR experiments. \\ 

All source code and experiments are available online. The link for the FVR method is \url{https://github.com/lukes611/phdThesis} \href{https://github.com/lukes611/phdThesis} {FVR} and the link for the Plane-Tree source code is \url{https://github.com/lukes611/PlaneTree} \href{https://github.com/lukes611/PlaneTree}{Plane-Tree}. Further discussion about the metrics used for testing these algorithms is given in section \ref{metricsSection}. \\  
