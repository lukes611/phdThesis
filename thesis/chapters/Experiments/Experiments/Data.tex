
In order to compare and analyse the FVR method as well as the MVVR and FVR-3D extensions, several real world data were captured for testing purposes. To analyse the robustness of the FVR method in terms of camera translation and rotation an indoor environment was captured using the Asus Xtion PRO live active camera. The results and descriptions for these experiments are shown in sections \ref{Sec:CamTransTrackExp} and \ref{Sec:CamRoteTrackExp}. This indoor environment contained several boxes and pieces of furniture. Therefore the data provide a useful and challenging but robust test for this 3D reconstruction method. The robustness of the FVR method in situations where moving objects impact registration is measured. Several 3D frames were captured for this experiment. These frames are captured from an indoor scene where furniture of different sizes are removed and added between frames. 

To compare and analyse the FVR method, several 3D depth frames were recorded using an 

Test Data was generated based on the testing requirements for this research. In order to robustly test different pose-estimation procedures / 3D-registration methods we generated data from a variety of scenes. These scenes are of both in-door and out-door. Some include large amounts of texture whilst others have little or no texture. Some scenes in the data-set were captured purposefully to cause confusion for algorithms which rely on texture information. These scenes have texture confusion. These scenes have multiple items which look similar locally but are actually different objects within the scene and have unique global positions. \\

Moreover, our test-data also measures the ability of each algorithm to register different camera movements. Each scene is captured by performing certain types of camera movement. Different camera transformations captured include: translation, zooming and rotation about different axes. \\

Some frames from the original test-set are shown in appendix \ref{AppendixA}. The first scene: the Apartment Texture Rotate scene was taken by rotating the camera around the y-axis across an apartment. This scene contains a lot of texture information. The Apartment Texture X Axis scene is similar in terms of texture but contains both x and y axis rotation. This tests the FVR's ability to handle multiple axes of rotation. \\

The boxes scene was captured under arbitrary-rotation, translation and zoom (in/out) camera motions. It contains some useful information on the boxes themselves, however the texture on the carpet is may cause texture confusion for feature based registration methods. The Desk Texture Translation scene contains a desk with a desktop computer and several items on it. The In-door space with texture-confusion contains a set of chairs and picture frames which look similar to each-other locally but may cause confusion for registration techniques which rely on local features.  \\

The kitchen scene was captured with both translation and zoom camera movements. It contains very little texture and the color is predominantly white. The Office textured blind-spot rotation scene is a textured office scene where the camera is rotated about the y-axis. The scene is focused on a large divider which separates two desks. The divider may confuse registration methods which rely too heavily on minimization by aligning the large divider as a priority rather than taking into account the smaller details within the scene. \\

The Office scenes contain a decent amount of usable texture and different sets were created by translating, rotating about the y-axis and rotating about the x-axis. Other office scenes, where the camera captures objects in both the foreground and the background and where the camera is lifted and rotated whilst focussed on an office desk and chair combo. \\

Some out-door scenes were also captured around the university. They are captured with both rotation and translation camera movements and are also labelled as being susceptible to texture confusion.