
In chapter \ref{ch:Metho} several methods proposed for 3D reconstruction and 3D data / 3D reconstruction compression. Four novel 3D reconstruction methods based on Fourier techniques were proposed: FVR, MVVR, FFVR and FVR-3D. In this chapter, these methods are evaluated and compared with several methods from the literature in order to answer the first research question, ``Can Fourier based registration techniques improve accuracy and noise robustness in 3D reconstruction applications?'' Chapter \ref{ch:Metho} also presented a novel hierarchical compression scheme named Plane-Tree. To answer the second research question (``Can hierarchical techniques improve compression, storage and processing of 3D reconstruction data?'') this method is compared with several others from the literature. Quantitative and qualitative results for these algorithms are analysed. Both qualitative and quantitative experiments were performed so more robust and in-depth analyses may be performed. \\

Several experiments were designed to evaluate the FVR, FFVR and FVR-3D methods as well as the Plane-Tree (and 3D Shade-Tree method). Results for these experiments and their context are presented in this chapter. In the section \ref{ToolsSection}, the tools used to run the experiments are introduced along with various error metrics (section \ref{metricsSection}) which provide means for comparisons between algorithms. \\

In experiments on reconstruction algorithms, results from three different sensor types are presented. In section \ref{StereoSOTA} FVR based techniques are compared to algorithms widely used in the literature in terms of stereo camera based 3D reconstruction. Section \ref{ActiveSOTA} presents results using active RGB sensors based 3D reconstruction and section \ref{Sec:MonocularSOTA} presents results for monocular camera based 3D reconstruction. To test stereo sensor data, the KITTI Vision Benchmark Suite was used \cite{Geiger13Vision}. For active sensor and monocular camera tests, data were specifically generated for these tests using an Asus Xtion Pro Live camera. \\

These experiments compare the performance of the FVR based techniques with the presently used algorithms in registering different types of sensor data. Results show that the FVR and FVR-3D outperform several state-of-the-art algorithms in terms of accuracy and noise robustness. Section \ref{Sec:CamTransTrackExp} presents results comparing the FVR based methods in terms of registering specific wide baseline camera movements, translations of 5-15 cm and rotations of 10-30 degrees. Results show that the FVR method can outperform others in terms of reducing registration error. \\

After these experiments, the Plane-Tree is compared to the octree. These experiments compare both methods' efficacy in compressing 3D mesh data. The Plane-Tree is also compared to various state-of-the-art compression techniques from the literature. Qualitative compression results are also presented comparing the Plane-Tree to several other methods. Finally, the Plane-Tree is compared to both the octree and the 3D-ShadeTree in compressing 3D reconstructions. \\
