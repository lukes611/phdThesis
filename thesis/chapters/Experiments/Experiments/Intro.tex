
In Chapter \ref{ch:Metho}, several methods were proposed for performing 3D reconstruction and 3D data compression. Four novel 3D reconstruction methods based on Fourier techniques were proposed: FVR, MVVR, FFVR and FVR-3D. In this chapter, these methods are evaluated and compared with several methods from the literature in order to answer the first research question, ``Can Fourier based registration techniques improve accuracy and noise robustness in 3D reconstruction applications?'' Also in Chapter \ref{ch:Metho}, a novel hierarchical compression scheme called the Plane-Tree was proposed. To answer the second research question (``Can hierarchical techniques improve compression, storage and processing of 3D reconstruction data?'') this method is compared with several others from the literature. Quantitative and qualitative results for these algorithms are analysed. Both qualitative and quantitative experiments were performed so more robust and in-depth analyses may be performed. \\

Several experiments were designed to evaluate the FVR, FFVR and FVR-3D methods as well as the Plane-Tree. Results from these experiments are presented in this chapter. In section \ref{ToolsSection}, the tools used to run the experiments are introduced along with various error metrics (section \ref{metricsSection}) which provide means for comparisons between algorithms. \\

For 3D reconstruction experiments, results from three different sensor types are presented. In section \ref{StereoSOTA}, FVR based techniques are compared to algorithms widely used in the literature in terms of stereo camera based 3D reconstruction. Section \ref{ActiveSOTA} presents results using active RGB sensors and section \ref{Sec:MonocularSOTA} presents results for monocular camera sensors. To test stereo sensor data, the KITTI Vision Benchmark Suite was used \cite{Geiger13Vision}. For active sensor and monocular camera sensor experiments, data were recorded using an Asus Xtion Pro Live camera. \\

These experiments compare the performance of these 3D reconstruction algorithms using input from different types of sensors. Results show that the FVR and FVR-3D outperform several state-of-the-art algorithms in terms of accuracy and noise robustness. Section \ref{Sec:CamTransTrackExp} presents results comparing the FVR based methods in terms of registering specific wide baseline camera movements, translations of 5-15 cm and rotations of 10-30 degrees. Results show that the FVR method can outperform others in terms of reducing registration error. \\

After these experiments, the Plane-Tree is compared with the octree. Results comparing both methods' efficacy in compressing 3D mesh data are presented. The Plane-Tree is also compared with various state-of-the-art compression techniques from the literature. Qualitative compression results are also used to compare the Plane-Tree and several other methods. Finally, the Plane-Tree is compared with both the octree and the 3D-ShadeTree in compressing 3D reconstructions. \\
