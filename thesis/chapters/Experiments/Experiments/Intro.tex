Several experiments were designed in order to evaluate both the FVR method, FFVR and FVR-3D as well as the Plane-Tree and 3D Shade-Tree method. The types of these experiments as well as their conditions and results are presented in this section. In the next section (in section \ref{ToolsSection}), the tools used to run these experiments are introduced along with various error metrics (section \ref{metricsSection}) which provide means for comparisons between algorithms. \\

In experiments on reconstruction algorithms, results from three different sensor types are presented. In section \ref{StereoSOTA} FVR based techniques are compared to algorithms widely used in the literature in terms of stereo camera based 3D reconstruction. Section \ref{ActiveSOTA} presents results using active RGB sensors based 3D reconstruction and section \ref{Sec:MonocularSOTA} presents results for monocular camera based 3D reconstruction. To test stereo sensor data, the KITTI Vision Benchmark Suite was used \cite{Geiger13Vision}. For Active sensor and monocular camera tests, data was generated specifically for these tests using an ASUS Xtion PRO Live camera. \\

These experiments compare the performance of the FVR based techniques with the presently used algorithms in registering different types of sensor data. Results show, that the FVR and FVR3D outperform several state of the art algorithms in terms of accuracy and noise robustness. Section \ref{Sec:CamTransTrackExp} presents results comparing the FVR based methods in terms of registering specific wide baseline camera movements, translations 5-15cm and rotations of 10-30 degrees. Results show that the FVR method is able to outperform others in terms of reducing registration error. \\

After these experiments, the Plane-Tree is compared to the Octree. These experiments compare both method's efficacy in compressing 3D mesh data. The Plane-Tree is also compared to various state-of-the-art compression techniques from the literature. Next, qualitative compression results are presented comparing the Plane-Tree to several other methods. Finally, the Plane-Tree is compared to both the Octree and the 3D-ShadeTree in compressing 3D reconstructions. \\
