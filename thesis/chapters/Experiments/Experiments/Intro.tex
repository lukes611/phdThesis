Several experiments were designed in order to evaluate both the FVR method, FFVR and FVR-3D as well as the Plane-Tree and 3D Shade-Tree method. The types of these experiments as well as their conditions and results are presented in this section. In the next section (section \ref{TestDataSection}) the test data used in experimentation are introduced. Following that (in section \ref{ToolsSection}), the tools used to run these experiments are also introduced along with various error metrics (section \ref{metricsSection}) which provide means for comparisons between algorithms. \\

Results from three different sensor types are presented, in section \ref{Sec:FVRStereoExperiments} FVR based techniques are compared to algorithms widely used in the literature in terms of stereo camera based 3D reconstruction, active camera based 3D reconstruction and monocular camera based 3D reconstruction. To test stereo sensor data, the KITTI Vision Benchmark Suite was used \cite{Geiger13Vision}. For Active sensor and monocular camera tests, data was generated specifically for these tests. \\

These experiments compare the performance of these FVR based techniques with the presently used algorithms in registering different types of sensor data. Results show, that the effectiveness of techniques based on FVR diminishes with the quality of the depth maps produced by these sensor types. \\

After these experiments, the Plane-Tree is compared to the Octree. These experiments compare both method's efficacy in compressing 3D mesh data. The Plane-Tree is also compared to various state-of-the-art compression techniques from the literature. Next, qualitative compression results are presented comparing the Plane-Tree to several other methods. Finally, the Plane-Tree is compared to both the Octree and the 3D-ShadeTree in compressing 3D reconstructions. \\
