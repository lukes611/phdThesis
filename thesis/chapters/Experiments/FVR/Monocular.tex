
During experiments it was found that MVVR (FVR applied to monocular camera sensor data) had considerably less accuracy and robustness compared to the other FVR methods (such as FVR, FFVR and FVR-3D) applied to more accurate depth data. Additionally, we found that MVVR was less robust to rotational transforms. This is due to the fact that depth maps produced by monocular sensor methods such as optical flow, do not produce accurate and reliable enough depth maps. Results from stereo and active camera tests suggest that this is due to the reduction in accurate and precise depth data generation. \\

Nevertheless, to evaluate the performance of the Monocular View Volume Registration (MVVR) method, quantitative experiments were performed on the Kitti Vision Benchmark Data Set. In these experiments, the MVVR method was compared against the common methods from the literature including: FM2D, FM3D. ICP and PCA.  Each depth map was projected into volume sizes of $256^3$ for processing by the rest of the MVVR method (namely the FVR part). To generate the depth maps, a local 2D block matching method was used. Kernel sizes used in the correlation procedure were $3 \times 3$ in size with a search area size of $21 \times 21$. The sizes of the kernel, search area and volume sizes were all chosen empirically. \\

Depth maps computed by the block matching method are only an estimation of the true scene depth and therefore the projection is only relative to the actual depth. However the 3D data in the registration is still spatially accurate relative to the different parts of the geometry within the scene. Comparing the noise levels between depth maps produced by block matching and the LIDAR system (figure \ref{fig:DepthGenerationExample}) we can see how much more noise the MVVR method must tolerate in order to generate a 3D reconstruction. \\

Results on the Kitti Benchmark 0001 Sync Dataset are presented in table \ref{table:MVVRQuantitativeExperimentResults}. In these experiments the ICP method outperformed the others by a successful margin. Each of the algorithms had larger errors since the depth map quality was poor. Here, FVR only performed second best achieving the best result ~29.25\% of the time. In comparison to the results between FVR-3D in table \ref{tab:kittidata0001sync}, the MVVR method dropped its competitiveness against ICP. In these experiments, due to the low quality of the projected depth data, FM2D and PCA failed completely to register the frames, and many of the algorithms had incorrectly registered frames at some point.

\begin{figure}
\centering
\begin{tabular}{ccc}
\hline
\textbf{Algorithm} & \textbf{Median Error $\times$ 1000} & \textbf{\% best results}\\ \hline
FM2D	& 3742.4 & 2.83\%\\
FM3D	& 918.05 & 14.15\%\\
ICP	& 772.48 & 50.94\%\\
PCA	& 2046.96 & 2.83\%\\
MVVR	& 944.81 & 29.25\%\\
\end{tabular}
\caption{Statistics for the Kitti Data 0001 Sync Data Set}
\label{table:MVVRQuantitativeExperimentResults}
\end{figure} 

It is expected if the quality of the depth map is improved, each algorithm would achieve a much higher level of reconstruction accuracy and the competitiveness of the FVR based methods would also improve. \\
