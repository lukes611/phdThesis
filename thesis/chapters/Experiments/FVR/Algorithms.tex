Different 3D-registration algorithms were implemented to test the Fourier Volume Registration (FVR) method. Feature matching methods are important to compare with because they are still dominant and very successful in image processing and computer vision. In this research we show that FVR is competitive with feature matching methods whilst beating them in certain contexts (such as little textured scenes or scenes where texture confusion may occur). \\ 

We test with both 2D feature matching and 3D feature matching. In 2D feature matching, the features are found and matched between a pair of 2D-images, then RANSAC is used with the corresponding matches and true 3D point to compute pose. The pose is then used to reconstruct the scene. We found that SURF performed best out of the other feature matching methods, so SURF was used in experiments. The 2D feature matching method is limited as it cannot register frames which have too few features or frames which contain texture confusion. It is also not able to handle wide base-lines. \\

We also test 3D-feature matching using an implementation of SIFT in 3D. This algorithm was tested and written in C/C++ and like the 2D counterpart, is also susceptible to failed registration in scenes with too few features and texture confusion but it is able to handle wide base-lines since it works in 3D. //

Another algorithm used in experiments is Iterative Closest Point or ICP. This method has become very popular in 3D reconstruction and works well on most scene types. One disadvantage is that this method may get stuck in a local minima and fail to register correctly. This typically occurs when registering against wide-baselines. \\

Another algorithm present in the experiments is Principal Components Analysis (PCA). This algorithm is used to find the mean and principal components of a multi-dimensional data set. This is useful for registration purposes as it works on wide-baselines, is very fast and provides additional information about a scene. The downside is that it is very susceptible to noise and misaligned data. The proposed FVR method makes use of information from PCA so it is important to compare the two to find out what improvements are made by FVR over PCA. \\

The final algorithm tested is the proposed FVR algorithm, which uses both PCA and Fourier Phase Correlation to find the registration transformation between two 3D data-sets as described in \ref{FullRecovery3DSection}. This algorithm was proposed to handle general transformations in terms of rotation, scaling and translation. It was also designed to be able to handle noisy data, data with texture-confusion and data with little or no texture. This makes it a viable option in the 3D registration and pose estimation research areas.