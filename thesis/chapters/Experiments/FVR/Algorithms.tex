
To further assess the set of proposed FVR based registration techniques, other frame registration techniques used in the literature were implemented for comparative purposes. It is important to compare the proposed algorithms with these methods because they represent the current standard in 3D reconstruction techniques. The techniques which are most competitive with the FVR based methods are the 2D Feature Matching + RANSAC method and ICP. The 2D Feature Matching + RANSAC method, discussed in section \ref{FMANDFM}, performs 2D feature matching to compare frames. Using these feature matches along with their correspondences to the 3D points within the projected depth maps, RANSAC is used to filter outliers and iteratively estimate camera location and pose. It is important to compare this method with the FVR based techniques because it is dominant within 3D reconstruction as well as in other areas of image processing and computer vision. In these tests, SURF was used with default values using OpenCV, this included using 3 octaves and selecting the top 200 feature matches. In RANSAC, a maximum of 500 iterations were used, and 4 random points were used to compute a camera pose model at each iteration. Our experiments indicate that the FVR techniques outperform this 2D Feature Matching + RANSAC method whilst remaining closed form solutions which are often more robust in certain situations (such as reconstructing scenes with little texture or scenes where texture confusion is prevalent). Through empirical tests, SURF was chosen as the optimal 2D feature matching algorithm to compare with the FVR techniques. \\ 


In addition to comparisons with 2D Feature Matching + RANSAC, 3D Feature Matching + RANSAC (discussed in section \ref{Sec:3DFMMethod}) was also implemented and compared. In 2D feature matching, the features are found and matched between a pair of 2D image frames and then RANSAC is used to compute camera pose. The camera pose is then used to reconstruct the scene. The 3D Feature Matching + RANSAC technique uses the same concept, but features are matched between the projected 3D frames rather than the 2D frames captured with standard cameras. The distinction between these approaches is that 2D feature matching is considerably faster, whilst 3D feature matching can handle wider baselines and takes into account certain spatial information which may give it an edge over the 2D feature matching approach. This advantage is greatest in scenes with less texture or repeated local textures (texture confusion). SIFT-3D was chosen as the 3D feature matching method to compare with the FVR techniques. This decision was made based on empirical testing. Due to computational power limitations, only a single octave is used, with Gaussian window sizes of $5 \time 5$. \\

The Iterative Closest Point algorithm (discussed in section \ref{ICPSection}) is another technique which has dominated 3D reconstruction and SLAM research. The ICP method works by iteratively refining a registration between two models based on each point's nearest neighbour. ICP is popular within both 3D reconstruction and SLAM research because of its accuracy and consistency. It also works well on most scene types. The major disadvantage of ICP is that it gets stuck in local extrema if there is too much noise, or too wide a base-line. In either of these cases, ICP fails to register frames. ICP is also known to fail in cases where too few features are present. In these experiments, ICP is used without pre-filtering the points and all points are used in each iteration with a maximum of 500 iterations. \\

Another algorithm present in the experiments is the Principal Components Analysis (PCA) method. PCA is used to find the mean and principal components of multi-dimensional data. The PCA method, used for registration purposes, uses the computed principal axes and mean to register two 3D frames. This method is useful for registration purposes as it works on wide baselines, is fast and provides additional information about the scene (in the form of principal components and a point representing the origin, the mean point). The downside to this method is that it is very susceptible to noise and misaligned data resulting in inaccurate and useless mean and secondary axes. As the proposed FVR-3D method makes use of the primary axis information from PCA, it is important to compare the two methods to determine what improvements are made by FVR-3D over a standard PCA approach. \\

These algorithms from the literature were compared with the set of FVR based techniques: FVR, presented in section \ref{Sec:AFVRApproach}, FVR-3D presented in section \ref{FullRecovery3DSection} and FFVR presented in section \ref{Sec:Efficiency}. These three algorithms are used to compute the registration between two frames. The FVR method is most robust to noise and has the highest accuracy in terms of registration performance at wider baselines. However, it is only capable of handling a single axis of rotation at a time. Conversely, the FVR-3D method is capable of handling full 3D rotation, translation and scale but is less accurate and robust. It is also not able to handle as wide a baseline as the FVR method. The FFVR method is also only able to handle a single axis of rotation but is faster than the FVR method at the expense of accuracy and robustness to noise. \\
