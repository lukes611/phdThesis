
In order to assess the FVR registration techniques further, other frame registration techniques used in the literature were implemented for comparative purposes. It is important to compare the FVR techniques with these methods because they represent the current standard in 3D reconstruction techniques. The technique which is thoroughly compared with the FVR methods is the 2D feature-matching + RANSAC based method. This method, discussed in section \ref{FMANDFM}, performs 2D feature matching to compare frames. Using these feature matches which correspond to 3D points within the projected depth maps, RANSAC is used to filter out outliers and iteratively estimate camera location and pose. This method is important to compare with the FVR techniques because it is still dominant within 3D reconstruction as well as in other areas of image processing and computer vision. Our experiments indicate that the FVR techniques outperform this 2D feature matching + RANSAC method whilst remaining closed form solutions which are often more robust in certain situations (such as reconstructing scenes with little texture or scenes where texture confusion is prevalent). Through empirical tests, SURF was chosen as the optimal 2D feature matching algorithm to use to compare with the FVR techniques. \\ 


In addition to comparisons with 2D feature matching + RANSAC, 3D feature matching + RANSAC (discussed in section \ref{Sec:3DFMMethod}) was also implemented and compared. In 2D feature matching, the features are found and matched between a pair of 2D-image frames and then RANSAC is used to compute camera pose. The camera pose is then used to reconstruct the scene. The 3D feature matching + RANSAC technique uses the same concept, but features are matched between the projected 3D frames rather than the 2D frames. The distinction between these approaches is that 2D feature matching is considerably faster, whilst 3D feature matching can handle wider baselines and takes into account certain spatial information which may give it an edge over the 2D feature matching approach, especially in scenes with less texture or repeated local textures (texture confusion). SIFT-3D was chosen as the 3D feature matching method of choice in comparing this technique with the FVR techniques. This decision was made based on empirical testing. \\

The Iterative Closest Point algorithm (discussed in section \ref{ICPSection}) is another technique which has dominated 3D reconstruction and SLAM research. The ICP method works by iteratively refining a registration between two models based on each point's nearest neighbour. ICP is popular within both 3D reconstruction and SLAM research because of its accuracy and consistency, additionally it works well on most scene types. The major disadvantage of this method is that it gets stuck in local minima if there is too much noise, or too wide a base-line. In either of these cases, ICP fails to register frames. ICP is also known to fail in cases where too few features are present. \\

Another algorithm present in the experiments is the Principal Components Analysis (PCA) method. PCA is used to find the mean and principal components of multi-dimensional data. The PCA method used for registration purposes, uses the computed principal axes and mean to register two 3D frames. This method is useful for registration purposes as it works on wide-baselines, is fast and provides additional information about the scene (in the form of principal components and mean point). The downside  to this method is that it is very susceptible to noise and misaligned data. The presence of which makes the mean and secondary axes inaccurate and useless. The proposed FVR-3D method makes use of the primary axis information from PCA so it is important to compare the two to find out what improvements are made by FVR-3D over a vanilla PCA approach. \\

In order to test the FVR approaches (FVR discussed in section \ref{Sec:AFVRApproach} and FVR-3D discussed in section \ref{FullRecovery3DSection}), both algorithms are used to compute the registration between frames. The registration with the lowest error margin is used for comparisons with other methods, this combines the FVR accuracy with the FVR-3D's ability to register arbitrary 3D rotation. The combined FVR algorithm is proposed to handle general transformations in terms of rotation, scaling and translation whilst still being a closed form solution which is robust to large amounts of data noise, texture-confusion and scenes with a considerable lack of texture. This makes it a viable alternative in the 3D registration, 3D reconstruction and SLAM research areas. \\
