
\subsection{Noise Robustness}


Here we present results for experiments comparing the level of noise robustness between each pose registration technique. These tests were performed on data captured via the ASUS Xtion PRO LIVE camera (presented in section \ref{TestDataSection} with example frames are shown in Appendix \ref{AppendixA}). Frames from these data-sets are transformed by various transformations with varying amounts of noise added. This noise is measured in both SNR and noise range. A noise range of 1.0 means random noise within the range [$-0.5$, $0.5$] was added. SNR is measured in decibels. \\

The accuracy of each algorithm is computed using the percent of points matched to a nearest neighbour post registration, for example 100\% match would be equivalent with a perfect registration. This metric was discussed in section \ref{TestDataSection}. Transformations tested include both y-axis rotation transformations and translation based ones. The y-axis rotation results were generated by registering frames which were separated by degrees ranging from 0 to 360 about the y-axis. Each algorithm's percent match results were then averaged over the 360 degrees. The translation experiment was performed by registering translations from 0 to 120 voxels. Again, for each algorithm the percent match results are averaged. \\


\subsubsection{Rotation Registration}

%yr-rotation
\begin{table}[!htb]
\centering
\scalebox{1.0}{
\begin{tabular}{ccccccc}
\hline
\textbf{SNR} & \textbf{Noise Range} & \textbf{FM-2D} & \textbf{FM-3D} & \textbf{ICP} & \textbf{PCA} & \textbf{FVR}\\ \hline
$\infty$ & 0 & 100\% & ~6.85\% & ~5.61\% & ~29.4 & 100\%\\
10,300 & 1 & ~77.26\% & ~6.14\% & ~8.55\% & ~29.4\% & 100\%\\
2,580 & 2 & ~93.11\% & ~6.73\% & ~8.43\% & ~29.4\% & 100\%\\
\end{tabular}}
\\
\caption{Average percent matched registration results for data rotated about the y-axis from 0 to 360 degrees under varying noise conditions.}
\label{table:YRNoiseT}
\end{table}


In the first set of experiments, scanned 3D data was rotated about the Y-axis with zero noise added. Table \ref{table:YRNoiseT} row 1 shows the average percent match results for each algorithm with an infinite SNR (no noise added). The 2D-FM method and FVR achieved perfect results with PCA performing next best with an average of ~30\% registration alignment. Both FM-3D and ICP performed poorly. In this context, FM-3D needs larger volume sizes to perform more accurately, which comes at a cost computation wise, and ICP cannot handle larger rotations (> 10 degrees) easily. \\

For the results (also in figure \ref{table:YRNoiseT}) in row 2 the noise was dropped to a SNR of $10300$ (noise range of 1.0). Here, similar results are seen. For FM-2D however, performance dropped from 100\% to ~77\%. In row 3 the SNR was reduced to $2580$ for which 2D-FM surprisingly increased in performance. Again for both row 2 and row 3 the FVR achieved the highest performance result. These results suggest that for Y-axis rotation FVR is superior in terms of both noise robustness and wide base-line registration.  \\

\subsubsection{Translation Registration}

%translation
\begin{table}[!htb]
\centering
\scalebox{1.0}{
\begin{tabular}{ccccccc}
\hline
\textbf{SNR} & \textbf{Noise Range} & \textbf{FM-2D} & \textbf{FM-3D} & \textbf{ICP} & \textbf{PCA} & \textbf{FVR}\\ \hline
$\infty$ & 0 & 100\% & 100\% & ~23.83\% & 100\% & 100\%\\
10,300 & 1 & 100\% & 100\% & ~9.16\% & 100\% & 100\%\\
2,580 & 2 & 100\% & 100\% & ~23.24\% & 100\% & 100\%\\
\end{tabular}}
\\
\caption{Average percent matched registration results for data translated 0 to 140 voxels under varying noise conditions.}
\label{table:TNoiseT}
\end{table}


The effects of noise on translation were also tested in these experiments. In the first experiment, 3D frames were translated by varying levels of x-axis translation. In this experiment, different 3D reconstruction techniques are compared using the percent match metric, to register translation transforms from 0 to 140 (in a frame space of $256^3$). Results are shown in table \ref{table:TNoiseT}. \\

In this table, row 1, the 2D-FM, 3D-FM, PCA and FVR methods were shown to have perfect accuracy. This is expected as there is zero noise added prior to the registration of these 3D frames. ICP can be seen falling in terms of accuracy in translations with too wide a base-line (larger than 10 voxels in a frames space of 256 voxels). In this case, a possible solution may be to introduce scale space registration to ICP at an increase in computational complexity. \\  

For experiments in row 2 and 3 the SNR was reduced to $10300$ and $2580$ respectfully. Again, with the exception of ICP, each algorithm seems to be equally robust at registering again translation at different levels of noise. Again, ICP is shown to be unable to handle wider baselines. These experiments show that when registering camera translation, each algorithm is equally robust to noise and can handle wide-baselines with the exception of ICP. \\
