


Experiments show that the FVR based techniques are capable of registration across the three primary types of sensors used in 3D reconstruction and SLAM research. These sensors include: Stereo sensors, which are capable of generating dense depth data with the highest accuracy and resolution, active cameras which are efficient hardware based solutions to depth generation and monocular cameras, which produce the least accurate and most noisy depth maps. As mentioned in the literature review these sensor types have associated advantages and disadvantages. \\

In section \ref{StereoSOTA}, results testing the stereo Kitti dataset are presented. These show that FVR based methods, particularly FVR3D are suitable for dense 3D reconstruction and are highly robust to noise, object movement and outdoor scenes. In particular it is shown that the FVR3D method is capable of outperforming other algorithms used in the literature in terms of registration accuracy. \\

Section \ref{ActiveSOTA} presents results testing scenes captured using the ASUS Xtion PRO LIVE active camera. Various types of scenes and camera movements were recorded and results show the FVR based methods perform robust and accurate 3D reconstruction. In particular it is shown that FVR3D can outperform other techniques from the literature, even when faced with a reduction in depth resolution when using the active camera as opposed to the stereo set-up tested with the Kitti Vision Benchmark Dataset.  \\

Lastly, section \ref{Sec:MonocularSOTA} presents results on the MVVR method, the FVR extension for monocular sensor input datasets. Experiments have revealed that the MVVR is not able to robustly handle rotation due to the noise generated when building the depth map, however translation is possibly. It can be said based on empirical results that the FVR based technique's accuracy and robustness increases with the resolution and accuracy of the depth maps used as input. \\
