\begin{savequote}[8cm]
  ``It is not knowledge, but the act of learning, not possession but the act of getting there, which grants the greatest enjoyment''
  \qauthor{Carl Friedrich Gauss}
\end{savequote}
\makeatletter
\chapter{Introduction}

\section{Introduction}

3D Reconstruction research requires the development, testing and analysis of functions which input video and image data and output 3D reconstructed environments. This area is very similar to Simultanious Localization and Mapping or SLAM. However, we have separated the areas as SLAM does not nececarily care about the full dense reconstruction of 3D data. It also has an added requirement of computing localization information. In 3D reconstruction, as long as pleasant, dense and useful 3D reconstructions are computed, localization does not matter.

3D reconstruction is imprortant in a wide variety of areas including business, engineerign and architecture, virtual reality and augmented reality. For example, an architect may want to record 3D structural data in order to study it later. An engineer may want to study the under area of a bridge in order to assess possible faults. Or a software engineer may want to create an augmented reality application where possible home buyers can take virtual tours through an existing property. The scpecifications and 3D structures of these areas may be recorded, in which case 3D models may be built by artists. This however, costs both time and money, furthermore there may be no existing blueprints for the structures, we may also want to provide virtual walking tours through a rainforest, this definately has no blueprint.

Using image and video capturing hardware coupled with 3D reconstruction software, we would be able to scan in an invornmemtn and generate a dense 3D model of it for use in egineering/architectural analysis as well as any kind of virtual reality application. Furthermore, autonomous navigation systems may also generate and use this information as they navigate through a previously unknown environment. Furthermore, with the recent progress made in 3D printing, we may come across a situation where 3D objects and environments may be scanned in using 3D recosntruction and copied via 3D printing.


Without such a system, artists, architects, engineers or scientists would have to build, draw or find some alternative means of generating the 3D data they require. This costs much time and money.

As in other areas of image processing, 3D reconstruction is dominated by feature matching and RANSAC techniques. This involves computing matches between 2D pixels across images, these matches are typically used with RANSAC to compute a camera relationship between the frames. 3D data is then projected and registered using the relationship. 

In the past few decades, transform and wavelet methods have dominated image compression. Lately, novel hierarchical methods have provided state of the art compression performance with reduced complexity. These techniques have not been extended to 3D model compression, which raises the question of how well these techniques may work if applied to 3D data. 

\section{Research Aims \& Contributions}

The primary aim of this research is to improve the storage, quantitative quality and perceptual quality of 3D models. To this end, hierarchical image compression schemes were investigated. This led to a new image codec called the Interpolating Leaf Quadtree (ILQT) \cite{Lincoln13Interpolating} which outperforms JPEG at low bit rates, another codec called the ShadeTree (ST) was also investigated. The techniques explored by the ST and ILQT were fundamentally different from wavelet and transform based codecs, yet still led to good results. This motivated the research question, ``Do hierarchical techniques bring state of the art compression performance to 3D data?'' The Shade-Octree 3D model codec was developed to answer this question and to accomplish the primary aim of this research. 

\section{Overview}

Chapter two presents a survey of 3D model representation and compression methods. Following this, chapter three introduces the SOT, which is used to answer the research question and accomplish the primary aim of this project. The fourth chapter details the experiments performed, then presents and analyses the results of these experiments. Finally, the sixth chapter concludes the thesis and discusses the results in terms of the primary aim and research question. 


