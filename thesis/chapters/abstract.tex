\makeatletter
\chapter{Abstract}

3D reconstruction algorithms generate 3D data from image or video data. The current focus of this research area is on iterative algorithms such as: feature matching/RANSAC, Iterative Closest Point, and other non-linear optimization strategies. These strategies tend to fail in scenes with few features or scenes which contain feature confusion. In 2D image registration research, feature matching is dominant but closed solution based Fourier registration techniques have been proven to outperform them with increased robustness to noise and low textured scenes. In this research, Fourier Volume Registration was explored in order to document the effects of 3D reconstruction and registration. Results are compared between Fourier Volume Registration, and several current techniques both quantitatively and qualitatively. Results show that the Fourier Volume Registration Technique often outperforms other methods in terms of minimizing registration error prior to optimization. Furthermore it is a closed solution which works well with parallel processing architectures. In conjunction, 3D data representations for 3D reconstruction data were also explored in order to improve storage and transmission of such data. Many current methods make use of Signed Distance Functions, volumetric occupancy grids or Octrees. Unlike previous work, lossy Octree compression is analysed. This direction paves the way for new storage and transmission rates of efficiency. A novel method called the Plane-Tree is proposed. The findings presented on both the Fourier Volume Registration method and the Plane-Tree indicate an improvement over existing methods and may lead to new research into the areas of Fourier registration and Hierarchical data representation research.


