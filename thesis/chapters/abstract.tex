\makeatletter
\chapter{Abstract}

In recent years, there has been a resergence of research and applications within the areas of human computer interaction, these include: virtual reality, augmented reality and 3D reconstruction. This thesis concentrates on the area of 3D reconstruction. In this domain, image and video data is processed to collect 3D structural information. This information has many applications in virtual reality, engineering, architecture and business. Currently, there exist many methods which are capable of extracting 3D structural information from image and video data. Methods include: Feature Matching, Principal Components Analysis and Iterative Closest Point. These algorithms work well in simplistic environments where data noise and corruption is of no concern. Experiments reveal that using Fourier based registration can also recover 3D structural information by registering depth maps. This approach is shown to be robust to noise and object movement. It is capable of solving 1 axis of rotation as well as scale and translation. In cases where 3D rotation must be registered, a novel PCA/Fourier registration methods is proposed. Experiments show that in terms of registration accuracy, this method improves over ICP, feature matching and basic PCA, especially in the presense of noise. During experiments, it was discovered that 3D reconstruction data uses large amounts of storage. 3D data compression methods were also researched. During the PhD, a novel 3D compression method was developed. Results show this outperforms several state of the art methods in terms of low bit rate compression. This is useful for the fourier based registration methods as they are robust to data noise.