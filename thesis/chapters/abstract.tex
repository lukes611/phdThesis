\makeatletter
\chapter{Abstract}

3D reconstruction algorithms generate 3D data from image or video data. The current focus of this research area is on feature matching/RANSAC or error function minimization techniques (such as ICP). In 2D image registration, feature matching is dominant but Fourier based registration techniques have been proven to outperform them with increased robustness to noise. In this research, Fourier Volume Registration was explored to see if 3D reconstruction/registration results may be improved. We compared results from several current techniques with Fourier Volume Registration quantitatively. Results show that the Fourier Volume Registration Technique often outperforms other methods in terms of minimizing registration error. Furthermore it is a closed solution which works well with parallel processing architectures. In conjunction, we also explored 3D representations for 3D reconstruction data and developed some novel techniques which outperform current methods. Most methods use either a SDF or OT for representation of the data. These representations may be compressed and are efficient for storage and transmission, however we propose a novel method called the Plane-Tree which compresses data further. The findings presented on both the Fourier Volume Registration method and the Plane-Tree indicate an improvement over existing methods and may lead to new research into the area of Fourier based registration.


