\makeatletter
\chapter{Abstract}

Three-dimensional reconstruction algorithms generate 3D data from two-dimensional image or video data. The current focus of this research area is on iterative algorithms such as: feature matching/RANSAC, Iterative Closest Point, and other non-linear optimization strategies. These strategies tend to fail in scenes with few features or scenes which contain feature confusion. In 2D image registration research, feature matching is dominant but closed form solution based Fourier registration techniques have been proven to outperform them with increased robustness to under textured scenes and image noise. This thesis investigates the application of Fourier Volume Registration to 3D reconstruction. Results are compared between Fourier Volume Registration, and several current techniques both quantitatively and qualitatively to find out if Fourier based techniques outperform iterative ones. Results show that the Fourier Volume Registration Technique often outperforms other methods in terms of minimizing registration error prior to optimization. Furthermore it is a closed form solution which works well with parallel processing architectures. 3D data representations for 3D reconstruction data are also explored to improve storage and transmission of such data. Many current methods make use of Signed Distance Functions, volumetric occupancy grids or octrees. In the work presented here, lossy octree compression is analysed to pave the way for new storage and transmission rates of efficiency. A novel method, called the Plane-Tree, is proposed based on the octree compression method. This Plane-Tree data structure was inspired by work completed in the author's honours thesis. When compared to the original octree data structure, the Plane-Tree is optimal in terms of compression performance. The findings presented on both the Fourier Volume Registration method and the Plane-Tree indicate improvements over existing methods.


