\begin{savequote}[8cm]
  ``It is not knowledge, but the act of learning, not possession but the act of getting there, which grants the greatest enjoyment''
  \qauthor{Carl Friedrich Gauss}
\end{savequote}
\makeatletter
\chapter{Introduction}

\section{Introduction}

3D Reconstruction research requires the development, testing and analysis of functions which input video and image data and output 3D reconstructed environments. This area is very similar to Simultanious Localization and Mapping or SLAM. However, we have separated the areas as SLAM does not nececarily care about the full dense reconstruction of 3D data. It also has an added requirement of computing localization information. In 3D reconstruction, as long as pleasant, dense and useful 3D reconstructions are computed, localization does not matter.

3D reconstruction is imprortant in a wide variety of areas including business, engineerign and architecture, virtual reality and augmented reality. For example, an architect may want to record 3D structural data in order to study it later. An engineer may want to study the under area of a bridge in order to assess possible faults. Or a software engineer may want to create an augmented reality application where possible home buyers can take virtual tours through an existing property. The scpecifications and 3D structures of these areas may be recorded, in which case 3D models may be built by artists. This however, costs both time and money, furthermore there may be no existing blueprints for the structures, we may also want to provide virtual walking tours through a rainforest, this definately has no blueprint.

Using image and video capturing hardware coupled with 3D reconstruction software, we would be able to scan in an invornmemtn and generate a dense 3D model of it for use in egineering/architectural analysis as well as any kind of virtual reality application. Furthermore, autonomous navigation systems may also generate and use this information as they navigate through a previously unknown environment. Furthermore, with the recent progress made in 3D printing, we may come across a situation where 3D objects and environments may be scanned in using 3D recosntruction and copied via 3D printing.


Without such a system, artists, architects, engineers or scientists would have to build, draw or find some alternative means of generating the 3D data they require. This costs much time and money.

As in other areas of image processing, 3D reconstruction is dominated by feature matching and RANSAC techniques. This involves computing matches between 2D pixels across images, these matches are typically used with RANSAC to compute a camera relationship between the frames. 3D data is then projected and registered using the relationship. This approach is efficient but is not robust to data noise. Furthermore, without some outlier removal function to filter matches, this method fails as feature detection methods typically over-compute matches to the point that only around 30 - 50 percent of features are actually matched. This method also fails in cases where large baselines are used, affine distortion is too large, feature confusion occurs or other times when feature matching fails. Another popular method Iterative Closest Point can be used. This method is more robust to failure than feature matching, and can also be used with RANSAC. However, it fails when the search reaches a local minima in terms of matching error.

Despite the apparent flaws in these methods, they are still popular in both research and industry. In image processing however, alternatives exist. One in particular: Fourier based registration works well at computing 2D rotation, scale and translation. The benefits of this technique are that it is robust to noise and outliers as it takes into account the full signal (it uses the frequency domain to perform fast correlation of data). It is also a closed form sollution (its speed does not depend on the amount of features or on the data itself) and lends itself more easily to parallel processing, a paradime which is set to take over the next generation of software and hardware. Such techniques are frequently used in medical image processing. This raises the question of how well such a technique or related set of techniques would apply to the area of 3D reconstruction.

During the research conducted in quest of answering this question, it was found that the storage and thus manipulation of 3D data became a bottleneck for database and network based operations to do with 3D reconstructed data. To alleviate this issue, compression techniques were also analysed. A set of novel techniques for compression and storage of 3D data are also proposed in this work.  

\section{Research Aims \& Contributions}

The primary aim of this research is to improve the accuracy, noise robustness, speed and storage, quantitative quality and perceptual quality of 3D models. To this end, fourier based registration schemes were investigated as well as compression systems. This motivated the research question, ``Can Fourier based registration techniques improve accuracy and noise robustness in 3D reconstruction applications?'' and ``Can hierarchical techniques improve compression, storage and processing of 3D reconstruction data?'' 

The quest to answer these questions has led to new 3D registration techniques \cite{Lincoln13Interpolating} which outperform ICP and feature matching approaches in terms of noise robustness and accuracy. A novel compression scheme was also developed which was shown to be capable of outperformign existing schemes, this method may be applied to 3D reconstruction storage and retrieval applications and research.

\section{Overview}

Chapter two presents a survey of techniques which may be used for 3D reconstruction. Following this, chapter three introduces some proposed techniques. These are used to answer the research questions and accomplish the primary aim of this project. The fourth chapter details the experiments performed, and presents both quantitative and qualitative results. The fifth chapter contains an analysis of these results and presents findings and discusses results. Finally, the sixth chapter concludes the thesis and discusses the results in terms of the primary aim and research question. 


