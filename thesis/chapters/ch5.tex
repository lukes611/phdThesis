\begin{savequote}[8cm]
  ``Learn from yesterday, live for today, hope for tomorrow. The important thing is not to stop questioning.''
  \qauthor{Albert Einstein}
\end{savequote}
\makeatletter
\chapter{Conclusion}

\section{Research Aims Revisited}

The primary aim of this research was to improve the storage, quantitative quality and perceptual quality of 3D models. Hierarchical methods were investigated and this led to a new image codec, ILQT \cite{Lincoln13Interpolating} as well as the proposed scheme, the Shade-Octree. The research question was, ``Do hierarchical techniques bring state of the art compression performance to 3D data?'' Results from chapter 4 show that the SOT outperforms the OT compression scheme as well as several state of the art methods in terms of rate-distortion and perceptual quality. The primary aim of this research has been accomplished because the SOT produces models which have improved storage efficiency, whilst improving quantitative quality and perceptual quality compared with other compression schemes. The proposed research question can also be answered. The SOT, a hierarchical codec, does achieve state of the art performance in the field of 3D data compression.

\section{Future Work}

There is still much work to be done in the area of hierarchical 3D data compression. One possible project could be to improve the SOT. Currently the SOT can produce holes in the output mesh, an investigation of this may lead to improved results. Other projects could investigate the extension of the PJQ, wedglet, BSP-tree and QB tree to 3D model compression.
