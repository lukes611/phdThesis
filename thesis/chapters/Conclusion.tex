\begin{savequote}[8cm]
  ``Learn from yesterday, live for today, hope for tomorrow. The important thing is not to stop questioning.''
  \qauthor{Albert Einstein}
\end{savequote}
\makeatletter
\chapter{Conclusion}
\label{ch:Conclusion}

In this thesis, novel techniques were proposed which improve the accuracy, noise robustness, speed and storage, quantitative quality and perceptual quality of 3D reconstructions. The FVR and FVR-3D methods are accurate and robust to noise. Both quantitative and qualitative results show these methods outperform existing techniques from the literature \ref{ch:Experiments}. The FFVR (Fast FVR) method improves upon the computational performance of the FVR. Although the FFVR achieves great performance gains at the cost of accuracy and robustness, it was still shown to be competitive. MVVR was proposed, allowing the FVR algorithm to work with monocular data to mixed results. To reduce storage requirements for 3D reconstructed models during processing and in general, a novel 3D data compression system was presented names, Plane-Tree. The Plane-Tree data representation is based on octree decomposition. The Plane-Tree is able to compress models at low bit-rates more effectively than several state-of-the-art algorithms. Further details on the findings made during this research are presented in forthcoming sections. \\

\section{Fourier Volume Registration}
\label{Sec:ConcFVR}

FVR and FVR-3D were developed to be accurate and noise robust closed form solutions to the 3D frame registration problem. Quantitative and qualitative results show the FVR-3D and FVR methods are capable of producing accurate 3D reconstructions given stereo or active sensor data. In a majority of experiments on data captured using active and stereo sensors, FVR-3D achieved the best quantitative performance.  When rectifying larger camera translations and rotations, the FVR method often performed best. Therefore, when camera displacement is large, FVR is the best choice for 3D reconstruction. \\

The FFVR performs similarly to the FVR but trades accuracy for speed. A theoretical speed analysis shows the FFVR method approaches the speed of PCA and is three times faster than next quickest algorithm, FVR. Despite having reduced accuracy compared with the FVR, FFVR is shown to have outperformed PCA in terms of average registration error in three out of five stereo experiments and seven out of seven active sensor experiments. MVVR is used to register 3D projections computed from monocular data. Results suggest that the MVVR has reduced performance compared with the FVR method because of the low quality depth data produced by monocular depth extraction techniques. If computed depth map accuracy approaches the quality of laser measured depth, MVVR's performance approaches FVR's. \\


\section{Plane-Tree 3D Data Compression}
\label{Sec:ConcPT}

The Plane-Tree 3D data compression scheme, based on the octree data structure, reduces the storage and transmission requirements of 3D reconstructed models and general 3D data. Quantitative experiments show the Plane-Tree improves upon the octree data representation technique in terms of compression. It also outperforms several state-of-the-art 3D model compression algorithms from the literature. When compressing 3D reconstructed data, Plane-Tree has better compression ability compared with the octree data structure. \\

\section{Research Aims Revisited}

The primary aim of this research is to improve the accuracy, noise robustness, speed and storage, quantitative quality and perceptual quality of 3D models generated from image data. The FVR, FFVR, FVR-3D and MVVR methods, as well as the Plane-Tree 3D data representation method, were proposed to achieve this aim. The research questions were: \\

1. ``Can Fourier based registration techniques improve accuracy and noise robustness in 3D reconstruction applications?'' \\

and \\

2. ``Can hierarchical techniques improve compression, storage and processing of 3D reconstruction data?'' \\


Results in Chapter \ref{ch:Experiments} show that FVR, FFVR, MVVR and FVR-3D are accurate, robust to noise and produce qualitatively pleasant 3D reconstructions. Quantitative experiments show that the FVR method improves upon existing solutions in terms of accuracy and noise robustness on especially when performing wide-baseline registration. The FVR-3D algorithm outperforms ICP, FM2D, FM3D and PCA when registering frames from active sensors and stereo camera pairs. Therefore, the answer to Question 1. is yes, Fourier based registration techniques can and do improve accuracy and noise robustness in 3D reconstruction applications. \\


Results show, hierarchical method Plane-Tree improves upon existing state-of-the-art methods compressing 3D models and outperforms the octree in compressing 3D reconstructions. Therefore, the answer to the second research question is yes, hierarchical techniques improve compression, storage and processing of 3D models, including reconstruction data. \\

\section{Future Work}

There are still many topics in 3D Fourier reconstruction and hierarchical data storage, processing and transmission to be researched. Further development of the FVR technique is required as presently, the system does not have recovery functionality for when individual frame registration fails. One possible topic is the effects of filtering registration data using a Kalman filter to improve reconstruction performance. There are also many hierarchical based representation variants to investigate. Investigating hierarchically compressing frames prior to model integration, would be another interesting topic. \\
